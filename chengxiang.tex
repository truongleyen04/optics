\subsection{几何光学成像规律}
\subsubsection{成像基本概念}
\insertfigure{picture/6.png}{不同的成像种类}{fig:6}
\noindent 实(虚)物: 物点发射出发散(会聚)的同心光束。

\noindent 实(虚)像: 像点形成自会聚(发散)的同心光束。
\subsubsection{共轴球面光具组傍轴成像}
单个球面的情况下,有:
\begin{equation}
    \frac{n}{s} + \frac{n'}{s'} = \frac{n'-n}{r} = \Phi \label{eq:1}
\end{equation}

$ \Phi$为光焦度
由于
\begin{align*}
f &= \frac{n}{n'-n}r = \frac{n}{\Phi} &
f' &= \frac{n'}{n'-n}r = \frac{n'}{\Phi}
\end{align*}
则可将\eqref{eq:1}化简为:
$$\frac{f}{s} + \frac{f'}{s'} = 1$$

\subsubsection*{傍轴物点成像和横向放大率}
\insertfigure{picture/7.png}{物点成像}{fig:7}
\noindent $\Pi$上所有的点都成像在$\Pi'$上(当然只限于\textbf{\color{red}傍轴区域})。

\noindent 这样一对由共轭点组成的平面叫做\textbf{\color{red}共轭平面},

\noindent 其中$\Pi$叫\textbf{\color{red}物平面},$\Pi'$叫\textbf{\color{red}像平面}。

\noindent 注意:
\noindent 若$P$ (或$P'$)在光轴上方, $y$ (或$y'$)$>0$;

\noindent \quad \quad \quad 在光轴下方, $y$ (或$y'$)$<0$。 % \quad 用于模拟图片中的缩进效果

\vspace{1em}
定义横向放大率为 $V = y'/y$

可导出:
$$V = - \frac{ns'}{n's}$$

对于:
\insertfigure{picture/8.png}{多个同轴球面}{fig:8}

有$V = V_1 V_2 V_3 \dots$

\subsubsection*{拉格朗日-亥姆霍兹定理}
\insertfigure{picture/9.png}{示意图}{fig:9}
在傍轴条件下,
\begin{center}
    \boxedmath{$ynu = y'n'u'$}
\end{center}
推广到多个共轴球面系统,
\begin{center}
    \boxedmath{$ynu = y'n'u' = y''n''u'' = \dots$}
\end{center}
\subsubsection{薄透镜及透镜组成像}
\insertfigure{picture/10.png}{薄透镜}{fig:10}
薄透镜要求:
$$d \ll f, f', s, s'$$
对于薄透镜的前后两个面,有
\begin{align*}
    \Sigma_1: \quad \frac{f_1}{s_1} + \frac{f'_1}{s'_1} &= 1 &
    \Sigma_2: \quad \frac{f_2}{s_2} + \frac{f'_2}{s'_2} &= 1
\end{align*}
令
\begin{align*}
  f &= \frac{f_1 f_2}{f'_1 + f_2}, &
f' &= \frac{f'_1 f'_2}{f'_1 + f_2} &
\end{align*}
两式合并,可得
\begin{center}
    \boxedmath{$ \frac{f}{s} + \frac{f'}{s'} = 1 $}
\end{center}

一般情况下,$n = n' \approx 1$
则
$$ 
\Phi = \frac{n_L - n}{r_1} + \frac{n' - n_L}{r_2} = (n_L - 1)\left(\frac{1}{r_1} - \frac{1}{r_2}\right) 
$$
而
$$
\left\{
\begin{aligned}
f &= \frac{n}{\frac{n_L - n}{r_1} + \frac{n' - n_L}{r_2}} = \frac{n}{\Phi} \\
f' &= \frac{n'}{\frac{n_L - n}{r_1} + \frac{n' - n_L}{r_2}} = \frac{n'}{\Phi}
\end{aligned}
\right.
$$
则
$$ f = \frac{n}{\Phi} = \frac{n'}{\Phi} = f' = \frac{1}{(n_L - 1)\left(\frac{1}{r_1} - \frac{1}{r_2}\right)} $$
此公式被称为\textbf{磨镜者公式}
\vspace{1em}
透镜度数和光焦度的关系为
$$ \text{度数} = P \times 100 $$
\subsubsection*{牛顿公式}
\insertfigure{picture/11.png}{牛顿公式示意图}{fig:11}

引入$x$
$$ x = s - f, \quad x' = s' - f' $$
推导可得
$$ V = -\frac{s'}{s} = -\frac{f'}{x} = -\frac{f}{x'} = -\frac{x'}{f'} = -\frac{x}{f} $$
这便是\textbf{薄透镜横向放大率公式}。
\vspace{1em}
\subsubsection*{密接透镜组}



