% 声明文档类型为 ctexart,这是专门为中文排版设计的文章类
\documentclass{ctexart}

% --- 引入宏包 (Packages) ---
% geometry 宏包用于设置页面边距
\usepackage{geometry}
\geometry{a4paper, left=2.5cm, right=2.5cm, top=2.5cm, bottom=2.5cm}

% amsmath 宏包提供了强大的数学公式排版功能
\usepackage{amsmath}

% graphicx 宏包用于插入图片
\usepackage{graphicx}


% --- 文档信息 ---
\title{高斯光束的传输特性分析}
\author{张三}
\date{\today} % \today 会自动生成当天的日期


% --- 正文开始 ---
\begin{document}

\maketitle % 这条命令会根据上面的标题、作者、日期信息生成标题页

\section{引言}

本文档旨在演示如何在 Visual Studio Code 中使用 LaTeX 编辑一份包含中文、英文和数学公式的简易光学报告。
高斯光束 (Gaussian Beam) 是光学中一种非常重要的激光束模型。

\section{基本理论}

高斯光束的束腰半径 $w_0$ 和瑞利范围 (Rayleigh range) $z_R$ 是描述其特性的两个核心参数。它们之间的关系如下:
$$
z_R = \frac{\pi w_0^2}{\lambda}
$$
其中 $\lambda$ 是光的波长。

当光线从一种介质入射到另一种介质时,会发生折射。其路径由斯涅尔定律 (Snell's Law) 决定:
$$
n_1 \sin \theta_1 = n_2 \sin \theta_2
$$
这里:
\begin{itemize}
    \item $n_1$ 和 $n_2$ 分别是两种介质的折射率。
    \item $\theta_1$ 和 $\theta_2$ 分别是入射角和折射角。
\end{itemize}

\section{实验光路图}



% \begin{figure}[h!]
% \centering
% \includegraphics[width=0.8\textwidth]{optical_setup.png}
% \caption{实验光路图示例}
% \label{fig:setup}
% \end{figure}


\end{document}
% --- 正文结束 ---