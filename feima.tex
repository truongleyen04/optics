


% --- 正文章节 ---
\section{几何光学基本原理与成像规律}
\subsection{费马原理}
\subsubsection{惠更斯原理与折射率}

% 使用 setspace 提供的环境来设定局部行距,1.5 约等于双倍行高。
% 2.5 的行距在学术写作中过于宽松,1.5 或 1.8 是更常见的选择。
\begin{spacing}{1.5}

\mydef{波面 (wave surface)}
波在同一时刻到达的各点所组成的面。一个波面上各点同时开始振动,具有相同的相位,因此波面又称同相面或等相面。

\mydef{平面波 (plane wave)}
波面为一系列平行平面的波。

\mydef{球面波 (spherical wave)}
波面为一系列同心球面的波。

\mydef{波前 (wavefront)}
指最前沿的波面。

\mydef{波线 (ray)}
沿着波传播方向的射线。在各向同性的介质中,波线恒垂直于波面。相关示意如图\ref{fig:wave}所示。

\end{spacing}

% --- 图表示例 ---
% 使用 [htbp] 参数给予 LaTeX 最大的灵活性来放置图片:
% h: here (这里)
% t: top (页面顶部)
% b: bottom (页面底部)
% p: page (独立一页)
% 常常使用 [htbp] 或 [tbp] 是最佳实践。
\begin{figure}[htbp]
    \centering
    % 使用 width=\linewidth 的相对宽度可以让图片自适应页宽,比固定 scale 更具弹性。
    % 例如 width=0.6\textwidth 表示图片宽度为文字区域宽度的 60%。
    \includegraphics[width=0.5\textwidth]{picture/wave.png}
    \caption{波前、波面与波线示意图}
    \label{fig:wave} % 使用 "类型:名称" 的标签格式 (如 fig:, tab:, eq:) 是个好习惯。
\end{figure}

% --- 使用段落来分隔不同概念,而不是手动空行 ---
\subsubsection*{惠更斯原理 (Huygens’ Principle)}
光扰动同时到达的空间曲面称为波面或波前。
波前上的每点都可以视为一个新的扰动中心,称为“子波源”。
每个子波源会向四周发出球面\textbf{“次波”(secondary wavelet)}。
在下一时刻,所有这些次波的公切面(或称为包络面)就构成了新的波前。
子波源与其次波切点的连线方向,即为该处光线的传播方向。

\begin{figure}[H]
    \centering
    \includegraphics[width = 0.6\textwidth]{picture/huigengsi.png}
    \caption{惠更斯原理示意图}
    \label{fig:huigengsi}
\end{figure}

\subsubsection*{折射定律的导出}

\insertfigure{picture/zheshe.png}{折射定律示意图}{zheshe}


首先,我们分析波前从 $ABC$ 传播到 $A'B'C'$ 的过程。如图\ref{fig:zheshe}所示,当波前上的 $A$ 点到达界面时,波前上的 $C$ 点还需要一段时间 $\Delta t$ 才能传播到界面上的 $C'$ 点。这段时间为:
$$
\Delta t = \frac{\overline{CC'}}{v_1}
$$

在相同的时间 $\Delta t$ 内,位于界面上 $A$ 点的子波源,会在介质2中形成一个半径为 $\overline{AA'}$ 的球面次波。其传播的距离为:

$$
\overline{AA'} = v_2 \cdot \Delta t = v_2 \frac{\overline{CC'}}{v_1}
$$

此时,从 $C'$ 点作这个球面次波的切线,切点为 $A'$。根据惠更斯原理,这条切线 $A'C'$ 即为在介质2中新的折射波前。而连接次波中心 $A$ 和切点 $A'$ 的直线 $AA'$,则代表了折射光线的传播方向,其与法线的夹角即为折射角 $i_2$。

接下来,我们在直角三角形 $\triangle AC'C$ 和 $\triangle AC'A'$ 中,可以分别得到入射角和折射角正弦值的几何关系:

\begin{align*}
\sin i_1 &= \frac{\overline{CC'}}{\overline{AC'}} \\
\sin i_2 &= \frac{\overline{AA'}}{\overline{AC'}} = \frac{v_2 \cdot \overline{CC'}}{v_1 \cdot \overline{AC'}}
\end{align*}


最后,将上述两式相除,消去公因子 $\overline{CC'}$ 和 $\overline{AC'}$,我们就得到了最终的\textbf{折射定律}:

\begin{gather*}
\frac{\sin i_1}{\sin i_2} = \frac{v_1}{v_2} = \text{const}
\end{gather*}


这个结果表明,入射角正弦值与折射角正弦值的比率,等于两介质中波速的比率,是一个常数。

定义:
$$
    n = \frac{c}{v}
$$
则有
\begin{center}
    \boxedmath{$n_1\sin i_1 = n_2\sin i_2$}
\end{center}

\subsubsection*{评述惠更斯原理}
{
\small\subsubsection*{惠更斯原理的不足}
    {
        \large
        \centering
        1.不能回答光振幅和光强的问题
         
        2.不能回答光相位的传播问题
        
    }
    \subsubsection*{惠更斯原理的精华}
    {
        \large
        \centering
        次波源概念的提出
        
    }
}


\subsubsection{光程概念及其意义}


\mydef{光程}光线路径的几何长度与所经过的介质折射率的乘积
\subsubsection*{介质分区均匀光程:}

$$
    L(QP) = n_1l_1 + n_2l_2 + ... = \sum_{i}^{}n_il_i  
$$

\subsubsection*{变折射率介质:}
$$
L(QP) = \int_{Q}^{P} n(r) \,dx 
$$


\subsubsection*{光程与相位差和时差的关系}

\insertfigure{picture/guangchengguanxi.png}{示意图}{fig:guangchengguanxi}


\begin{enumerate}
    \item \textbf{相位差与光程的关系}
    
    \noindent\hspace*{1.5em} $\mathrm{Q \rightarrow M \rightarrow N \cdots \rightarrow P}$,相位逐步落后;
    
    \begin{align*}
        \varphi(P) - \varphi(Q) &= - \left(\frac{2\pi}{\lambda_1} l_1 + \frac{2\pi}{\lambda_2} l_2 + \cdots + \frac{2\pi}{\lambda_4} l_4 \right) \\
        &= - \left(\frac{2\pi n_1}{\lambda_0} l_1 + \frac{2\pi n_2}{\lambda_0} l_2 + \cdots + \frac{2\pi n_4}{\lambda_0} l_4 \right) \\
        &= - \frac{2\pi}{\lambda_0} \sum n_i l_i = - \frac{2\pi}{\lambda_0} L(QP)
    \end{align*}
    
    \noindent\hspace*{1.5em} 即:\boxedmath{$\varphi(P) - \varphi(Q) = - \frac{2\pi}{\lambda_0} L(QP)$}
    
    \item \textbf{时差与光程的关系}
    
    \begin{align*}
        t_P - t_Q &= \sum\Delta t_i = \sum \frac{l_i}{v_i} = \sum \frac{n_i l_i}{c} = \frac{1}{c} L(QP)
    \end{align*}
    
    \noindent\hspace*{1.5em} 即:$t_P - t_Q = \frac{L(QP)}{c}$ \quad 或 \quad \boxedmath{$L(QP) = c \cdot t(QP)$}

\end{enumerate}

\vspace{1em}

\noindent 给出光程的又一新定义:\textbf{ 光线经历QP两点的光程等于传播时间乘以真空光速,虽然光线实际上传播于介质中。}

\vspace{1em}

\subsubsection{费马原理及其数学形式}

\mydef{费马原理}光线沿光程为平稳值的路径而传播

\insertfigure{picture/fmshiyitu.png}{费马原理示意图}{fig:fm}

\begin{center}
    $L(QP) = \int_{Q}^{P} n(r) \,ds \rightarrow $平稳值
\end{center}

\subsubsection*{平稳值的三种含义}

\begin{itemize}
    \item 极小值$\rightarrow$常见情况
    \item 常数$\rightarrow$成像系统的物像关系
    \item 极大值$\rightarrow$个别现象
\end{itemize}
\insertfigure{picture/jdzexample.png}{极大值情况实例}{fig:jdzexample}

积分是路径的泛函,平稳值要求变分为零

\vspace{1em}

即:
\begin{center}
    \boxedmath
    {$
        \delta \int_{Q}^{P} n(r)\,ds = 0 
    $
    \quad
    或
    \quad
    $
    \delta L(l) = 0
    $}

\end{center}

\subsubsection{由费马原理导出光学的三大实验定律}

\subsubsection*{反射定律}

\insertfigure{picture/fanshe1.png}{导出反射定律示意图1}{fig:fanshe1}
\insertfigure{picture/fanshe2.png}{导出反射定律示意图2}{fig:fanshe2}

由\textbf{线段距离最短}可得:

\textbf{当入射角$i$等于出射角$i'$时},光程最短
$$
    i = i'
$$

\subsubsection*{折射定律}

\insertfigure{picture/zheshe2.png}{导出折射定律示意图}{fig:zheshe}

\begin{itemize}
    \item[(1)] 折射光线在入射面内,方法和反射定律推导一样。
    \item[(2)] 入射角和折射角的关系;$Q \rightarrow M \rightarrow P$的光程:
\end{itemize}

$$ L = n_1 \overline{QM} + n_2 \overline{MP} = n_1 \sqrt{y_1^2 + (x-x_1)^2} + n_2 \sqrt{y_2^2 + (x_2-x)^2} $$

根据费马原理,$\delta L=0$,对$x$的一阶导数等于零:

$$ \frac{dL}{dx} = n_1 \frac{x-x_1}{\sqrt{y_1^2 + (x-x_1)^2}} - n_2 \frac{x_2-x}{\sqrt{y_2^2 + (x_2-x)^2}} = 0 $$

$$ \frac{dL}{dx} = n_1 \frac{\overline{QA}}{\overline{QM}} - n_2 \frac{\overline{PB}}{\overline{PM}} = n_1 \sin i_1 - n_2 \sin i_2 = 0 $$

$$ \Rightarrow n_1 \sin i_1 = n_2 \sin i_2 $$

此即 \textbf{Snell定律}。

\subsubsection{物像等光程性}

\mydef{物像等光程性}物像之间各条光线的光程是相等的。

\insertfigure{picture/dgcx.png}{等光程性示意图}{fig:dgcx}

\begin{center}
    物点Q $\longrightarrow$ 象点Q' \\
    同心光束 $\longrightarrow$ 同心光束 \\

    (同心光束的共轭变换)
\end{center}


\bigskip

等光程性是指 $$L(QM_1N_1Q') = L(QM_2N_2Q') = \dots$$

即 $$L(QM_iN_iQ') = \text{const.}$$ 与$i$无关。可取反证法证之。



\subsubsection*{理想光学系统成像}
\insertfigure{picture/lxgxxtcx.png}{示意图}{fig:lxgxxtcx}


\begin{align*}
    \text{\textbf{严格等光程}} &\Leftrightarrow \text{\textbf{严格成像,}} \\
    \text{\textbf{近似等光程}} &\Leftrightarrow \text{\textbf{近似成像,}} \\
    \text{\textbf{非等光程}} &\Leftrightarrow \text{\textbf{不成像,}}
\end{align*}

\subsubsection*{虚光程}
\textbf{虚光程的计算:}\\
\insertfigure{picture/xgc.png}{虚光程}{fig:xgc}

$$
L(N_iQ') = n' \cdot \overline{N_iQ'}
$$
% 數學公式推導部分
% 使用 align* 環境來對齊多行公式
\begin{align*}
    & L(QM_1N_1) + L(N_1V_1) = L(QM_2N_2) + L(N_2V_2) = \dots \\
    & L(N_1V_1) = n'(\overline{V_1Q'} - \overline{N_1Q'}), \quad L(N_2V_2) = n'(\overline{V_2Q'} - \overline{N_2Q'}), \quad \overline{V_1Q'} = \overline{V_2Q'} \\
    & L(QM_1N_1) - L(N_1Q') = L(QM_2N_2) - L(N_2Q') = \dots \\
    & = L(QM_iN_i) - L(N_iQ') = \dots \\[2ex] % \\[2ex] 增加行間距
    & -L(N_iQ') = -n' \cdot \overline{N_iQ'} = \underbrace{n'' }_{\text{虛折射率}}\cdot \overline{N_iQ'} = L'(N_iQ') 
\end{align*}

% 底部結論公式
\boxedmath{$ L(QM_1N_1) + \textcolor{red}{L'(N_1Q')} = \dots = L(QM_iN_i) + \textcolor{red}{L'(N_iQ')} = \dots$}
   


\subsubsection{费马原理应用于球面折反系统}
\insertfigure{picture/1.png}{单个球面的傍轴光线成像}{fig:1}

根据等光程性
$$L(QMQ')=L(QOQ')$$
可得:
$$\frac{n}{s} + \frac{n'}{x} = \frac{n'-n}{r}$$
通常写为:
\boxedmath{$\frac{n}{s} + \frac{n'}{s'} = \frac{n'-n}{r}$}

\noindent 注:

\noindent 正负规定:(参考点为$O$)

\noindent $r>0$,球心在$O$右侧;$QO>0$,$Q$在$O$左侧;$Q'O>0$,$Q'$在$O$右侧

\noindent 傍轴条件: $\Delta \ll s, r, x$ (paraxial rays)
\insertfigure{picture/2.png}{傍轴条件下的反射成像}{fig:2}
同样的,有:
\begin{center}
    \boxedmath{$\frac{1}{s} + \frac{1}{s'} = \frac{2}{r} , f = f' = -\frac{r}{2}$}
\end{center}

\subsubsection{某些特例}

\insertfigure{picture/3.png}{反射等光程面}{fig:3}

\subsubsection*{球面折射的齐明点}
\insertfigure{picture/4.png}{齐明点}{fig:4}
对于单球面折射,一般而言只能实现傍轴成像,但也存在一对特殊共轭点 $(Q_0, Q'_0)$,
可以宽光束严格成像。这一对共轭点称为 \textbf{\color{red}齐明点}。

显微镜就工作于齐明点。

\subsubsection*{阿贝正弦定理}
\insertfigure{picture/5.png}{阿贝正弦定理}{fig:5}
有$$n y \sin u = n' y' \sin u'$$

% ... 内容接续 ...




