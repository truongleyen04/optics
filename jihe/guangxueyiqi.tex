\subsubsection{光学仪器简介}


\subsubsection*{近视和远视}

\noindent首先是我们的眼睛

\noindent最主要的一个重要常数是

$$s_{\text{dv}} = 250 \text{ mm}$$

\noindent$s_{\text{dv}}$为明视距离
\vspace{1em}
\noindent正常情况下,\noindent我们的视网膜要看清一个物体应该要保证这个物体发出的平行光经过晶状体的折射后能够在视网膜上汇聚成一个点

\insertfigure{picture/yiqi1.png}{近视眼}{fig:13}

\noindent如图,\noindent近视眼即成像位置在视网膜前

\noindent那么如何通过数学计算来衡量一个人的近视(远视)程度呢?

$$\frac{n}{s} + \frac{n'}{s'} = \frac{n'-n}{r} = \Phi$$

\noindent通过这个式子可以推导出其光焦度,而光焦度*100的绝对值正是近视(远视)的度数

下面是具体的例子:

\insertfigure{picture/yiqi2.png}{例子}{fig:14}

\subsubsection*{放大镜和显微镜}

放大镜-放大率公式:
\boxedmath{$ M = \frac{s_{dv}}{f} = \frac{250}{f} $}

显微镜-放大率公式:
\boxedmath{$ M = V_o M_e = -\frac{s_{dv} \Delta}{f_o f_e} $}

这里的$\Delta$为光学筒长,$f_o$为显微镜目镜焦距,$f_e$为物镜焦距
