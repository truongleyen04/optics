\documentclass[12pt]{ctexart} % 使用 ctexart 文档类,专为中文排版设计。12pt 为默认字体大小。

% --- 核心功能宏包 ---
\usepackage{amsmath}        % 美国数学学会 (AMS) 宏包,提供进阶的数学公式环境。
\usepackage{amssymb}        % 提供 amsmath 未包含的额外数学符号。
\usepackage{graphicx}       % 用于插入图片的核心宏包,提供 \includegraphics 命令。
\usepackage{xcolor}         % 提供颜色支持,可用于文字、背景、表格等。
\usepackage{setspace}       % 用于灵活调整行距,例如 \onehalfspacing, \doublespacing。

% --- 排版与格式设定宏包 ---
\usepackage{geometry}       % 方便地设定页面边距与版心。
\usepackage{siunitx}        % 专业地排版带有国际单位 (SI) 的数字与单位。
\usepackage{abstract}       % 提供摘要格式的自定义功能。
\usepackage{cite}           % 美化与管理引用文献标号,例如将 [1,2,3] 压缩为 [1-3]。
\usepackage[colorlinks=true, linkcolor=blue, citecolor=red, urlcolor=cyan]{hyperref} % 建立 PDF 超链接,通常建议放在绝大多数宏包之后加载。
\usepackage{cleveref}       % 智能引用,可自动识别图、表、公式,生成如 “图 1” 而非仅 “1”。
\usepackage{float}
\usepackage{todonotes}
\usepackage{siunitx}
% --- 页面布局设定 (geometry) ---
\geometry{
    a4paper,
    left=25mm,
    right=25mm,
    top=30mm,
    bottom=30mm,
}
\setlength{\parindent}{0pt}



% %%%%%%%%%%%%%%%%%%%%%%%%%%%%%%%%%%%%%%%%%%%%%%%%%%%%%
% 2. 自定义命令与全局设定
% --- 为了代码的简洁与一致性 ---
% %%%%%%%%%%%%%%%%%%%%%%%%%%%%%%%%%%%%%%%%%%%%%%%%%%%%%

% --- 自定义一个用于“定义”的命令 ---
% #1 代表第一个参数 (术语名称)
% 使用 \noindent 来避免段首缩进,使其看起来更像一个标题。
\newcommand{\mydef}[1]{\par\noindent\textbf{#1}:}

% --- 自定义图片插入命令 (基础版) ---
% #1: 图片路径 (例如: picture/image.png)
% #2: 图片说明 (caption)
% #3: 引用标签 (label, 无需加 fig: 前缀)
\newcommand{\insertfigure}[3]{
    \begin{figure}[H]
        \centering
        \includegraphics[width=0.6\textwidth]{#1}
    \end{figure}
}
\newcommand{\dd}{\mathrm{d}}
\newcommand{\ii}{\mathrm{i}} 
\newcommand{\boxedmath}[1]{
  \fcolorbox{black}{white}{\scalebox{1.3}{#1}}
}
\usepackage{amsmath}
\DeclareMathOperator{\sinc}{sinc}
\usepackage{amsmath}
\usepackage{amsfonts}

% 定义自定义数学算子,确保字体为正体且间距正确
\DeclareMathOperator{\comb}{comb}
\DeclareMathOperator{\rect}{rect}
\newcommand{\circfunc}{\text{circ}} % 或者定义为 \DeclareMathOperator{\circfunc}{circ}

% 如果不想定义新命令,可以直接在正文使用 \operatorname{...}

% --- 文档信息 ---
\title{\bfseries 光学} % 标题内容。字体大小由文档类自动处理,这里只指定粗体。
\author{施维圣}
\date{2025年9月22日 $-$ \today}                 % 使用 \today 自动生成编译当天的日期。



\begin{document}
\maketitle



% --- 正文章节 ---
\section{几何光学基本原理与成像规律}
\subsection{费马原理}
\subsubsection{惠更斯原理与折射率}

% 使用 setspace 提供的环境来设定局部行距,1.5 约等于双倍行高。
% 2.5 的行距在学术写作中过于宽松,1.5 或 1.8 是更常见的选择。
\begin{spacing}{1.5}

\mydef{波面 (wave surface)}
波在同一时刻到达的各点所组成的面。一个波面上各点同时开始振动,具有相同的相位,因此波面又称同相面或等相面。

\mydef{平面波 (plane wave)}
波面为一系列平行平面的波。

\mydef{球面波 (spherical wave)}
波面为一系列同心球面的波。

\mydef{波前 (wavefront)}
指最前沿的波面。

\mydef{波线 (ray)}
沿着波传播方向的射线。在各向同性的介质中,波线恒垂直于波面。相关示意如图\ref{fig:wave}所示。

\end{spacing}

% --- 图表示例 ---
% 使用 [htbp] 参数给予 LaTeX 最大的灵活性来放置图片:
% h: here (这里)
% t: top (页面顶部)
% b: bottom (页面底部)
% p: page (独立一页)
% 常常使用 [htbp] 或 [tbp] 是最佳实践。
\begin{figure}[htbp]
    \centering
    % 使用 width=\linewidth 的相对宽度可以让图片自适应页宽,比固定 scale 更具弹性。
    % 例如 width=0.6\textwidth 表示图片宽度为文字区域宽度的 60%。
    \includegraphics[width=0.5\textwidth]{picture/wave.png}
    \caption{波前、波面与波线示意图}
    \label{fig:wave} % 使用 "类型:名称" 的标签格式 (如 fig:, tab:, eq:) 是个好习惯。
\end{figure}

% --- 使用段落来分隔不同概念,而不是手动空行 ---
\subsubsection*{惠更斯原理 (Huygens’ Principle)}
光扰动同时到达的空间曲面称为波面或波前。
波前上的每点都可以视为一个新的扰动中心,称为“子波源”。
每个子波源会向四周发出球面\textbf{“次波”(secondary wavelet)}。
在下一时刻,所有这些次波的公切面(或称为包络面)就构成了新的波前。
子波源与其次波切点的连线方向,即为该处光线的传播方向。

\begin{figure}[H]
    \centering
    \includegraphics[width = 0.6\textwidth]{picture/huigengsi.png}
    \caption{惠更斯原理示意图}
    \label{fig:huigengsi}
\end{figure}

\subsubsection*{折射定律的导出}

\insertfigure{picture/zheshe.png}{折射定律示意图}{zheshe}


首先,我们分析波前从 $ABC$ 传播到 $A'B'C'$ 的过程。如图\ref{fig:zheshe}所示,当波前上的 $A$ 点到达界面时,波前上的 $C$ 点还需要一段时间 $\Delta t$ 才能传播到界面上的 $C'$ 点。这段时间为:
$$
\Delta t = \frac{\overline{CC'}}{v_1}
$$

在相同的时间 $\Delta t$ 内,位于界面上 $A$ 点的子波源,会在介质2中形成一个半径为 $\overline{AA'}$ 的球面次波。其传播的距离为:

$$
\overline{AA'} = v_2 \cdot \Delta t = v_2 \frac{\overline{CC'}}{v_1}
$$

此时,从 $C'$ 点作这个球面次波的切线,切点为 $A'$。根据惠更斯原理,这条切线 $A'C'$ 即为在介质2中新的折射波前。而连接次波中心 $A$ 和切点 $A'$ 的直线 $AA'$,则代表了折射光线的传播方向,其与法线的夹角即为折射角 $i_2$。

接下来,我们在直角三角形 $\triangle AC'C$ 和 $\triangle AC'A'$ 中,可以分别得到入射角和折射角正弦值的几何关系:

\begin{align*}
\sin i_1 &= \frac{\overline{CC'}}{\overline{AC'}} \\
\sin i_2 &= \frac{\overline{AA'}}{\overline{AC'}} = \frac{v_2 \cdot \overline{CC'}}{v_1 \cdot \overline{AC'}}
\end{align*}


最后,将上述两式相除,消去公因子 $\overline{CC'}$ 和 $\overline{AC'}$,我们就得到了最终的\textbf{折射定律}:

\begin{gather*}
\frac{\sin i_1}{\sin i_2} = \frac{v_1}{v_2} = \text{const}
\end{gather*}


这个结果表明,入射角正弦值与折射角正弦值的比率,等于两介质中波速的比率,是一个常数。

定义:
$$
    n = \frac{c}{v}
$$
则有
\begin{center}
    \boxedmath{$n_1\sin i_1 = n_2\sin i_2$}
\end{center}

\subsubsection*{评述惠更斯原理}
{
\small\subsubsection*{惠更斯原理的不足}
    {
        \large
        \centering
        1.不能回答光振幅和光强的问题
         
        2.不能回答光相位的传播问题
        
    }
    \subsubsection*{惠更斯原理的精华}
    {
        \large
        \centering
        次波源概念的提出
        
    }
}


\subsubsection{光程概念及其意义}


\mydef{光程}光线路径的几何长度与所经过的介质折射率的乘积
\subsubsection*{介质分区均匀光程:}

$$
    L(QP) = n_1l_1 + n_2l_2 + ... = \sum_{i}^{}n_il_i  
$$

\subsubsection*{变折射率介质:}
$$
L(QP) = \int_{Q}^{P} n(r) \,dx 
$$


\subsubsection*{光程与相位差和时差的关系}

\insertfigure{picture/guangchengguanxi.png}{示意图}{fig:guangchengguanxi}


\begin{enumerate}
    \item \textbf{相位差与光程的关系}
    
    \noindent\hspace*{1.5em} $\mathrm{Q \rightarrow M \rightarrow N \cdots \rightarrow P}$,相位逐步落后;
    
    \begin{align*}
        \varphi(P) - \varphi(Q) &= - \left(\frac{2\pi}{\lambda_1} l_1 + \frac{2\pi}{\lambda_2} l_2 + \cdots + \frac{2\pi}{\lambda_4} l_4 \right) \\
        &= - \left(\frac{2\pi n_1}{\lambda_0} l_1 + \frac{2\pi n_2}{\lambda_0} l_2 + \cdots + \frac{2\pi n_4}{\lambda_0} l_4 \right) \\
        &= - \frac{2\pi}{\lambda_0} \sum n_i l_i = - \frac{2\pi}{\lambda_0} L(QP)
    \end{align*}
    
    \noindent\hspace*{1.5em} 即:\boxedmath{$\varphi(P) - \varphi(Q) = - \frac{2\pi}{\lambda_0} L(QP)$}
    
    \item \textbf{时差与光程的关系}
    
    \begin{align*}
        t_P - t_Q &= \sum\Delta t_i = \sum \frac{l_i}{v_i} = \sum \frac{n_i l_i}{c} = \frac{1}{c} L(QP)
    \end{align*}
    
    \noindent\hspace*{1.5em} 即:$t_P - t_Q = \frac{L(QP)}{c}$ \quad 或 \quad \boxedmath{$L(QP) = c \cdot t(QP)$}

\end{enumerate}

\vspace{1em}

\noindent 给出光程的又一新定义:\textbf{ 光线经历QP两点的光程等于传播时间乘以真空光速,虽然光线实际上传播于介质中。}

\vspace{1em}

\subsubsection{费马原理及其数学形式}

\mydef{费马原理}光线沿光程为平稳值的路径而传播

\insertfigure{picture/fmshiyitu.png}{费马原理示意图}{fig:fm}

\begin{center}
    $L(QP) = \int_{Q}^{P} n(r) \,ds \rightarrow $平稳值
\end{center}

\subsubsection*{平稳值的三种含义}

\begin{itemize}
    \item 极小值$\rightarrow$常见情况
    \item 常数$\rightarrow$成像系统的物像关系
    \item 极大值$\rightarrow$个别现象
\end{itemize}

积分是路径的泛函,平稳值要求变分为零

\vspace{1em}

即:
\begin{center}
    \boxedmath
    {$
        \delta \int_{Q}^{P} n(r)\,ds = 0 
    $
    \quad
    或
    \quad
    $
    \delta L(l) = 0
    $}

\end{center}

\subsubsection{由费马原理导出光学的三大实验定律}

\subsubsection*{反射定律}

\insertfigure{picture/fanshe1.png}{导出反射定律示意图1}{fig:fanshe1}
\insertfigure{picture/fanshe2.png}{导出反射定律示意图2}{fig:fanshe2}

由\textbf{线段距离最短}可得:

\textbf{当入射角$i$等于出射角$i'$时},光程最短
$$
    i = i'
$$

\subsubsection*{折射定律}

\insertfigure{picture/zheshe2.png}{导出折射定律示意图}{fig:zheshe}

\begin{itemize}
    \item[(1)] 折射光线在入射面内,方法和反射定律推导一样。
    \item[(2)] 入射角和折射角的关系;$Q \rightarrow M \rightarrow P$的光程:
\end{itemize}

$$ L = n_1 \overline{QM} + n_2 \overline{MP} = n_1 \sqrt{y_1^2 + (x-x_1)^2} + n_2 \sqrt{y_2^2 + (x_2-x)^2} $$

根据费马原理,$\delta L=0$,对$x$的一阶导数等于零:

$$ \frac{dL}{dx} = n_1 \frac{x-x_1}{\sqrt{y_1^2 + (x-x_1)^2}} - n_2 \frac{x_2-x}{\sqrt{y_2^2 + (x_2-x)^2}} = 0 $$

$$ \frac{dL}{dx} = n_1 \frac{\overline{QA}}{\overline{QM}} - n_2 \frac{\overline{PB}}{\overline{PM}} = n_1 \sin i_1 - n_2 \sin i_2 = 0 $$

$$ \Rightarrow n_1 \sin i_1 = n_2 \sin i_2 $$

此即 \textbf{Snell定律}。

\subsubsection{物像等光程性}


% ... 内容接续 ...





\subsection{几何光学成像规律}
\subsubsection{成像基本概念}
\insertfigure{picture/6.png}{不同的成像种类}{fig:6}
\noindent 实(虚)物: 物点发射出发散(会聚)的同心光束。

\noindent 实(虚)像: 像点形成自会聚(发散)的同心光束。
\subsubsection{共轴球面光具组傍轴成像}
单个球面的情况下,有:
\begin{equation}
    \frac{n}{s} + \frac{n'}{s'} = \frac{n'-n}{r} = \Phi \label{eq:1}
\end{equation}

$ \Phi$为光焦度
由于
\begin{align*}
f &= \frac{n}{n'-n}r = \frac{n}{\Phi} &
f' &= \frac{n'}{n'-n}r = \frac{n'}{\Phi}
\end{align*}
则可将\eqref{eq:1}化简为:
$$\frac{f}{s} + \frac{f'}{s'} = 1$$

\subsubsection*{傍轴物点成像和横向放大率}
\insertfigure{picture/7.png}{物点成像}{fig:7}
\noindent $\Pi$上所有的点都成像在$\Pi'$上(当然只限于\textbf{\color{red}傍轴区域})。

\noindent 这样一对由共轭点组成的平面叫做\textbf{\color{red}共轭平面},

\noindent 其中$\Pi$叫\textbf{\color{red}物平面},$\Pi'$叫\textbf{\color{red}像平面}。

\noindent 注意:
\noindent 若$P$ (或$P'$)在光轴上方, $y$ (或$y'$)$>0$;

\noindent \quad \quad \quad 在光轴下方, $y$ (或$y'$)$<0$。 % \quad 用于模拟图片中的缩进效果

\vspace{1em}
定义横向放大率为 $V = y'/y$

可导出:
$$V = - \frac{ns'}{n's}$$

对于:
\insertfigure{picture/8.png}{多个同轴球面}{fig:8}

有$V = V_1 V_2 V_3 \dots$

\subsubsection*{拉格朗日-亥姆霍兹定理}
\insertfigure{picture/9.png}{示意图}{fig:9}
在傍轴条件下,
\begin{center}
    \boxedmath{$ynu = y'n'u'$}
\end{center}
推广到多个共轴球面系统,
\begin{center}
    \boxedmath{$ynu = y'n'u' = y''n''u'' = \dots$}
\end{center}
\subsubsection{薄透镜及透镜组成像}
\insertfigure{picture/10.png}{薄透镜}{fig:10}
薄透镜要求:
$$d \ll f, f', s, s'$$
对于薄透镜的前后两个面,有
\begin{align*}
    \Sigma_1: \quad \frac{f_1}{s_1} + \frac{f'_1}{s'_1} &= 1 &
    \Sigma_2: \quad \frac{f_2}{s_2} + \frac{f'_2}{s'_2} &= 1
\end{align*}
令
\begin{align*}
  f &= \frac{f_1 f_2}{f'_1 + f_2}, &
f' &= \frac{f'_1 f'_2}{f'_1 + f_2} &
\end{align*}
两式合并,可得
\begin{center}
    \boxedmath{$ \frac{f}{s} + \frac{f'}{s'} = 1 $}
\end{center}

一般情况下,$n = n' \approx 1$
则
$$ 
\Phi = \frac{n_L - n}{r_1} + \frac{n' - n_L}{r_2} = (n_L - 1)\left(\frac{1}{r_1} - \frac{1}{r_2}\right) 
$$
而
$$
\left\{
\begin{aligned}
f &= \frac{n}{\frac{n_L - n}{r_1} + \frac{n' - n_L}{r_2}} = \frac{n}{\Phi} \\
f' &= \frac{n'}{\frac{n_L - n}{r_1} + \frac{n' - n_L}{r_2}} = \frac{n'}{\Phi}
\end{aligned}
\right.
$$
则
$$ f = \frac{n}{\Phi} = \frac{n'}{\Phi} = f' = \frac{1}{(n_L - 1)\left(\frac{1}{r_1} - \frac{1}{r_2}\right)} $$
此公式被称为\textbf{磨镜者公式}
\vspace{1em}
透镜度数和光焦度的关系为
$$ \text{度数} = P \times 100 $$
\subsubsection*{牛顿公式}
\insertfigure{picture/11.png}{牛顿公式示意图}{fig:11}

引入$x$
$$ x = s - f, \quad x' = s' - f' $$
推导可得
$$ V = -\frac{s'}{s} = -\frac{f'}{x} = -\frac{f}{x'} = -\frac{x'}{f'} = -\frac{x}{f} $$
这便是\textbf{薄透镜横向放大率公式}。
\vspace{1em}
\subsubsection*{密接透镜组}




\subsubsection{光学仪器简介}


\subsubsection*{近视和远视}

\noindent首先是我们的眼睛

\noindent最主要的一个重要常数是

$$s_{\text{dv}} = 250 \text{ mm}$$

\noindent$s_{\text{dv}}$为明视距离
\vspace{1em}
\noindent正常情况下,\noindent我们的视网膜要看清一个物体应该要保证这个物体发出的平行光经过晶状体的折射后能够在视网膜上汇聚成一个点

\insertfigure{picture/yiqi1.png}{近视眼}{fig:13}

\noindent如图,\noindent近视眼即成像位置在视网膜前

\noindent那么如何通过数学计算来衡量一个人的近视(远视)程度呢?

$$\frac{n}{s} + \frac{n'}{s'} = \frac{n'-n}{r} = \Phi$$

\noindent通过这个式子可以推导出其光焦度,而光焦度*100的绝对值正是近视(远视)的度数

下面是具体的例子:

\insertfigure{picture/yiqi2.png}{例子}{fig:14}

\subsubsection*{放大镜和显微镜}

放大镜-放大率公式:
\boxedmath{$ M = \frac{s_{dv}}{f} = \frac{250}{f} $}

显微镜-放大率公式:
\boxedmath{$ M = V_o M_e = -\frac{s_{dv} \Delta}{f_o f_e} $}

这里的$\Delta$为光学筒长,$f_o$为显微镜目镜焦距,$f_e$为物镜焦距


\section{波动光学引论}

\begin{spacing}{1.5}

\subsection{光的电磁理论基础}
\mydef{麦克斯韦方程组与波动方程}
光是电磁波。在无源($\rho=0, \vec{J}=0$)、线性、各向同性、均匀介质中,电场 $\vec{E}$ 和磁场 $\vec{B}$ 均满足矢量波动方程:
\begin{equation}
    \nabla^2 \vec{E} - \mu\varepsilon \frac{\partial^2 \vec{E}}{\partial t^2} = 0
\end{equation}
相速度 $v = 1/\sqrt{\mu\varepsilon}$。真空中 $c \approx \SI{3e8}{m/s}$。
折射率 $n = c/v = \sqrt{\varepsilon_r \mu_r} \approx \sqrt{\varepsilon_r}$(大多数光学介质中 $\mu_r \approx 1$)。

\mydef{光强 }
光强定义为坡印廷矢量  $\vec{S} = \vec{E} \times \vec{H}$ 大小的物理平均值。在同一种介质中,光强正比于电场振幅的平方:
\begin{equation}
    I \propto \langle |\vec{E}|^2 \rangle
\end{equation}

\mydef{标量波近似}
当光波在均匀介质中传播,且不涉及边界上的偏振转换(如菲涅耳反射公式)或各向异性介质(如晶体双折射)时,可忽略电场的矢量性,用标量函数 $U(\vec{r}, t)$ 描述光场。

\subsection{定态光波与复振幅}
核心思想:\textbf{用复数运算简化波动分析}。
对于单色光(角频率 $\omega$),实数波函数 $U(\vec{r}, t) = A(\vec{r}) \cos[\omega t - \varphi(\vec{r})]$ 可写为:
\begin{equation}
    U(\vec{r}, t) = \operatorname{Re} [\tilde{U}(\vec{r}) e^{\ii \omega t}]
\end{equation}
其中 \textbf{复振幅} $\tilde{U}(\vec{r})$ 定义为:
\begin{equation}
    \tilde{U}(\vec{r}) = A(\vec{r}) e^{-\ii \varphi(\vec{r})}
\end{equation}
复振幅满足不含时的 \textbf{亥姆霍兹方程}:
\begin{equation}
    \nabla^2 \tilde{U} + k^2 \tilde{U} = 0, \quad k = \frac{n\omega}{c} = \frac{2\pi}{\lambda}
\end{equation}

\subsection{典型光波及其复振幅}
\begin{enumerate}
    \item \textbf{平面波}:传播方向为 $\vec{k}$。
    $$ \tilde{U}(\vec{r}) = A e^{-\ii \vec{k} \cdot \vec{r}} $$
    等相面方程:$\vec{k} \cdot \vec{r} = \text{C}$(平面)。
    
    \item \textbf{球面波}:从点源发散。
    $$ \tilde{U}(r) = \frac{A}{r} e^{-\ii kr} $$
    振幅 $A/r$ 随距离衰减,能量守恒($I \propto 1/r^2$)。

    \textbf{傍轴近似}:在距离 $z$ 远处,球面波因子 $e^{-\ii kr} \approx e^{-\ii k z} e^{-\ii \frac{k}{2z}(x^2+y^2)}$,这在衍射积分中非常重要。
\end{enumerate}

\subsection{波动传播的重要近似条件与模型}
\mydef{傍轴条件}
当光线传播方向与光轴(设为 $z$ 轴)的夹角 $\theta$ 很小($\theta \ll 1$ rad)时,满足傍轴条件。
\begin{itemize}
    \item \textbf{几何近似}:$\sin\theta \approx \tan\theta \approx \theta$,$\cos\theta \approx 1 - \theta^2/2$。
    \item \textbf{相位近似}:对于球面波传播距离 $r = \sqrt{z^2 + x^2 + y^2}$,泰勒展开保留至二次项:$r \approx z + \frac{x^2+y^2}{2z}$。该近似要求高阶相位项带来的误差远小于 $\pi$,即 $k \frac{(x^2+y^2)^2}{8z^3} \ll \pi$。
    \item \textbf{物理意义}:在傍轴条件下,球面波可近似视为一个带有\textbf{二次曲面相位因子}的平面波。
\end{itemize}

\mydef{远场条件}
通常指观察距离 $z$ 远大于波源或孔径尺寸 $D$ 的平方与波长的比值的情况。
\begin{itemize}
    \item \textbf{判据}:$z \gg \frac{D^2}{\lambda}$
    \item \textbf{物理意义}:在远场区域,波源各点发出的子波到达观察点时,其光程差主要由线性项决定,二次相位项 $\frac{k(x^2+y^2)}{2z}$ 趋于常数或可忽略。此时球面波波前在观察范围内非常平坦,局部可视为\textbf{平面波}。这也是\textbf{夫琅禾费衍射}成立的前提。
\end{itemize}



\end{spacing}

\section{光的干涉}

\subsection{波的叠加与相干条件}
\begin{spacing}{1.5}
\mydef{叠加原理}
$\tilde{U} = \tilde{U}_1 + \tilde{U}_2$。总光强 $I = \tilde{U}\tilde{U}^*$:
\begin{equation}
    I = I_1 + I_2 + 2\sqrt{I_1 I_2} \cos \delta
\end{equation}
干涉项 $J_{12} = 2\sqrt{I_1 I_2} \cos \delta$,其中 $\delta(\vec{r}) = \varphi_2 - \varphi_1$ 为相位差。

\mydef{相干条件 (Coherence Conditions)}
为了观察到稳定的干涉图样($J_{12}$ 不随时间平均为零):
\begin{itemize}
    \item \textbf{频率相同} ($\omega_1 = \omega_2$):否则产生拍频,时间平均光强无干涉项。
    \item \textbf{振动方向平行}:$\vec{E}_1 \cdot \vec{E}_2 \neq 0$(菲涅耳-阿拉戈定律)。
    \item \textbf{相位差恒定}:$\delta$ 不随时间随机跳变。
\end{itemize}

\mydef{条纹可见度 (Visibility/Contrast)}
描述干涉条纹的清晰程度:
\begin{equation}
    V = \frac{I_{\max} - I_{\min}}{I_{\max} + I_{\min}} = \frac{2\sqrt{I_1 I_2}}{I_1 + I_2}
\end{equation}
当 $I_1 = I_2$ 且光源完全相干($|\gamma_{12}|=1$)时,$V=1$(最佳)。

\mydef{驻波}
两列振幅相同、相向传播的相干波叠加形成驻波。维纳实验证明了光波中的感光作用主要是由电场矢量引起的(波腹处感光最强)。
\end{spacing}

\subsection{分波前干涉}
物理上将点光源的波前分割为两部分,分别通过不同路径后汇聚。

\subsubsection{杨氏双缝干涉}
\textbf{装置参数}:缝间距 $d$,屏距 $D$,波长 $\lambda$。
\textbf{光程差} $\Delta \approx d \frac{x}{D}$。
\textbf{光强分布}:
\begin{equation}
    I(x) = I_0(1 + \cos k\frac{d}{D}x)
\end{equation}
\textbf{特征}:
\begin{itemize}
    \item 亮纹:$x_k = k \frac{D\lambda}{d}$;暗纹:$x_k = (k+0.5)\frac{D\lambda}{d}$。
    \item 条纹等间距 $\Delta x = \frac{D\lambda}{d}$。
\end{itemize}

\subsubsection{其他分波前装置}
全部将其转化为杨氏双缝干涉:
\begin{itemize}
    \item \textbf{菲涅耳双棱镜}:利用折射产生两个虚光源。$\Delta x = \frac{(B+C)\lambda}{2\alpha B}$
    \insertfigure{picture/2025-12-24-17-33-17.png}{0}{fig:0}
    \item \textbf{菲涅耳双面镜}:利用反射产生两个虚光源。$\Delta x = \frac{(B+C)\lambda}{2(n-1)\alpha B}$
    \insertfigure{picture/12}{12}{fig:12}
    \item \textbf{洛埃镜}:利用直射光与反射光干涉。$\Delta x = \frac{D\lambda}{2\alpha}$
    \insertfigure{picture/2025-12-24-17-37-11.png}{0}{fig:0}
    \textbf{重要现象}:在镜面边缘接触点处(光程差趋于0),出现暗条纹。这直接证明了光在光密介质表面反射时存在半波损失(相位突变 $\pi$)。
\end{itemize}

\subsubsection{自然光的分解模型}
自然光是大量原子随机发光的集合,宏观上无固定偏振态,但在处理干涉时常采用以下模型:
\begin{itemize}
    \item \textbf{正交分解}:可将自然光等效为两束振动方向互相垂直($x$ 和 $y$)、振幅相等($A_x=A_y$)且\textbf{互不相干}的线偏振光的叠加。
    \item \textbf{光强关系}:$I_{\text{nat}} = I_x + I_y = \frac{1}{2}I_0 + \frac{1}{2}I_0$。
    \item \textbf{干涉表现}:在同一自然光源分成的两束小角度自然光之间的干涉,其衬比度为$\gamma = \frac{1}{2}(1+\cos\alpha)$
\end{itemize}

\subsubsection{空间相干性}
实际光源具有宽度 $b$。
\begin{itemize}
    \item 光源上不同点发出的波列是不相干的,产生的干涉条纹在屏上发生位移。
    \item 当边缘点产生的条纹相对中心点位移达 $\Delta x / 2$ 时,条纹完全模糊。
\end{itemize}
\textbf{相干极限}:
\begin{equation}
    b \cdot \frac{d}{D} \le \lambda \implies b \cdot \theta_s \le \lambda
\end{equation}
其中 $\theta_s \approx d/D$ 为干涉孔径角。
\textbf{结论}:光源宽度越窄,允许的干涉孔径角越大,空间相干性越好。

\subsection{分振幅干涉}

\subsubsection{薄膜干涉基本原理}
\textbf{模型}:折射率 $n_2$,厚度 $h$ 的薄膜,置于 $n_1$ 和 $n_3$ 之间。
\textbf{光程差公式}:
\begin{equation}
    \Delta = 2 n_2 h \cos i_2 + \frac{\lambda}{2} (\text{若存在半波损失})
\end{equation}
\textbf{半波损失判据}:
当 $n_1 < n_2 < n_3$ 或 $n_1 > n_2 > n_3$ 时,上下表面反射性质相同,无附加项。
当 $n_1 < n_2$ 且 $n_2 > n_3$(如空气中肥皂膜),仅上表面有半波损失,需加 $\lambda/2$。

\subsubsection{等倾干涉}
\textbf{条件}:$h$ 均匀,面光源。
\textbf{规律}:$\Delta$ 仅随入射角 $i$ (或折射角 $i_2$) 变化。
\textbf{图样}:定域于无穷远(焦平面)的同心圆环。
\textbf{吞吐现象}:膜厚 $h$ 增加,光程差增大,中心条纹级次冒出(“吐”),条纹整体外扩变密。

\subsubsection{等厚干涉}
\textbf{条件}:$h$ 不均匀,平行光(或准平行光)入射。
\textbf{规律}:$\Delta$ 仅随厚度 $h$ 变化。条纹描绘了膜厚的等高线。
\textbf{定域}:薄膜表面附近。

\textbf{1. 劈尖}
条纹间距 $L = \frac{\lambda}{2n \sin \alpha} \approx \frac{\lambda}{2n\alpha}$。
应用:检测表面平整度(条纹弯曲度)、测量微小直径。

\textbf{2. 牛顿环}
光程差 $\Delta = 2h + \lambda/2 \approx r^2/R + \lambda/2$。
\begin{itemize}
    \item \textbf{反射光}中心为暗斑(接触点 $h=0, \Delta=\lambda/2$)。
    \item \textbf{透射光}中心为亮斑(无半波损失)。
\end{itemize}

\subsubsection{光学薄膜应用}
\begin{itemize}
    \item \textbf{增透膜}:利用干涉相消。单层膜条件:$n_{膜}h = \lambda/4$, $n_{膜} = \sqrt{n_{基}}$。
    \item \textbf{高反膜}:利用干涉相长。多层介质膜堆叠。
\end{itemize}

\subsection{迈克耳孙干涉仪与时间相干性}

\subsubsection{仪器结构与原理}
利用分光板 $G_1$ 分光。补偿板 $G_2$ 保证两臂玻璃光程相等,使仪器能用于白光干涉。
等效为空气层厚度 $d$ 的薄膜干涉。
\begin{itemize}
    \item $M_1, M_2$ 严格垂直 $\to$ 等倾干涉(圆环)。
    \item $M_1, M_2$ 有微小倾角 $\to$ 等厚干涉(直条纹,定域在楔形表面)。
\end{itemize}

\subsubsection{时间相干性}
实际光源非单色(谱宽 $\Delta \nu$ 或 $\Delta \lambda$)。
\textbf{波列长度}:原子发光过程持续时间有限(约 $\SI{e-8}{s}$),形成有限长度波列。
\textbf{相干长度} $L_c$:能够发生干涉的最大光程差。
\begin{equation}
    L_c \approx \frac{c}{\Delta \nu} = \frac{\lambda^2}{\Delta \lambda}
\end{equation}
\textbf{白光干涉}:由于白光 $\Delta \lambda$ 很大,$L_c$ 极短(仅几微米)。仅在零级条纹附近($\Delta \approx 0$)可见彩色条纹,中心为黑色(半波损失)。常用于确定“零光程差”位置。

\subsection{多光束干涉}

\subsubsection{法布里-珀罗 干涉仪}
由两块平行的高反射率($R \to 1$)平板组成。
\textbf{原理}:入射光在腔内多次反射,形成振幅递减、相位等差的无穷多束光叠加。

\todo{这里有内容需要补充}

\subsubsection{艾里公式}
透射光强 $I_T$ 随相位差 $\delta = \frac{4\pi n h \cos i}{\lambda}$ 的分布:
\begin{equation}
    I_T = I_{\max} \frac{1}{1 + F \sin^2 (\delta/2)}
\end{equation}
\textbf{精细度系数}:$F = \frac{4R}{(1-R)^2}$。
\begin{itemize}
    \item $R$ 越大,$F$ 越极大,条纹极其细锐。
    \item 半高全宽 $\varepsilon = 4/\sqrt{F}$。
\end{itemize}

\subsubsection{重要应用参数}
\begin{itemize}
    \item \textbf{自由光谱范围}:相邻两个干涉级次之间的波长差。$\Delta \lambda_{fsr} \approx \frac{\lambda^2}{2nh}$。
    \item \textbf{分辨本领}:$\mathcal{R} = \frac{\lambda}{\delta \lambda} = k \mathcal{N}$,其中 $\mathcal{N}$ 为有效光束数,正比于精细度 $\mathcal{F} = \frac{\pi\sqrt{F}}{2}$。
\end{itemize}

\section{介质界面光学 (Optics at Interfaces)}

当光波从一种介质(折射率 $n_1$)射向另一种介质(折射率 $n_2$)时,在界面上发生的反射和折射现象由电磁场边界条件决定。

\subsection{菲涅耳公式 (Fresnel Formulas)}
将入射光的电矢量 $\vec{E}$ 分解为两个互相垂直的分量:
\begin{itemize}
    \item \textbf{s 分量}(垂直分量,$\perp$):电矢量垂直于入射面。
    \item \textbf{p 分量}(平行分量,$\parallel$):电矢量平行于入射面。
\end{itemize}
设入射角为 $i_1$,折射角为 $i_2$(满足斯涅耳定律 $n_1 \sin i_1 = n_2 \sin i_2$)。

\mydef{反射系数 (Reflection Coefficients)}
\begin{equation}
    r_s = \frac{E'_{s0}}{E_{s0}} = - \frac{\sin(i_1 - i_2)}{\sin(i_1 + i_2)}, \quad
    r_p = \frac{E'_{p0}}{E_{p0}} = \frac{\tan(i_1 - i_2)}{\tan(i_1 + i_2)}
\end{equation}

\mydef{透射系数 (Transmission Coefficients)}
\begin{equation}
    t_s = \frac{E''_{s0}}{E_{s0}} = \frac{2 \sin i_2 \cos i_1}{\sin(i_1 + i_2)}, \quad
    t_p = \frac{E''_{p0}}{E_{p0}} = \frac{2 \sin i_2 \cos i_1}{\sin(i_1 + i_2) \cos(i_1 - i_2)}
\end{equation}

\textbf{注意}:
\begin{itemize}
    \item 符号法则:通常取 $r_s$ 的负号表示反射光 s 分量在光疏射向光密时存在相位突变 $\pi$(半波损失)。
    \item 当 $i_1 \to 0$(正入射)时:$r_s = r_p = \frac{n_1 - n_2}{n_1 + n_2}$(符号差异取决于坐标系定义,本质是一致的)。
\end{itemize}

\subsection{反射率与透射率 (Reflectance and Transmittance)}
光强 $I \propto n |\vec{E}|^2$。能量守恒要求穿过界面的能流连续。

\mydef{反射率 $R$}
反射光强与入射光强之比:
\begin{equation}
    R = |r|^2 \quad \Rightarrow \quad R_s = r_s^2, \quad R_p = r_p^2
\end{equation}
自然光入射时的总反射率:$R = \frac{1}{2}(R_s + R_p)$。

\mydef{透射率 $T$}
透射光强与入射光强之比(注意需考虑光束截面积变化因子 $\frac{\cos i_2}{\cos i_1}$):
\begin{equation}
    T = \frac{n_2 \cos i_2}{n_1 \cos i_1} |t|^2
\end{equation}
\textbf{能量守恒}:$R + T = 1$ 对 s 和 p 分量分别成立。

\subsection{布儒斯特角 (Brewster's Angle)}
\mydef{定义}
当 $r_p = 0$ 时,反射光中完全没有 p 分量,只有 s 分量,此时反射光为\textbf{完全线偏振光}。对应的入射角称为布儒斯特角 $i_B$。
\begin{equation}
    \tan i_B = \frac{n_2}{n_1}
\end{equation}
\textbf{物理图景}:此时反射光线与折射光线互相垂直 ($i_1 + i_2 = 90^\circ$)。折射介质中的偶极子振动方向平行于反射方向,因此不向该方向辐射能量(即无 p 分量反射)。



\subsection{全反射现象}
当光从光密介质 ($n_1$) 射向光疏介质 ($n_2$),且入射角大于临界角 $i_c$ 时,发生全反射。
\begin{equation}
    \sin i_c = \frac{n_2}{n_1}
\end{equation}
此时 $R=1$,能量全部反射回第一介质。

\subsection{相位变化与古斯-汉欣位移}
\mydef{相位变化}
在全反射区,反射系数 $r_s$ 和 $r_p$ 变为复数,模为 1,但存在相位因子 $e^{\ii \delta}$。这导致反射光产生相位延迟,且 $\delta_s \neq \delta_p$。
应用:\textbf{菲涅耳菱体}利用两次全反射将线偏振光转变为圆偏振光。

\mydef{古斯-汉欣位移 (Goos-Hänchen Shift)}
实验发现,全反射时实际反射光束相对于几何光学的反射点在界面上有一个微小的侧向位移 $D$(约波长量级)。这表明光波实际上穿透到了第二介质中一段距离后再返回。

\subsection{倏逝波 (Evanescent Wave)}
\mydef{物理本质}
在全反射条件下,第二介质(光疏介质)中虽然没有传输能量的折射波,但存在一种非均匀波,称为\textbf{倏逝波}(或隐失波)。
其电场分布特征为:
\begin{itemize}
    \item \textbf{沿界面传播}:相位因子 $e^{\ii (k_x x - \omega t)}$,沿界面 $x$ 方向传播。
    \item \textbf{垂直界面衰减}:振幅因子 $e^{-z/d_p}$,沿垂直方向 $z$ 指数衰减。
\end{itemize}
穿透深度 (Penetration Depth) $d_p$:
\begin{equation}
    d_p = \frac{\lambda_1}{2\pi \sqrt{\sin^2 i_1 - (n_2/n_1)^2}}
\end{equation}
通常 $d_p$ 仅为波长量级。倏逝波不向 $z$ 方向传输平均能流,能量被束缚在界面附近。

\section{近场光学显微镜 (Near-field Optical Microscopy)}

\begin{spacing}{1.5}

\subsection{远场衍射极限 (Far-field Diffraction Limit)}
传统光学显微镜接收的是从物体传播到远处的\textbf{传播波 (Propagating Wave)}。
根据瑞利判据或阿贝成像原理,其分辨率受限于波长:
\begin{equation}
    \delta \approx \frac{\lambda}{2 \text{NA}} \approx \frac{\lambda}{2}
\end{equation}
物体的高频空间信息(对应微小细节)往往携带在倏逝波中。由于倏逝波在垂直传播方向上指数衰减,无法到达远场,导致细节信息丢失。

\subsection{近场光学原理 (SNOM/NSOM)}
\mydef{核心思想}
利用极细的探针(孔径 $a \ll \lambda$)伸入到距离样品表面极近的区域(近场区,$z < \lambda$),通过探测或干扰界面上的\textbf{倏逝波},将其转化为可传输的传播波,从而被探测器接收。

\subsection{扫描近场光学显微镜 (SNOM)}
\textbf{结构特点}:
\begin{itemize}
    \item \textbf{探针}:通常为拉制的锥形光纤,尖端镀金属膜,留有几十纳米的通光孔。
    \item \textbf{距离控制}:利用剪切力 (Shear Force) 反馈系统,保持探针与样品间距恒定(约几纳米)。
    \item \textbf{扫描方式}:逐点扫描成像。
\end{itemize}

\textbf{分辨率}:
不再受光波长 $\lambda$ 限制,而是取决于\textbf{探针孔径大小}(可达 $\SI{20}{nm} \sim \SI{50}{nm}$)和\textbf{探针与样品的距离}。

\begin{figure}[htbp]
    \centering
    [图片: 倏逝波探测原理与SNOM探针示意图]
    \caption{近场光学探测原理:探针将束缚在表面的倏逝波转化为传播波}
    \label{fig:snom}
\end{figure}
\end{spacing}
\section{干涉装置和光场时空相干性}

\subsection{分波前装置}
全部将其转化为杨氏双缝干涉:
\begin{itemize}
    \item \textbf{菲涅耳双棱镜}:利用折射产生两个虚光源。$\Delta x = \frac{(B+C)\lambda}{2\alpha B}$
    \insertfigure{picture/2025-12-24-17-33-17.png}{0}{fig:0}
    \item \textbf{菲涅耳双面镜}:利用反射产生两个虚光源。$\Delta x = \frac{(B+C)\lambda}{2(n-1)\alpha B}$
    \insertfigure{picture/12}{12}{fig:12}
    \item \textbf{洛埃镜}:利用直射光与反射光干涉。$\Delta x = \frac{D\lambda}{2\alpha}$
    \insertfigure{picture/2025-12-24-17-37-11.png}{0}{fig:0}
    \textbf{重要现象}:在镜面边缘接触点处(光程差趋于0),出现暗条纹。这直接证明了光在光密介质表面反射时存在半波损失(相位突变 $\pi$)。
\end{itemize}

\subsection{点源位移导致条纹移动}
当点光源 $S$ 偏离光轴时,零级亮纹的位置也会发生改变。
\insertfigure{picture/2025-12-25-15-10-45.png}{0}{fig:0}
\begin{enumerate}
    \item \textbf{现象}:若点光源 $S$ 在垂直于光轴方向上移动 $\delta_s$,则干涉条纹整体将向\textbf{相反}方向移动。
    \item \textbf{定量关系}:
    设光源到双缝平面的距离为 $R$(对应上文 $B$),双缝到观察屏的距离为 $D$(对应上文 $C$)。
    根据光程差公式 $\Delta r = \frac{d}{D}x + \frac{d}{R}\delta_s = 0$(零级亮纹条件),可得条纹位移 $\delta x$:
    \[ \delta x = - \frac{D}{R} \delta_s \]
    \item \textbf{物理意义}:
    \begin{itemize}
        \item 负号表示条纹移动方向与光源移动方向相反。
        \item 位移量被放大了 $D/R$ 倍(干涉系统的放大倍率)。
    \end{itemize}
\end{enumerate}

\subsection{空间相干性}
只考虑同一时刻 $\tau=0$ 不同空间点 $\vec{r}_1, \vec{r}_2$ 的相关性。
\mydef{物理根源}
光源的扩展线度(非点光源)。
\insertfigure{picture/2025-12-25-15-11-50.png}{0}{fig:0}
\textbf{相干面积}:相干度下降到一定值(基本为$b_0^2$的空间范围)。
$$ \text{相干孔径角} \, \theta_c \approx \frac{\lambda}{b} $$
 $$\theta_c = \frac{b}{R} $$

\subsection{时间相干性}
只考虑同一点 $\vec{r}$ 在不同时刻的相关性,即 $\vec{r}_1 = \vec{r}_2$。
\mydef{物理根源}
光源的非单色性(频谱宽度 $\Delta \nu$)。


这意味着:\textbf{光谱越窄(单色性越好),相干时间越长}。

\mydef{相干时间与相干长度}
\begin{itemize}
    \item \textbf{相干时间} $\tau_c \approx 1 / \Delta \nu$。
    \item \textbf{相干长度} $L_c = c \tau_c \approx \lambda^2 / \Delta \lambda$。
\end{itemize}



\section{一维多元结构:光栅衍射}

光栅是大量等宽、等间距的平行狭缝(或反射面)的组合。它是多光束干涉与单缝衍射的综合体现。

\subsection{光栅衍射公式}
设光栅有 $N$ 条狭缝,缝宽为 $a$,不透光部分宽度为 $b$,光栅常数(周期) $d = a + b$。
对于波长 $\lambda$,衍射角 $\theta$,光强分布 $I(\theta)$ 为:
\begin{equation}
    I(\theta) = I_0 \underbrace{\left( \frac{\sin \alpha}{\alpha} \right)^2}_{\text{单缝衍射因子}} \cdot \underbrace{\left( \frac{\sin (N \delta )}{\sin (\delta )} \right)^2}_{\text{多缝干涉因子}}
\end{equation}
其中参数定义为:
\begin{itemize}
    \item \textbf{半波带相位差}:$\alpha = \frac{\pi a}{\lambda} \sin \theta$
    \item \textbf{槽间相位差}:$\delta = \frac{\pi d}{\lambda} \sin \theta$
\end{itemize}

\subsection{主极大与极小}
\mydef{主极大}
由干涉因子决定。当分母 $\sin(\delta) = 0$ 时,利用洛必达法则可得干涉因子趋于 $N^2$。
条件:$\delta = k\pi \quad (k = 0, \pm 1, \pm 2, \dots)$。
即光栅方程:
\begin{equation}
    d \sin \theta = k \lambda
\end{equation}
\textbf{特征}:亮度极大($I \propto N^2$),条纹细锐。

\mydef{极小值}
当分子 $\sin(N\delta) = 0$ 但分母不为 0 时,光强为 0。
条件:$N\delta = m\pi$ 且 $m \neq N \cdot \text{整数}$。
\textbf{特征}:在两个主极大之间,有 $N-1$ 个暗纹。

\mydef{次极大}
在两个主极大之间,存在 $N-2$ 个次极大。随着 $N$ 增大,次极大光强迅速衰减(相比主极大可忽略),背景变得非常干净。

\subsection{半角宽度}
主极大条纹具有一定的宽度。通常用**半角宽度** $\Delta \theta$ 来衡量,定义为主极大中心到第一极小值(零点)之间的角距离。

由光栅衍射极小值条件 $N \delta = k' \pi$(其中 $k'$ 为非 $N$ 整数倍的整数),第一极小值对应 $N \delta$ 比主极大多 $\pi$。
\begin{equation}
    N \frac{\pi d}{\lambda} \sin(\theta + \Delta \theta) - N \frac{\pi d}{\lambda} \sin \theta = \pi
\end{equation}
利用微分近似 $d(\sin \theta) = \cos \theta \, d\theta$,可得:
\begin{equation}
    \Delta \theta = \frac{\lambda}{N d \cos \theta}
\end{equation}
\textbf{物理意义}:
\begin{itemize}
    \item $N$ 越大(缝数越多),条纹越细锐。
    \item $\cos \theta$ 在分母,意味着衍射角越大,条纹越宽。
    \item 有效宽度(线宽度)与总缝数 $N$ 成反比。
\end{itemize}

\subsection{角、线色散本领和色分辨本领}

\subsubsection{角色散本领 (Angular Dispersion)}
描述光栅将不同波长的光在空间角度上分开的能力。定义为衍射角 $\theta$ 对波长 $\lambda$ 的变化率。
对光栅方程 $d \sin \theta = k \lambda$ 俩边微分:
\begin{equation}
    D = \frac{d\theta}{d\lambda} = \frac{k}{d \cos \theta}
\end{equation}
\textbf{结论}:级次 $k$ 越高、光栅常数 $d$ 越小,角色散越大。

\subsubsection{色分辨本领 (Resolving Power)}
描述光栅分辨两条靠得很近的谱线 $\lambda$ 和 $\lambda + \Delta \lambda$ 的能力。
根据\textbf{瑞利判据}:波长为 $\lambda + \Delta \lambda$ 的主极大,恰好落在波长为 $\lambda$ 的第一极小值位置时,两谱线刚好能分辨。
\begin{equation}
    R = \frac{\lambda}{\Delta \lambda} = k N
\end{equation}
\textbf{结论}:分辨本领正比于光谱级次 $k$ 和光栅总缝数 $N$。要提高分辨率,最有效的方法是增加 $N$(即加宽光栅宽度或减小 $d$)。



\subsection{缺级现象}
当某一干涉主极大级次 $k$ 的位置,恰好落在单缝衍射的暗纹位置(衍射因子为 0)时,该级主极大消失。
\begin{itemize}
    \item 干涉主极大条件:$d \sin \theta = k \lambda$
    \item 衍射暗纹条件:$a \sin \theta = m' \lambda$
\end{itemize}
两式相比得\textbf{缺级条件}:
\begin{equation}
    \frac{d}{a} = \frac{k}{m'}
\end{equation}
例如:若 $d=3a$,则 $k = \pm 3, \pm 6, \dots$ 等级次缺级。

\subsection{闪耀光栅}
\textbf{问题}:普通平面光栅的主要能量集中在 0 级(无色散),高级次光谱亮度很低。
\textbf{原理}:通过刻划锯齿状槽面,使每个狭缝(刻痕)表面倾斜一个角度 $\theta_b$(闪耀角)。
\begin{itemize}
    \item \textbf{几何控制}:控制单缝衍射因子(槽面反射光)的中央主极大位置,使其从 0 级移到某一特定的干涉主极大 $k$ 级位置。
    \item \textbf{闪耀波长}:当入射光垂直于光栅平面入射时,满足闪耀条件的波长 $\lambda_b$ 约为:
    \begin{equation}
        2 d \sin \theta_b \approx k \lambda_b \quad (\text{近似反射公式})
    \end{equation}
\end{itemize}
\insertfigure{picture/2026-01-04-15-11-09.png}{0}{fig:0}
\textbf{效果}:能量集中在特定级次,极大提高了光谱仪的效率。
\insertfigure{picture/2026-01-04-15-10-33.png}{0}{fig:0}
\subsection{晶体衍射}
可见光波长太长,无法被晶格(间距 $\sim 10^{-10}$ m)衍射。需使用 X 射线(波长 $\sim 0.1$ nm)。

\subsubsection{布拉格条件}
视晶体为一组组平行的原子晶面,间距为 $d$。X 射线在晶面上发生镜面反射并干涉。
\begin{itemize}
    \item 注意:此处 $\theta$ 定义为\textbf{掠射角}(入射光与晶面的夹角),而非光学中常用的入射角(与法线夹角)。
    \item 光程差:$\Delta L = 2d \sin \theta$。
\end{itemize}
\textbf{布拉格方程}:
\begin{equation}
    2d \sin \theta = k \lambda \quad (k=1, 2, \dots)
\end{equation}


\subsubsection{劳厄斑和德拜相}
\begin{itemize}
    \item \textbf{劳厄斑}:
    \begin{itemize}
        \item \textbf{对象}:单晶体。
        \item \textbf{光源}:连续谱 X 射线(白光)。
        \item \textbf{现象}:由于不同晶面间距 $d$ 不同,会自动“挑选”满足布拉格条件的特定 $\lambda$ 形成衍射,胶片上出现不连续的斑点。用于测定晶体取向和对称性。
    \end{itemize}
    
    \item \textbf{德拜-谢乐圆环}:
    \begin{itemize}
        \item \textbf{对象}:多晶体(粉末)。
        \item \textbf{光源}:单色 X 射线。
        \item \textbf{现象}:粉末中微小晶粒取向随机,满足衍射条件的反射光线形成圆锥面。在平面底片上呈现一系列同心圆环。用于物相分析。
    \end{itemize}
\end{itemize}

\subsection{分形光学基础(扩展,不做要求)}

分形 (Fractal) 的核心特征是\textbf{自相似性},即局部与整体在形态上相似。

\subsubsection{分形结构的衍射}
当衍射屏的透光孔径具有分形结构时,其远场衍射图样(频谱)也往往表现出分形特征。

\mydef{康托尔集}
最简单的一维分形。
\begin{itemize}
    \item \textbf{生成规则}:将一条线段三等分,去掉中间一段;对剩下的两段重复此操作,无限迭代。
    \item \textbf{分维数}:$D_f = \frac{\ln 2}{\ln 3} \approx 0.6309$。
\end{itemize}

\subsubsection{康托尔集的衍射图样}
\begin{itemize}
    \item \textbf{结构自相似}:康托尔光栅在空间结构上具有 $S=3$ 的标度不变性。
    \item \textbf{频谱自相似}:其衍射光强分布在空间频率域也表现出“层层嵌套”的包络结构。
    \item \textbf{主要结论}:$N$ 级康托尔集的衍射因子是各级单缝衍射因子与多缝干涉因子的无穷乘积。衍射图样中会出现“光强隙”,对应于空间结构中的“空隙”。
\end{itemize}

\subsubsection{二维分形:谢尔宾斯基地毯与三角}
\begin{itemize}
    \item \textbf{谢尔宾斯基地毯}:将正方形九等分,挖去中间一个;迭代。
    \item \textbf{衍射特征}:衍射图样也是二维分形结构,具有极丰富的高频分量和嵌套的零点分布。
\end{itemize}

\section{二维傅里叶变换 (Two-Dimensional Fourier Transform)}

在波动光学中,空间复振幅分布 $U(x, y)$ 与其空间频谱 $\mathcal{F}\{U\}$ 构成傅里叶变换对。这揭示了光场在**空域 (Space Domain)** 和**频域 (Frequency Domain)** 的对偶关系。

\subsection{定义与存在条件}
设 $f(x, y)$ 为空间域函数,$u, v$ 为空间频率变量(单位:$\text{mm}^{-1}$)。
\mydef{正变换 (Forward Transform)}
\begin{equation}
    F(u, v) = \mathcal{F}\{f(x, y)\} = \iint_{-\infty}^{\infty} f(x, y) e^{-\ii 2\pi (ux + vy)} \dd x \dd y
\end{equation}

\mydef{逆变换 (Inverse Transform)}
\begin{equation}
    f(x, y) = \mathcal{F}^{-1}\{F(u, v)\} = \iint_{-\infty}^{\infty} F(u, v) e^{\ii 2\pi (ux + vy)} \dd u \dd v
\end{equation}

\textbf{物理意义}:逆变换公式表明,任意复杂的波场 $f(x, y)$ 都可以看作是无穷多个不同方向、不同振幅和相位的**平面波** $e^{\ii 2\pi (ux + vy)}$ 的线性叠加。$F(u, v)$ 即为这些平面波的复振幅权重(角谱)。

\section{常用函数及其频谱 (Common Functions)}

钟锡华教材非常强调对这几个“积木”函数的熟练掌握,它们是分析复杂光学系统的基础。

\subsection{矩形函数 (Rectangle Function)}
定义:$\rect(x) = \begin{cases} 1, & |x| < 1/2 \\ 0, & |x| > 1/2 \end{cases}$
\begin{itemize}
    \item \textbf{二维矩孔}:$f(x, y) = \rect(\frac{x}{a}) \rect(\frac{y}{b})$
    \item \textbf{频谱}:$F(u, v) = |ab| \sinc(au) \sinc(bv)$
    \item \textbf{注}:钟书定义 $\sinc(x) = \frac{\sin(\pi x)}{\pi x}$。
\end{itemize}

\subsection{圆域函数 (Circle Function)}
定义:$\circfunc(r) = \begin{cases} 1, & r < 1 \\ 0, & r > 1 \end{cases}$,其中 $r = \sqrt{x^2+y^2}$。
\begin{itemize}
    \item \textbf{二维圆孔}:$f(x, y) = \circfunc(\frac{\sqrt{x^2+y^2}}{a})$
    \item \textbf{频谱}:$F(\rho) = |a|^2 \frac{J_1(2\pi a \rho)}{a \rho}$,其中 $\rho = \sqrt{u^2+v^2}$。
    \item 这是艾里斑强度的振幅基础。
\end{itemize}

\subsection{狄拉克 $\delta$ 函数 (Delta Function)}
\mydef{筛选性质}
$\int_{-\infty}^{\infty} f(x) \delta(x-x_0) \dd x = f(x_0)$。
\begin{itemize}
    \item $\mathcal{F}\{\delta(x, y)\} = 1$ (点源的频谱是均匀的平面波谱)。
    \item $\mathcal{F}\{1\} = \delta(u, v)$ (均匀平面波对应单一空间频率)。
\end{itemize}

\subsection{梳状函数 (Comb Function)}
定义:$\comb(x) = \sum_{n=-\infty}^{\infty} \delta(x-n)$。
\begin{itemize}
    \item \textbf{性质}:$\mathcal{F}\{\comb(x)\} = \comb(u)$。梳状函数的傅里叶变换仍是梳状函数。
    \item \textbf{应用}:描述光栅、采样过程。
\end{itemize}

\section{基本定理与卷积 (Basic Theorems)}

\subsection{位移定理 (Shift Theorem)}
\begin{itemize}
    \item \textbf{空域位移} $\leftrightarrow$ \textbf{频域线性相移}:
    $$ f(x-x_0, y-y_0) \Longleftrightarrow F(u, v) e^{-\ii 2\pi (u x_0 + v y_0)} $$
    \item \textbf{频域位移} $\leftrightarrow$ \textbf{空域线性相移}:
    $$ f(x, y) e^{\ii 2\pi (u_0 x + v_0 y)} \Longleftrightarrow F(u-u_0, v-v_0) $$
    这是**斜入射**照明分析的基础。
\end{itemize}

\subsection{卷积定理 (Convolution Theorem)}
定义卷积:$g(x, y) = f(x, y) * h(x, y) = \iint f(\xi, \eta) h(x-\xi, y-\eta) \dd \xi \dd \eta$。
\begin{itemize}
    \item \textbf{时域卷积} $\leftrightarrow$ \textbf{频域乘积}:
    $$ \mathcal{F}\{f * h\} = \mathcal{F}\{f\} \cdot \mathcal{F}\{h\} $$
    这构成了**线性系统理论**的基础:输出 = 输入 * 脉冲响应 $\implies$ 输出频谱 = 输入频谱 $\cdot$ 传递函数。
\end{itemize}

\subsection{巴塞伐尔定理 (Parseval's Theorem)}
能量守恒在空域和频域的体现:
$$ \iint |f(x, y)|^2 \dd x \dd y = \iint |F(u, v)|^2 \dd u \dd v $$

\section{相因子分析 (Phase Factor Analysis)}

这是本章的精华。在傍轴光学中,光场的变换主要体现为相位的调制。钟锡华老师将复杂的衍射积分简化为对两种核心相因子的操作。

\subsection{线性相因子 (Linear Phase Factor)}
形式:$e^{\ii 2\pi u_0 x}$ 或 $e^{\ii k x \sin \theta}$。
\begin{itemize}
    \item \textbf{物理意义}:代表光波传播方向的改变(倾斜)。
    \item \textbf{作用}:
        \begin{itemize}
            \item 在**空域**乘以线性相因子 $\implies$ **频谱**发生平移(位移定理)。
            \item 例如:棱镜的作用就是引入线性相因子。
        \end{itemize}
\end{itemize}

\subsection{二次相因子 (Quadratic Phase Factor)}
形式:$e^{\ii \frac{k}{2R} (x^2+y^2)}$。
\begin{itemize}
    \item \textbf{物理意义}:傍轴近似下的**球面波**。
    \item \textbf{参数符号}:
        \begin{itemize}
            \item $R > 0$:发散球面波(点源在左)。
            \item $R < 0$:会聚球面波(焦点在右)。
        \end{itemize}
    \item \textbf{自由空间传播}:
    菲涅耳衍射积分核即包含二次相因子 $e^{\ii \frac{k}{2z} (x^2+y^2)}$。这表明光在自由空间传播一段距离 $z$,相当于与一个二次相因子进行卷积。
\end{itemize}

\subsection{薄透镜的位相变换作用}
透镜是光学信息处理的核心元件。
\mydef{透镜的复振幅透射率}
$$ t_l(x, y) = e^{-\ii \frac{k}{2f} (x^2+y^2)} $$
(忽略常数相位因子 $e^{\ii k n \Delta_0}$)
\begin{itemize}
    \item \textbf{物理图像}:透镜将入射的平面波(波前平直)转化为会聚球面波(波前弯曲)。
    \item \textbf{相位抵消}:若入射光是发散球面波 $e^{\ii \frac{k}{2d_1}(x^2+y^2)}$,经过透镜后:
    $$ U_{out} = e^{\ii \frac{k}{2d_1}(x^2+y^2)} \cdot e^{-\ii \frac{k}{2f}(x^2+y^2)} = e^{\ii \frac{k}{2} (\frac{1}{d_1} - \frac{1}{f}) (x^2+y^2)} $$
    若满足成像公式 $\frac{1}{d_1} + \frac{1}{d_2} = \frac{1}{f}$,则输出变为 $e^{-\ii \frac{k}{2d_2}(x^2+y^2)}$,即会聚到 $d_2$ 处的球面波。
\end{itemize}

\subsection{透镜的傅里叶变换性质}
这是傅里叶光学的核心结论。
\textbf{结论}:若物体置于透镜的**前焦面**,则在透镜的**后焦面**上,光场分布准确地正比于物函数的傅里叶变换。
$$ U_f(x_f, y_f) \propto \mathcal{F}\{t_0(x_0, y_0)\} \bigg|_{u = \frac{x_f}{\lambda f}, v = \frac{y_f}{\lambda f}} $$
透镜的作用就是在后焦面上物理地实现二维傅里叶变换。

\section{全息术的基本原理}
\insertfigure{picture/2026-01-04-15-34-56.png}{0}{fig:0}
如图所示,一激光束经显微镜头、分束器、反射镜等元件,被扩束、分解和变向,而生成两束宽孔径的相干光束。其中,一束直接投射到记录介质(乳胶干板)$H$ 面上,称其为参考光波 $\tilde{R}$;另一束投射到物体上,经物体上各点漫反射而形成一物光波 $\tilde{O}$,传播到记录介质 $H$ 面上。于是,记录介质平面 $H$ 上存在 $\tilde{O}$ 光与 $\tilde{R}$ 光的干涉场。

$$ \tilde{U}_H(x,y) = \tilde{O}(x,y) + \tilde{R}(x,y) \eqno{(7.1)} $$

其中,参考光 $\tilde{R}$ 通常被调节为平面波或球面波,其传播方向或倾角,也可以人为地控制;而物光波 $\tilde{O}$,从微观上看,是物体上各点源发出的大量次波的相干叠加,

$$ \tilde{O}(x,y) = \sum_n \tilde{u}_n(x,y) = A_O(x,y) \cdot e^{i\varphi_O(x,y)} \eqno{(7.2)} $$

其中,每一次波 $\tilde{u}_n(x,y)$,决定于相应物点的亮度和位置,$H$ 面上存在的这 $\tilde{O}$ 光波前,正是这些次波的自相干场,其振幅分布 $A_O(x,y)$,尤其是相位分布 $\varphi_O(x,y)$ 反映了物体的三维形貌或形象,虽然记录的是二维光场信息。

当然,记录介质感受的依然是光强分布,这与普通照相的胶片并无区别。那么,$\tilde{O}$ 波与 $\tilde{R}$ 波叠加的干涉强度分布为

$$
\begin{aligned}
I_H(x,y) &= \tilde{U}_H \cdot \tilde{U}_H^* = (\tilde{O} + \tilde{R}) \cdot (\tilde{O}^* + \tilde{R}^*) \\
&= |\tilde{O}|^2 + |\tilde{R}|^2 + \tilde{R}^* \cdot \tilde{O} + \tilde{R} \cdot \tilde{O}^* \\
&= A_O^2(x,y) + A_R^2(x,y) + A_R e^{-i\varphi_R} \cdot \tilde{O} + A_R e^{i\varphi_R} \cdot \tilde{O}^* \end{aligned}
\eqno{(7.3)}
$$

这里,我们将人为安排的参考光波 $\tilde{R}$ 的波前函数表示为

$$ \tilde{R}(x,y) = A_R(x,y) \cdot e^{i\varphi_R(x,y)} \eqno{(7.4)} $$

这光强分布 $I_H$ 要被记录或存储下来,还必须经化学溶液的处理,即所谓的显影和定影,简言之“冲洗”,且要求这次冲洗满足线性条件——冲洗后的这张底片,其透过率函数 $\tilde{t}_H$ 与干涉强度函数 $I_H$ 之间是线性关系,

$$
\begin{aligned}
\tilde{t}_H(x,y) &= t_0 + \beta I_H(x,y) \\
&= t_0 + \beta (A_O^2 + A_R^2) + \beta \tilde{R}^* \cdot \tilde{O} + \beta \tilde{R} \cdot \tilde{O}^*
\end{aligned}
\eqno{(7.5)}
$$

这里,$t_0, \beta$ 是常数。于是,便制成了一张全息图。

\subsection{全息图的衍射场 —— 相因子分析法的运用}
\insertfigure{picture/2026-01-04-15-35-29.png}{0}{fig:0}
用一准单色光波 $\tilde{R}'$ 照射一张全息图,如图所示。那么,这全息图作为一个衍射屏,在 $\tilde{R}'$ 波照射下,将产生一复杂的衍射场,其波前函数为

$$
\begin{aligned}
\tilde{U}'_H(x,y) &= \tilde{t}_H \cdot \tilde{R}' \\
&= (t_0 + \beta A_R^2 + \beta A_O^2) \cdot \tilde{R}' + \beta \tilde{R}' \tilde{R}^* \cdot \tilde{O} + \beta \tilde{R}' \tilde{R} \cdot \tilde{O}^* \\
&= \tilde{T}_1 \cdot \tilde{R}' + \tilde{T}_2 \cdot \tilde{O} + \tilde{T}_3 \cdot \tilde{O}^*
\end{aligned}
\eqno{(7.6)}
$$

其中,照射光波 $\tilde{R}'$ 的波前函数可表示为

$$ \tilde{R}'(x,y) = A'_R(x,y) e^{i\varphi'_R(x,y)} \eqno{(7.7)} $$

凭借相因子分析法,在不同记录或照射条件下,可以逐项解析那三个操作系数的变换作用。具体说明如下:

\begin{itemize}
    \item[(1)] \textbf{变换因子 $\tilde{T}_1$},按 (7.6) 式,
    $$ \tilde{T}_1 = (t_0 + \beta A_R^2 + \beta A_O^2) \eqno{(7.8)} $$
    一般情况下,参考波 $\tilde{R}$ 是一列平面波或傍轴球面波,故其振幅分布 $A_R \approx$ 常数,而原物光波的振幅分布 $A_O(x,y)$ 虽复杂,但可近似考虑 $A_O \approx$ 常数。这样,变换因子 $\tilde{T}_1 \approx$ 常数,$\tilde{T}_1 \cdot \tilde{R}'$ 就表示了照射光波 $\tilde{R}'$ 的直接透射波,也就是全息图的 0 级衍射波。

    \item[(2)] \textbf{变换因子 $\tilde{T}_2$ 和 $\tilde{T}_3$},按 (7.6) 式,
    $$ \tilde{T}_2 = \beta \tilde{R}' \tilde{R}^* = \beta A'_R A_R e^{i(\varphi'_R - \varphi_R)} \eqno{(7.9)} $$
    $$ \tilde{T}_3 = \beta \tilde{R}' \tilde{R} = \beta A'_R A_R e^{i(\varphi'_R + \varphi_R)} \eqno{(7.10)} $$
\end{itemize}

\subsubsection{几种典型情况分析}

\textbf{典型情况之一:} $\tilde{R}'$ 波与 $\tilde{R}$ 波系同平面波,且正入射。这时,可设 $\varphi'_R = \varphi_R = 0$,于是 $\tilde{T}_2 = \tilde{T}_3 = \beta A'_R A_R =$ 常数,这就表明,

$$ \tilde{T}_2 \cdot \tilde{O} = \beta A'_R A_R \tilde{O} \quad \text{—— 物光波前的再现} $$
$$ \tilde{T}_3 \cdot \tilde{O}^* = \beta A'_R A_R \tilde{O}^* \quad \text{—— 物光共轭波前的伴生} $$

前者为 +1 级衍射波,是发散的,生成一虚像;后者为 -1 级衍射波,是会聚的,生成一实像。而且,在目前条件下,两者镜像对称,与原物尺寸亦相等。我们称这种情况下的这一对孪生像为“原生像”。

\textbf{典型情况之二:} $\tilde{R}'$ 波与 $\tilde{R}$ 波系同球面波,且斜入射。这时,相位分布函数 $\varphi'_R = \varphi_R =$ 线性相因子,于是
$$ \tilde{T}_2 = \beta A'_R A_R \approx \text{常数} $$
$$ \tilde{T}_3 = \beta A'_R A_R e^{i2\varphi_R} \quad \text{—— 等效棱镜} $$
显然,$\tilde{T}_2 \cdot \tilde{O}$ 项表示了原物光波前的真实再现,而 $\tilde{T}_3 \cdot \tilde{O}^*$ 项表明孪生的共轭波 $\tilde{O}^*$ 受到一等效棱镜的作用,发生了偏转。

\textbf{典型情况之三:} $\tilde{R}'$ 波与 $\tilde{R}$ 波系同球面波。这时,相位分布函数 $\varphi'_R = \varphi_R =$ 二次相因子,于是
$$ \tilde{T}_2 = \beta A'_R A_R \approx \text{常数} $$
$$ \tilde{T}_3 = \beta A'_R A_R e^{i2\varphi_R} \quad \text{—— 等效透镜} $$
故 $\tilde{T}_2 \cdot \tilde{O}$ 项表示了原物光波前的真实再现;而 $\tilde{T}_3 \cdot \tilde{O}^*$ 项表明孪生共轭波 $\tilde{O}^*$ 受到一等效透镜的作用,发生了移位、缩放和偏转。

\textbf{典型情况之四:} $\tilde{R}'$ 波与 $\tilde{R}$ 波互为一对共轭波。这时,波前函数 $\tilde{R}' = \tilde{R}^*$。例如,记录时参考波 $\tilde{R}$ 是一自上而下斜射的发散球面波束,则照射光 $\tilde{R}'$ 是一自下而上斜射的会聚球面波束。于是,相位函数 $\varphi'_R = -\varphi_R$,故 $\varphi'_R - \varphi_R = -2\varphi_R$,$\varphi'_R + \varphi_R = 0$。
此时 $\tilde{T}_3 \cdot \tilde{O}^*$ 项倒是一个真实的原物孪生像(实像)。

\subsubsection{波长变换}
照射光波 $\tilde{R}'$ 与参考波 $\tilde{R}$ 的波长可以不同,$\lambda' \neq \lambda$。例如,X 光全息图,可以用可见光照射而再现物光波前。其结果是再现物与原物相比,在几何上有一缩小或放大,其放大率正比于波长之比值:

$$ V \propto \frac{\lambda'}{\lambda} \eqno{(7.11)} $$

这为缩放图像提供了一新的技术途径。


\section{晶体光学}

\subsection{双折射现象}
当光束射入各向异性晶体时,一般会分裂成两束折射光,这种现象称为双折射。
\begin{itemize}
    \item \textbf{o光 (寻常光)}:遵守折射定律,光速 $v_o$ 和折射率 $n_o$ 在各个方向上是一个常数,波面为球面。
    \item \textbf{e光 (非寻常光)}:一般不遵守折射定律,光速 $v_e$ 和折射率 $n_e$ 随方向而改变,波面为旋转椭球面。
\end{itemize}

\subsubsection{折射率椭球}
为了描述晶体中折射率随光振动方向的变化,引入折射率椭球方程:
$$ \frac{x^2}{n_x^2} + \frac{y^2}{n_y^2} + \frac{z^2}{n_z^2} = 1 $$
对于单轴晶体(如方解石),以光轴为旋转轴,$n_x=n_y=n_o, n_z=n_e$。
\begin{itemize}
    \item \textbf{主折射率}:$n_o$ 和 $n_e$。
    \item \textbf{任意方向折射率}:设波法线与光轴夹角为 $\theta$,e光的折射率 $n'(\theta)$ 满足:
    $$ \frac{1}{n'^2} = \frac{\cos^2\theta}{n_o^2} + \frac{\sin^2\theta}{n_e^2} $$
\end{itemize}

\subsection{晶体偏振器件}
利用晶体的双折射性质,可以制成各种功能的偏振器件。

\subsubsection{晶体棱镜}
主要用于产生线偏振光或分离两束正交偏振光。
\insertfigure{picture/2026-01-04-15-50-30.png}{0}{fig:0}
\begin{itemize}
    \item \textbf{尼科耳棱镜 (Nicol)}:利用全反射原理将 o 光反射吸收,只透射 e 光,获得线偏振光。
    \item \textbf{洛雄棱镜 (Rochon)} 和 \textbf{沃拉斯顿棱镜 (Wollaston)}:用于分离 o 光和 e 光,使它们具有不同的传播方向(分束角)。
\end{itemize}

\subsubsection{波晶片}
波晶片是厚度精确的晶体薄片,光轴平行于表面。o 光和 e 光通过波片后会产生光程差和相位差。
\insertfigure{picture/2026-01-04-15-51-17.png}{0}{fig:0}
\begin{itemize}
    \item \textbf{相位差公式}:
    $$ \delta = \frac{2\pi}{\lambda_0} (n_e - n_o) d $$
    其中 $d$ 为波片厚度。
    \item \textbf{四分之一波片 ($\lambda/4$片)}:$\delta = \pi/2$。
        \begin{itemize}
            \item 作用:将线偏振光转换为圆偏振光或椭圆偏振光,反之亦然。
        \end{itemize}
    \item \textbf{半波片 ($\lambda/2$片)}:$\delta = \pi$。
        \begin{itemize}
            \item 作用:使线偏振光的振动面转动 $2\theta$ 角($\theta$ 为入射偏振面与波片快轴的夹角)。
        \end{itemize}
    \item \textbf{补偿器}:如巴比涅补偿器、索雷尔补偿器。可以提供线性可变的相位差,用于精确测量相位。
\end{itemize}
\insertfigure{picture/2026-01-04-15-53-07.png}{0}{fig:0}
\subsection{偏振光的干涉}
典型的偏振光干涉装置由“起偏器 $P_1$ — 波晶片 — 检偏器 $P_2$”组成。

\textbf{干涉光强公式}:
设 $P_1$ 与 $P_2$ 的透光轴夹角为 $\gamma$(图中为 $\beta$ 关联项),波片引入相位差为 $\delta$。
\insertfigure{picture/2026-01-04-15-49-54.png}{0}{fig:0}
\begin{itemize}
    \item 当 $P_1 \parallel P_2$ (平行尼科耳) 时:
    $$ I_{\parallel} = I_0 \cos^2 \frac{\delta}{2} $$
    这时呈现互补色。
    \item 当 $P_1 \perp P_2$ (正交尼科耳) 时:
    $$ I_{\perp} = I_0 \sin^2 2\alpha \sin^2 \frac{\delta}{2} $$
    其中 $\alpha$ 是入射偏振光与波片光轴的夹角。
    \begin{itemize}
        \item 若 $\alpha = 45^\circ$(对角位),则 $I = I_0 \sin^2 \frac{\delta}{2}$,干涉最明显。
        \item 若 $\alpha = 0^\circ$ 或 $90^\circ$(消光位),则 $I=0$,无干涉。
    \end{itemize}
\end{itemize}

\subsection{旋光性与电光效应}
\subsubsection{旋光性}
某些物质(如石英、糖溶液)能使线偏振光的振动面发生旋转。
\insertfigure{picture/2026-01-04-15-53-39.png}{0}{fig:0}
\begin{itemize}
    \item \textbf{菲涅耳唯象解释}:线偏振光可分解为一束左旋圆偏振光和一束右旋圆偏振光。在旋光介质中,左旋和右旋光的折射率不同($n_L \neq n_R$),传播速度不同,导致出射时合成矢量的方向发生偏转。
    \item \textbf{旋光角公式}:$\psi = \alpha d$ ($\alpha$ 为旋光率)。
\end{itemize}

\subsubsection{电光效应}
外加电场改变介质折射率的现象。
\begin{itemize}
    \item \textbf{克尔效应 (Kerr Effect)}:各向同性介质在强电场下变为双折射介质,诱导的双折射率差 $\Delta n$ 与电场平方成正比:
    $$ \Delta n \propto \lambda E^2 $$
    \item \textbf{泡克耳斯效应 (Pockels Effect)}:线性电光效应,折射率变化与电场 $E$ 的一次方成正比。
\end{itemize}


\end{document} % 文档结束