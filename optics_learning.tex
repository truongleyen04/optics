% !TEX program = xelatex
% ===============================================================
% LaTeX 论文模板 - 优化学习版 v2.0
% 说明:此版本强调了代码的结构化、语义化与排版最佳实践。
% 建议使用 XeLaTeX 引擎编译,以完美支持中文。
% ===============================================================

% %%%%%%%%%%%%%%%%%%%%%%%%%%%%%%%%%%%%%%%%%%%%%%%%%%%%%
% 1. 前言区 (Preamble)
% --- 文档类型定义与宏包加载 ---
% %%%%%%%%%%%%%%%%%%%%%%%%%%%%%%%%%%%%%%%%%%%%%%%%%%%%%

\documentclass[12pt]{ctexart} % 使用 ctexart 文档类,专为中文排版设计。12pt 为默认字体大小。

% --- 核心功能宏包 ---
\usepackage{amsmath}        % 美国数学学会 (AMS) 宏包,提供进阶的数学公式环境。
\usepackage{amssymb}        % 提供 amsmath 未包含的额外数学符号。
\usepackage{graphicx}       % 用于插入图片的核心宏包,提供 \includegraphics 命令。
\usepackage{xcolor}         % 提供颜色支持,可用于文字、背景、表格等。
\usepackage{setspace}       % 用于灵活调整行距,例如 \onehalfspacing, \doublespacing。

% --- 排版与格式设定宏包 ---
\usepackage{geometry}       % 方便地设定页面边距与版心。
\usepackage{siunitx}        % 专业地排版带有国际单位 (SI) 的数字与单位。
\usepackage{abstract}       % 提供摘要格式的自定义功能。
\usepackage{cite}           % 美化与管理引用文献标号,例如将 [1,2,3] 压缩为 [1-3]。
\usepackage[colorlinks=true, linkcolor=blue, citecolor=red, urlcolor=cyan]{hyperref} % 建立 PDF 超链接,通常建议放在绝大多数宏包之后加载。
\usepackage{cleveref}       % 智能引用,可自动识别图、表、公式,生成如 “图 1” 而非仅 “1”。

% --- 页面布局设定 (geometry) ---
\geometry{
    a4paper,
    left=25mm,
    right=25mm,
    top=30mm,
    bottom=30mm,
}




% %%%%%%%%%%%%%%%%%%%%%%%%%%%%%%%%%%%%%%%%%%%%%%%%%%%%%
% 2. 自定义命令与全局设定
% --- 为了代码的简洁与一致性 ---
% %%%%%%%%%%%%%%%%%%%%%%%%%%%%%%%%%%%%%%%%%%%%%%%%%%%%%

% --- 自定义一个用于“定义”的命令 ---
% #1 代表第一个参数 (术语名称)
% 使用 \noindent 来避免段首缩进,使其看起来更像一个标题。
\newcommand{\mydef}[1]{\par\noindent\textbf{#1}:}

% --- 自定义图片插入命令 (基础版) ---
% #1: 图片路径 (例如: picture/image.png)
% #2: 图片说明 (caption)
% #3: 引用标签 (label, 无需加 fig: 前缀)
\newcommand{\insertfigure}[3]{
    \begin{figure}[htbp]
        \centering
        \includegraphics[width=0.6\textwidth]{#1}
        \caption{#2}
        \label{fig:#3}
    \end{figure}
}

\newcommand{\boxedmath}[1]{
  \fcolorbox{black}{white}{\scalebox{1.3}{#1}}
}


% --- 文档信息 ---
\title{\bfseries 光学} % 标题内容。字体大小由文档类自动处理,这里只指定粗体。
\author{施维圣}
\date{\today}                 % 使用 \today 自动生成编译当天的日期。

% %%%%%%%%%%%%%%%%%%%%%%%%%%%%%%%%%%%%%%%%%%%%%%%%%%%%%
% 3. 正文区 (Body)
% --- 所有可见的内容都在这里 ---
% %%%%%%%%%%%%%%%%%%%%%%%%%%%%%%%%%%%%%%%%%%%%%%%%%%%%%

\begin{document}

\maketitle % 自动生成标题、作者、日期区块。

% --- 摘要区 (如果需要) ---
% \begin{abstract}
%     这是一个关于几何光学基础原理的摘要...
% \end{abstract}

% --- 正文章节 ---
\section{几何光学基本原理与成像规律}
\subsection{费马原理}
\subsubsection{惠更斯原理与折射率}

% 使用 setspace 提供的环境来设定局部行距,1.5 约等于双倍行高。
% 2.5 的行距在学术写作中过于宽松,1.5 或 1.8 是更常见的选择。
\begin{spacing}{1.5}

\mydef{波面 (wave surface)}
波在同一时刻到达的各点所组成的面。一个波面上各点同时开始振动,具有相同的相位,因此波面又称同相面或等相面。

\mydef{平面波 (plane wave)}
波面为一系列平行平面的波。

\mydef{球面波 (spherical wave)}
波面为一系列同心球面的波。

\mydef{波前 (wavefront)}
指最前沿的波面。

\mydef{波线 (ray)}
沿着波传播方向的射线。在各向同性的介质中,波线恒垂直于波面。相关示意如图\ref{fig:wave}所示。

\end{spacing}

% --- 图表示例 ---
% 使用 [htbp] 参数给予 LaTeX 最大的灵活性来放置图片:
% h: here (这里)
% t: top (页面顶部)
% b: bottom (页面底部)
% p: page (独立一页)
% 常常使用 [htbp] 或 [tbp] 是最佳实践。
\begin{figure}[htbp]
    \centering
    % 使用 width=\linewidth 的相对宽度可以让图片自适应页宽,比固定 scale 更具弹性。
    % 例如 width=0.6\textwidth 表示图片宽度为文字区域宽度的 60%。
    \includegraphics[width=0.5\textwidth]{picture/wave.png}
    \caption{波前、波面与波线示意图}
    \label{fig:wave} % 使用 "类型:名称" 的标签格式 (如 fig:, tab:, eq:) 是个好习惯。
\end{figure}

% --- 使用段落来分隔不同概念,而不是手动空行 ---
\subsubsection*{惠更斯原理 (Huygens’ Principle)}
光扰动同时到达的空间曲面称为波面或波前。
波前上的每点都可以视为一个新的扰动中心,称为“子波源”。
每个子波源会向四周发出球面\textbf{“次波”(secondary wavelet)}。
在下一时刻,所有这些次波的公切面(或称为包络面)就构成了新的波前。
子波源与其次波切点的连线方向,即为该处光线的传播方向。

\begin{figure}[htbp]
    \centering
    \includegraphics[width = 0.6\textwidth]{picture/huigengsi.png}
    \caption{惠更斯原理示意图}
    \label{fig:huigengsi}
\end{figure}

\subsubsection*{折射定律的导出}

\insertfigure{picture/zheshe.png}{折射定律示意图}{zheshe}


首先,我们分析波前从 $ABC$ 传播到 $A'B'C'$ 的过程。如图\ref{fig:zheshe}所示,当波前上的 $A$ 点到达界面时,波前上的 $C$ 点还需要一段时间 $\Delta t$ 才能传播到界面上的 $C'$ 点。这段时间为:
$$
\Delta t = \frac{\overline{CC'}}{v_1}
$$

在相同的时间 $\Delta t$ 内,位于界面上 $A$ 点的子波源,会在介质2中形成一个半径为 $\overline{AA'}$ 的球面次波。其传播的距离为:
$$
\overline{AA'} = v_2 \cdot \Delta t = v_2 \frac{\overline{CC'}}{v_1}
$$

此时,从 $C'$ 点作这个球面次波的切线,切点为 $A'$。根据惠更斯原理,这条切线 $A'C'$ 即为在介质2中新的折射波前。而连接次波中心 $A$ 和切点 $A'$ 的直线 $AA'$,则代表了折射光线的传播方向,其与法线的夹角即为折射角 $i_2$。

接下来,我们在直角三角形 $\triangle AC'C$ 和 $\triangle AC'A'$ 中,可以分别得到入射角和折射角正弦值的几何关系:

\begin{align*}
\sin i_1 &= \frac{\overline{CC'}}{\overline{AC'}} \\
\sin i_2 &= \frac{\overline{AA'}}{\overline{AC'}} = \frac{v_2 \cdot \overline{CC'}}{v_1 \cdot \overline{AC'}}
\end{align*}


最后,将上述两式相除,消去公因子 $\overline{CC'}$ 和 $\overline{AC'}$,我们就得到了最终的\textbf{折射定律}:

\begin{gather*}
\frac{\sin i_1}{\sin i_2} = \frac{v_1}{v_2} = \text{const}
\end{gather*}


这个结果表明,入射角正弦值与折射角正弦值的比率,等于两介质中波速的比率,是一个常数。

定义:
$$
    n = \frac{c}{v}
$$
则有
\begin{center}
    \boxedmath{$n_1\sin i_1 = n_2\sin i_2$}
\end{center}



\subsubsection{光程概念及其意义}
% ... 内容接续 ...

\subsection{成像}
% ... 内容接续 ...

\subsection{光学仪器}
% ... 内容接续 ...

\end{document} % 文档结束