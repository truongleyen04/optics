% !TEX program = xelatex
% ===============================================================
% LaTeX 論文模板學習版
% 說明:每一行控制碼都附有中文註解
% 建議使用 XeLaTeX 引擎編譯
% ===============================================================

% %%%%%%%%%%%%%%%%%% 1. 前言區 (Preamble) %%%%%%%%%%%%%%%%%%
% --- 從這裡到 \begin{document} 之前,都屬於前言區 ---
% --- 主要用來設定整份文件的基本格式與載入功能 ---

\documentclass[12pt]{ctexart} % 定义文件的类型。所有 LaTeX 文件都必须以这个指令开始。
                                            % [a4paper] 设置纸张为 A4 尺寸,[twocolumn] 设置为双栏排版。
                                            % {ctexart} 使用 ctexart 这个文件类别,专为排版中文文章设计,art 指的是 article (文章)。

% --- 载入各种功能宏包 (Packages) ---
\usepackage{graphicx}       % 载入宏包,用于插入图片 (\includegraphics)。
\usepackage{amsmath}        % 载入宏包,美国数学学会 (AMS) 提供的数学公式增强功能。
\usepackage{amssymb}        % 载入宏包,提供更多数学符号,需与 amsmath 搭配使用。
\usepackage{siunitx}        % 载入宏包,用于排版带有国际标准单位 (SI) 的数字,格式非常专业。
\usepackage{xcolor}         % 载入宏包,让您能使用颜色,例如设定文字颜色。
\usepackage{geometry}       % 载入宏包,方便地设定页面的边界 (margin)。
\usepackage{abstract}       % 载入宏包,提供自定义摘要 (abstract) 格式的功能。
\usepackage{cite}           % 载入宏包,美化与管理引用文献的标号,例如将 [1],[2],[3] 合并为 [1-3]。
\usepackage{hyperref}       % 载入宏包,建立 PDF 文件中的超链接,让目录、引用、网址都可以点击。

% --- 页面布局设置 (使用 geometry 宏包) ---
\geometry{                  % 这是 geometry 宏包的设定区块的开始。
    a4paper,                % 再次确认纸张尺寸为 A4。
    total={180mm,257mm},    % 设定文字内容区域的总宽度与高度。
    left=15mm,              % 设定左边界为 15 毫米。
    top=20mm,               % 设定上边界为 20 毫米。
}                           % 区块设定的结束。

% --- 超链接设置 (使用 hyperref 宏包) ---
\hypersetup{                % 这是 hyperref 宏包的设定区块的开始。
    colorlinks=true,        % 将超链接设定为彩色文字,而不是预设的方框。
    linkcolor=blue,         % 设定内部链接 (如目录、图表引用) 的颜色为蓝色。
    filecolor=magenta,      % 设定文件链接的颜色为洋红色。
    urlcolor=cyan,          % 设定网址链接的颜色为青色。
    citecolor=red,          % 设定文献引用链接的颜色为红色。
    pdftitle={A Study on a Novel Nonlinear Optical Crystal}, % 设定 PDF 文件属性中的「标题」。
    pdfauthor={Author A},   % 设定 PDF 文件属性中的「作者」。
}                           % 区块设定的结束。

% --- 文件信息定义 (标题、作者、日期) ---
\title{\huge \bfseries 光学} % 设定整份文件的标题。
                                            % {\huge \bfseries ...} 是格式指令,\huge 是字体大小(极大),\bfseries 是字体样式(粗体)。

\author{                    % 设定作者信息。
    施维圣          
                        % 手动换行,并增加 1em 的垂直间距 (em 是一个相对长度单位)。
}                           % 作者信息设定的结束。

\date{\today}               % 设定文件日期。`\today` 这个特殊指令会自动填入编译当天的日期。

%%%%%%%%%%%%%%%%%%%%%%%%%%%%%%%%%%%%%%%%%%%%%%%%%%%%%%%
%                  文档正文开始
%%%%%%%%%%%%%%%%%%%%%%%%%%%%%%%%%%%%%%%%%%%%%%%%%%%%%%%
\begin{document}            % 文件的正文内容从这里开始,这是 LaTeX 的主体区域。之前都是前言区。

\maketitle                  % 这是一个非常重要的指令,它会根据前面用 \title, \author, \date 设定的信息,自动产生标题区块。

% --- 摘要区 ---
% 摘要内容放在这里... (您提供的代码中省略了摘要内容)
% \begin{abstract}
% ...
% \end{abstract}

% --- 正文章节 ---

\section{几何光学基本原理与成像规律}
\subsection{费马原理}
\subsubsection{惠更斯原理 \& 折射率}



\subsubsection{光程概念及其意义}
\subsection{成像}
\subsection{光学仪器}
\end{document}              % 文件的结束标记。所有在此之后的内容都会被 LaTeX 忽略。