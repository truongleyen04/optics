\section{晶体的光学各向异性 (Optical Anisotropy)}

在各向同性介质(如玻璃、水)中,电位移矢量 $\vec{D}$ 与电场强度 $\vec{E}$ 方向相同,即 $\vec{D} = \varepsilon \vec{E}$,折射率 $n$ 为常数。
而在**晶体**中,介质的极化性质随方向而变(各向异性),$\vec{D}$ 与 $\vec{E}$ 通常**不平行**。

\subsection{介电张量 (Dielectric Tensor)}
在晶体的主轴坐标系 $(x, y, z)$ 中,本构关系表示为张量形式:
\begin{equation}
    \begin{pmatrix} D_x \\ D_y \\ D_z \end{pmatrix} = 
    \begin{pmatrix} 
    \varepsilon_x & 0 & 0 \\ 
    0 & \varepsilon_y & 0 \\ 
    0 & 0 & \varepsilon_z 
    \end{pmatrix} 
    \begin{pmatrix} E_x \\ E_y \\ E_z \end{pmatrix}
\end{equation}
定义**主折射率 (Principal Refractive Indices)**:
\begin{equation}
    n_x = \sqrt{\frac{\varepsilon_x}{\varepsilon_0}}, \quad n_y = \sqrt{\frac{\varepsilon_y}{\varepsilon_0}}, \quad n_z = \sqrt{\frac{\varepsilon_z}{\varepsilon_0}}
\end{equation}

\subsection{晶体的分类}
\begin{itemize}
    \item **各向同性体**:$n_x = n_y = n_z$(立方晶系)。
    \item **单轴晶体 (Uniaxial Crystal)**:$n_x = n_y \neq n_z$(四方、六方、三方晶系)。
        \begin{itemize}
            \item $z$ 轴称为**光轴 (Optical Axis)**。
            \item 令 $n_x = n_y = n_o$(寻常光折射率),$n_z = n_e$(非寻常光主折射率)。
        \end{itemize}
    \item **双轴晶体 (Biaxial Crystal)**:$n_x \neq n_y \neq n_z$(正交、单斜、三斜晶系)。
\end{itemize}

\section{折射率椭球 (Index Ellipsoid)}

钟锡华教材特别推崇利用几何方法——**折射率椭球**(又称光率体)来分析光在晶体中的传播特性。

\subsection{椭球方程}
\begin{equation}
    \frac{x^2}{n_x^2} + \frac{y^2}{n_y^2} + \frac{z^2}{n_z^2} = 1
\end{equation}
这是一个三维几何曲面,其主轴长度分别为 $2n_x, 2n_y, 2n_z$。

\subsection{几何分析法 (Method of Index Ellipsoid)}
给定波法线方向 $\vec{s}$(即平面波的传播方向 $\vec{k}$ 的方向),如何确定允许的偏振方向和折射率?
\begin{enumerate}
    \item 过椭球中心作一个**平面**,该平面垂直于波法线 $\vec{s}$。
    \item 该平面与折射率椭球的交线是一个**椭圆**。
    \item 该椭圆的**长轴**和**短轴**方向,即为允许的两个线性偏振态($\vec{D}$ 的振动方向)。
    \item 椭圆**半长轴**和**半短轴**的长度,即为对应的两个折射率 $n'$ 和 $n''$。
\end{enumerate}
\textbf{结论}:一般情况下,沿任意方向 $\vec{s}$ 传播的光波,会分裂为两束偏振方向互相垂直、相速度不同的线偏振光。这就是**双折射 (Birefringence)**。

\section{菲涅耳波法线方程 (Fresnel's Equation of Wave Normals)}

这是描述晶体光学的代数基础。设相速度为 $v_p = c/n$,波法线方向余弦为 $(s_x, s_y, s_z)$。
从麦克斯韦方程组出发,可推导出关于相速度 $v_p$ 的方程:
\begin{equation}
    \frac{s_x^2}{v_p^2 - v_x^2} + \frac{s_y^2}{v_p^2 - v_y^2} + \frac{s_z^2}{v_p^2 - v_z^2} = 0
\end{equation}
其中 $v_i = c/n_i$ 为主相速度。
这是一个关于 $v_p^2$ 的二次方程,对于给定的方向 $\vec{s}$,一般有两个正实根 $v_{p1}, v_{p2}$,对应两种传播模式。

\section{单轴晶体中的双折射 (Birefringence in Uniaxial Crystals)}

单轴晶体 ($n_x=n_y=n_o, n_z=n_e$) 是最常见的研究对象。

\subsection{o 光与 e 光}
利用折射率椭球分析可知,对于单轴晶体:
\begin{enumerate}
    \item **寻常光 (Ordinary ray, o光)**:
    \begin{itemize}
        \item 折射率恒为 $n_o$。
        \item 相速度 $v_o = c/n_o$ 与方向无关。
        \item 振动方向 ($\vec{D}_o$) 垂直于光轴和波法线构成的平面。
    \end{itemize}
    \item **非寻常光 (Extraordinary ray, e光)**:
    \begin{itemize}
        \item 折射率 $n_e(\theta)$ 随波法线与光轴夹角 $\theta$ 变化。
        \item \textbf{公式}:
        \begin{equation}
            \frac{1}{n_e^2(\theta)} = \frac{\cos^2\theta}{n_o^2} + \frac{\sin^2\theta}{n_e^2}
        \end{equation}
        \item 振动方向 ($\vec{D}_e$) 在光轴与波法线构成的平面内。
    \end{itemize}
\end{enumerate}

\subsection{正晶体与负晶体}
\begin{itemize}
    \item **正晶体 (Positive)**:$n_e > n_o$(即 $v_e < v_o$)。例如:石英 (Quartz)。
    \item **负晶体 (Negative)**:$n_e < n_o$(即 $v_e > v_o$)。例如:方解石 (Calcite)。
\end{itemize}

\subsection{惠更斯作图法 (Huygens' Construction)}
钟锡华教材中经典的作图法,用于确定折射光线的方向。
\begin{itemize}
    \item **原理**:界面上的每一点作为次波源,发射子波。
    \item **波面形状**:
        \begin{itemize}
            \item o 光波面:球面。
            \item e 光波面:旋转椭球面。
        \end{itemize}
    \item **作图步骤**:
        1. 以入射点为中心画出 o 光球面和 e 光椭球面。
        2. 根据折射定律的时空匹配条件(公切线/包络面),确定新的波前。
        3. 连接波源中心与切点的连线即为**光线方向 (Ray Direction)**。
\end{itemize}
\textbf{注意}:在晶体中,**光线方向**(能量传播方向,坡印廷矢量)与**波法线方向**(相位传播方向,$\vec{k}$)通常\textbf{不重合}(o 光除外)。

\section{晶体光学器件 (Crystal Optical Devices)}

\subsection{偏振棱镜 (Polarizing Prisms)}
利用双折射性质和全反射原理,将 o 光和 e 光分离开,获得线偏振光。
\begin{itemize}
    \item **尼科耳棱镜 (Nicol Prism)**:早期使用,利用加拿大树胶粘合方解石。o 光在胶层发生全反射被吸收,e 光透过。
    \item **格兰-泰勒棱镜 (Glan-Taylor Prism)**:空气隙结构,透射率高,消光比极高(现代常用)。
    \item **沃拉斯顿棱镜 (Wollaston Prism)**:由两块光轴互相垂直的直角棱镜胶合而成。o 光和 e 光折射率突变方向相反,导致两束光分角射出(双光束偏振分束器)。
\end{itemize}

\subsection{波片 (Wave Plates)}
利用晶体的双折射产生两个正交分量之间的相位差(延迟)。通常光轴平行于表面切割。
设波片厚度为 $d$,工作波长 $\lambda$。
\mydef{相位延迟量 $\delta$}
\begin{equation}
    \delta = \frac{2\pi}{\lambda} |n_o - n_e| d
\end{equation}

\begin{enumerate}
    \item **全波片 ($\lambda$片)**:$\delta = 2m\pi$。偏振态不变。
    \item **半波片 ($\lambda/2$片)**:$\delta = (2m+1)\pi$。
    \begin{itemize}
        \item 作用:线偏振光入射,出射仍为线偏振光,但振动面转过 $2\alpha$ 角($\alpha$ 为入射振动面与快轴夹角)。
    \end{itemize}
    \item **四分之一波片 ($\lambda/4$片)**:$\delta = (2m+1)\pi/2$。
    \begin{itemize}
        \item 作用:线偏振光($\alpha=45^\circ$)入射 $\to$ 圆偏振光;圆偏振光入射 $\to$ 线偏振光。
    \end{itemize}
\end{enumerate}

\section{偏振光的干涉 (Interference of Polarized Light)}

\subsection{干涉装置}
起偏器 $P_1$ $\to$ 晶体波片 $C$ $\to$ 检偏器 $P_2$。
由于晶体的双折射,o 光和 e 光在出射后获得相位差 $\delta$,在通过检偏器 $P_2$ 投影到同一方向后发生干涉。

\subsection{透射光强公式}
设 $P_1, P_2$ 透光轴夹角为 $\beta$,波片光轴与 $P_1$ 夹角为 $\alpha$。
\begin{equation}
    I = I_0 [ \cos^2 \beta - \sin 2\alpha \sin 2(\beta-\alpha) \sin^2(\delta/2) ]
\end{equation}
\textbf{特例}:
\begin{itemize}
    \item **正交偏振 ($P_1 \perp P_2, \beta = 90^\circ$)**:
    $$ I_{\perp} = I_0 \sin^2 2\alpha \sin^2(\delta/2) $$
    若 $\delta = 2m\pi$(全波片),$I=0$(黑)。若 $\delta = (2m+1)\pi$(半波片),$I$ 最大。
    \item **平行偏振 ($P_1 \parallel P_2, \beta = 0^\circ$)**:
    $$ I_{\parallel} = I_0 (1 - \sin^2 2\alpha \sin^2(\delta/2)) $$
    与正交情况互补:$I_{\perp} + I_{\parallel} = I_0$(色互补)。
\end{itemize}

\subsection{会聚光干涉 (Conoscopic Interference)}
当用会聚光束照射晶体切片时,不同入射角对应不同的光程差 $\delta$,在焦平面上形成特定的干涉图样。
\begin{itemize}
    \item **单轴晶体垂直光轴切片**:在正交偏振场下,呈现一系列同心彩色圆环(等色线),并被一个黑色的**马耳他十字 (Maltese Cross)**(等倾线)所分割。
\end{itemize}
