\section{晶体光学}

\subsection{双折射现象}
当光束射入各向异性晶体时,一般会分裂成两束折射光,这种现象称为双折射。
\begin{itemize}
    \item \textbf{o光 (寻常光)}:遵守折射定律,光速 $v_o$ 和折射率 $n_o$ 在各个方向上是一个常数,波面为球面。
    \item \textbf{e光 (非寻常光)}:一般不遵守折射定律,光速 $v_e$ 和折射率 $n_e$ 随方向而改变,波面为旋转椭球面。
\end{itemize}

\subsubsection{折射率椭球}
为了描述晶体中折射率随光振动方向的变化,引入折射率椭球方程:
$$ \frac{x^2}{n_x^2} + \frac{y^2}{n_y^2} + \frac{z^2}{n_z^2} = 1 $$
对于单轴晶体(如方解石),以光轴为旋转轴,$n_x=n_y=n_o, n_z=n_e$。
\begin{itemize}
    \item \textbf{主折射率}:$n_o$ 和 $n_e$。
    \item \textbf{任意方向折射率}:设波法线与光轴夹角为 $\theta$,e光的折射率 $n'(\theta)$ 满足:
    $$ \frac{1}{n'^2} = \frac{\cos^2\theta}{n_o^2} + \frac{\sin^2\theta}{n_e^2} $$
\end{itemize}

\subsection{晶体偏振器件}
利用晶体的双折射性质,可以制成各种功能的偏振器件。

\subsubsection{晶体棱镜}
主要用于产生线偏振光或分离两束正交偏振光。
\insertfigure{picture/2026-01-04-15-50-30.png}{0}{fig:0}
\begin{itemize}
    \item \textbf{尼科耳棱镜 (Nicol)}:利用全反射原理将 o 光反射吸收,只透射 e 光,获得线偏振光。
    \item \textbf{洛雄棱镜 (Rochon)} 和 \textbf{沃拉斯顿棱镜 (Wollaston)}:用于分离 o 光和 e 光,使它们具有不同的传播方向(分束角)。
\end{itemize}

\subsubsection{波晶片}
波晶片是厚度精确的晶体薄片,光轴平行于表面。o 光和 e 光通过波片后会产生光程差和相位差。
\insertfigure{picture/2026-01-04-15-51-17.png}{0}{fig:0}
\begin{itemize}
    \item \textbf{相位差公式}:
    $$ \delta = \frac{2\pi}{\lambda_0} (n_e - n_o) d $$
    其中 $d$ 为波片厚度。
    \item \textbf{四分之一波片 ($\lambda/4$片)}:$\delta = \pi/2$。
        \begin{itemize}
            \item 作用:将线偏振光转换为圆偏振光或椭圆偏振光,反之亦然。
        \end{itemize}
    \item \textbf{半波片 ($\lambda/2$片)}:$\delta = \pi$。
        \begin{itemize}
            \item 作用:使线偏振光的振动面转动 $2\theta$ 角($\theta$ 为入射偏振面与波片快轴的夹角)。
        \end{itemize}
    \item \textbf{补偿器}:如巴比涅补偿器、索雷尔补偿器。可以提供线性可变的相位差,用于精确测量相位。
\end{itemize}
\insertfigure{picture/2026-01-04-15-53-07.png}{0}{fig:0}
\subsection{偏振光的干涉}
典型的偏振光干涉装置由“起偏器 $P_1$ — 波晶片 — 检偏器 $P_2$”组成。

\textbf{干涉光强公式}:
设 $P_1$ 与 $P_2$ 的透光轴夹角为 $\gamma$(图中为 $\beta$ 关联项),波片引入相位差为 $\delta$。
\insertfigure{picture/2026-01-04-15-49-54.png}{0}{fig:0}
\begin{itemize}
    \item 当 $P_1 \parallel P_2$ (平行尼科耳) 时:
    $$ I_{\parallel} = I_0 \cos^2 \frac{\delta}{2} $$
    这时呈现互补色。
    \item 当 $P_1 \perp P_2$ (正交尼科耳) 时:
    $$ I_{\perp} = I_0 \sin^2 2\alpha \sin^2 \frac{\delta}{2} $$
    其中 $\alpha$ 是入射偏振光与波片光轴的夹角。
    \begin{itemize}
        \item 若 $\alpha = 45^\circ$(对角位),则 $I = I_0 \sin^2 \frac{\delta}{2}$,干涉最明显。
        \item 若 $\alpha = 0^\circ$ 或 $90^\circ$(消光位),则 $I=0$,无干涉。
    \end{itemize}
\end{itemize}

\subsection{旋光性与电光效应}
\subsubsection{旋光性}
某些物质(如石英、糖溶液)能使线偏振光的振动面发生旋转。
\insertfigure{picture/2026-01-04-15-53-39.png}{0}{fig:0}
\begin{itemize}
    \item \textbf{菲涅耳唯象解释}:线偏振光可分解为一束左旋圆偏振光和一束右旋圆偏振光。在旋光介质中,左旋和右旋光的折射率不同($n_L \neq n_R$),传播速度不同,导致出射时合成矢量的方向发生偏转。
    \item \textbf{旋光角公式}:$\psi = \alpha d$ ($\alpha$ 为旋光率)。
\end{itemize}

\subsubsection{电光效应}
外加电场改变介质折射率的现象。
\begin{itemize}
    \item \textbf{克尔效应 (Kerr Effect)}:各向同性介质在强电场下变为双折射介质,诱导的双折射率差 $\Delta n$ 与电场平方成正比:
    $$ \Delta n \propto \lambda E^2 $$
    \item \textbf{泡克耳斯效应 (Pockels Effect)}:线性电光效应,折射率变化与电场 $E$ 的一次方成正比。
\end{itemize}
