\section{介质界面光学}

当光波从一种介质(折射率 $n_1$)射向另一种介质(折射率 $n_2$)时,在界面上发生的反射和折射现象由电磁场边界条件决定。

\subsection{菲涅耳公式}
将入射光的电矢量 $\vec{E}$ 分解为两个互相垂直的分量:
\begin{itemize}
    \item \textbf{s 分量}(垂直分量,$\perp$):电矢量垂直于入射面。
    \item \textbf{p 分量}(平行分量,$\parallel$):电矢量平行于入射面。
\end{itemize}
设入射角为 $i_1$,折射角为 $i_2$(满足斯涅耳定律 $n_1 \sin i_1 = n_2 \sin i_2$)。

\mydef{反射系数}
\begin{equation}
    r_s = \frac{E'_{s0}}{E_{s0}} = - \frac{\sin(i_1 - i_2)}{\sin(i_1 + i_2)}, \quad
    r_p = \frac{E'_{p0}}{E_{p0}} = \frac{\tan(i_1 - i_2)}{\tan(i_1 + i_2)}
\end{equation}

\mydef{透射系数}
\begin{equation}
    t_s = \frac{E''_{s0}}{E_{s0}} = \frac{2 \sin i_2 \cos i_1}{\sin(i_1 + i_2)}, \quad
    t_p = \frac{E''_{p0}}{E_{p0}} = \frac{2 \sin i_2 \cos i_1}{\sin(i_1 + i_2) \cos(i_1 - i_2)}
\end{equation}

\textbf{注意}:
\begin{itemize}
    \item 符号法则:通常取 $r_s$ 的负号表示反射光 s 分量在光疏射向光密时存在相位突变 $\pi$(半波损失)。
    \item 当 $i_1 \to 0$(正入射)时:$r_s = r_p = \frac{n_1 - n_2}{n_1 + n_2}$(符号差异取决于坐标系定义,本质是一致的)。
\end{itemize}

\subsection{反射率与透射率}
光强 $I \propto n |\vec{E}|^2$。能量守恒要求穿过界面的能流连续。

\mydef{反射率 $R$}
反射光强与入射光强之比:
\begin{equation}
    R = |r|^2 \quad \Rightarrow \quad R_s = r_s^2, \quad R_p = r_p^2
\end{equation}
自然光入射时的总反射率:$R = \frac{1}{2}(R_s + R_p)$。

\mydef{透射率 $T$}
透射光强与入射光强之比(注意需考虑光束截面积变化因子 $\frac{\cos i_2}{\cos i_1}$):
\begin{equation}
    T = \frac{n_2 \cos i_2}{n_1 \cos i_1} |t|^2
\end{equation}
\textbf{能量守恒}:$R + T = 1$ 对 s 和 p 分量分别成立。

\subsection{布儒斯特角}
\mydef{定义}
当 $r_p = 0$ 时,反射光中完全没有 p 分量,只有 s 分量,此时反射光为\textbf{完全线偏振光}。对应的入射角称为布儒斯特角 $i_B$。
\begin{equation}
    \tan i_B = \frac{n_2}{n_1}
\end{equation}
\textbf{物理图景}:此时反射光线与折射光线互相垂直 ($i_1 + i_2 = 90^\circ$)。折射介质中的偶极子振动方向平行于反射方向,因此不向该方向辐射能量(即无 p 分量反射)。



\subsection{全反射现象}
当光从光密介质 ($n_1$) 射向光疏介质 ($n_2$),且入射角大于临界角 $i_c$ 时,发生全反射。
\begin{equation}
    \sin i_c = \frac{n_2}{n_1}
\end{equation}
此时 $R=1$,能量全部反射回第一介质。


\subsection{倏逝波}
\mydef{物理本质}
在全反射条件下,第二介质(光疏介质)中虽然没有传输能量的折射波,但存在一种非均匀波,称为\textbf{倏逝波}(或隐失波)。
其电场分布特征为:
\begin{itemize}
    \item \textbf{沿界面传播}:相位因子 $e^{\ii (k_x x - \omega t)}$,沿界面 $x$ 方向传播。
    \item \textbf{垂直界面衰减}:振幅因子 $e^{-z/d_p}$,沿垂直方向 $z$ 指数衰减。
\end{itemize}
穿透深度 (Penetration Depth) $d_p$:
\begin{equation}
    d_p = \frac{\lambda_1}{2\pi \sqrt{\sin^2 i_1 - (n_2/n_1)^2}}
\end{equation}
通常 $d_p$ 仅为波长量级。倏逝波不向 $z$ 方向传输平均能流,能量被束缚在界面附近。

