\section{介质界面光学 (Optics at Interfaces)}

当光波从一种介质(折射率 $n_1$)射向另一种介质(折射率 $n_2$)时,在界面上发生的反射和折射现象由电磁场边界条件决定。

\subsection{菲涅耳公式 (Fresnel Formulas)}
将入射光的电矢量 $\vec{E}$ 分解为两个互相垂直的分量:
\begin{itemize}
    \item \textbf{s 分量}(垂直分量,$\perp$):电矢量垂直于入射面。
    \item \textbf{p 分量}(平行分量,$\parallel$):电矢量平行于入射面。
\end{itemize}
设入射角为 $i_1$,折射角为 $i_2$(满足斯涅耳定律 $n_1 \sin i_1 = n_2 \sin i_2$)。

\mydef{反射系数 (Reflection Coefficients)}
\begin{equation}
    r_s = \frac{E'_{s0}}{E_{s0}} = - \frac{\sin(i_1 - i_2)}{\sin(i_1 + i_2)}, \quad
    r_p = \frac{E'_{p0}}{E_{p0}} = \frac{\tan(i_1 - i_2)}{\tan(i_1 + i_2)}
\end{equation}

\mydef{透射系数 (Transmission Coefficients)}
\begin{equation}
    t_s = \frac{E''_{s0}}{E_{s0}} = \frac{2 \sin i_2 \cos i_1}{\sin(i_1 + i_2)}, \quad
    t_p = \frac{E''_{p0}}{E_{p0}} = \frac{2 \sin i_2 \cos i_1}{\sin(i_1 + i_2) \cos(i_1 - i_2)}
\end{equation}

\textbf{注意}:
\begin{itemize}
    \item 符号法则:通常取 $r_s$ 的负号表示反射光 s 分量在光疏射向光密时存在相位突变 $\pi$(半波损失)。
    \item 当 $i_1 \to 0$(正入射)时:$r_s = r_p = \frac{n_1 - n_2}{n_1 + n_2}$(符号差异取决于坐标系定义,本质是一致的)。
\end{itemize}

\subsection{反射率与透射率 (Reflectance and Transmittance)}
光强 $I \propto n |\vec{E}|^2$。能量守恒要求穿过界面的能流连续。

\mydef{反射率 $R$}
反射光强与入射光强之比:
\begin{equation}
    R = |r|^2 \quad \Rightarrow \quad R_s = r_s^2, \quad R_p = r_p^2
\end{equation}
自然光入射时的总反射率:$R = \frac{1}{2}(R_s + R_p)$。

\mydef{透射率 $T$}
透射光强与入射光强之比(注意需考虑光束截面积变化因子 $\frac{\cos i_2}{\cos i_1}$):
\begin{equation}
    T = \frac{n_2 \cos i_2}{n_1 \cos i_1} |t|^2
\end{equation}
\textbf{能量守恒}:$R + T = 1$ 对 s 和 p 分量分别成立。

\subsection{布儒斯特角 (Brewster's Angle)}
\mydef{定义}
当 $r_p = 0$ 时,反射光中完全没有 p 分量,只有 s 分量,此时反射光为\textbf{完全线偏振光}。对应的入射角称为布儒斯特角 $i_B$。
\begin{equation}
    \tan i_B = \frac{n_2}{n_1}
\end{equation}
\textbf{物理图景}:此时反射光线与折射光线互相垂直 ($i_1 + i_2 = 90^\circ$)。折射介质中的偶极子振动方向平行于反射方向,因此不向该方向辐射能量(即无 p 分量反射)。



\subsection{全反射现象}
当光从光密介质 ($n_1$) 射向光疏介质 ($n_2$),且入射角大于临界角 $i_c$ 时,发生全反射。
\begin{equation}
    \sin i_c = \frac{n_2}{n_1}
\end{equation}
此时 $R=1$,能量全部反射回第一介质。

\subsection{相位变化与古斯-汉欣位移}
\mydef{相位变化}
在全反射区,反射系数 $r_s$ 和 $r_p$ 变为复数,模为 1,但存在相位因子 $e^{\ii \delta}$。这导致反射光产生相位延迟,且 $\delta_s \neq \delta_p$。
应用:\textbf{菲涅耳菱体}利用两次全反射将线偏振光转变为圆偏振光。

\mydef{古斯-汉欣位移 (Goos-Hänchen Shift)}
实验发现,全反射时实际反射光束相对于几何光学的反射点在界面上有一个微小的侧向位移 $D$(约波长量级)。这表明光波实际上穿透到了第二介质中一段距离后再返回。

\subsection{倏逝波 (Evanescent Wave)}
\mydef{物理本质}
在全反射条件下,第二介质(光疏介质)中虽然没有传输能量的折射波,但存在一种非均匀波,称为\textbf{倏逝波}(或隐失波)。
其电场分布特征为:
\begin{itemize}
    \item \textbf{沿界面传播}:相位因子 $e^{\ii (k_x x - \omega t)}$,沿界面 $x$ 方向传播。
    \item \textbf{垂直界面衰减}:振幅因子 $e^{-z/d_p}$,沿垂直方向 $z$ 指数衰减。
\end{itemize}
穿透深度 (Penetration Depth) $d_p$:
\begin{equation}
    d_p = \frac{\lambda_1}{2\pi \sqrt{\sin^2 i_1 - (n_2/n_1)^2}}
\end{equation}
通常 $d_p$ 仅为波长量级。倏逝波不向 $z$ 方向传输平均能流,能量被束缚在界面附近。

\section{近场光学显微镜 (Near-field Optical Microscopy)}

\begin{spacing}{1.5}

\subsection{远场衍射极限 (Far-field Diffraction Limit)}
传统光学显微镜接收的是从物体传播到远处的\textbf{传播波 (Propagating Wave)}。
根据瑞利判据或阿贝成像原理,其分辨率受限于波长:
\begin{equation}
    \delta \approx \frac{\lambda}{2 \text{NA}} \approx \frac{\lambda}{2}
\end{equation}
物体的高频空间信息(对应微小细节)往往携带在倏逝波中。由于倏逝波在垂直传播方向上指数衰减,无法到达远场,导致细节信息丢失。

\subsection{近场光学原理 (SNOM/NSOM)}
\mydef{核心思想}
利用极细的探针(孔径 $a \ll \lambda$)伸入到距离样品表面极近的区域(近场区,$z < \lambda$),通过探测或干扰界面上的\textbf{倏逝波},将其转化为可传输的传播波,从而被探测器接收。

\subsection{扫描近场光学显微镜 (SNOM)}
\textbf{结构特点}:
\begin{itemize}
    \item \textbf{探针}:通常为拉制的锥形光纤,尖端镀金属膜,留有几十纳米的通光孔。
    \item \textbf{距离控制}:利用剪切力 (Shear Force) 反馈系统,保持探针与样品间距恒定(约几纳米)。
    \item \textbf{扫描方式}:逐点扫描成像。
\end{itemize}

\textbf{分辨率}:
不再受光波长 $\lambda$ 限制,而是取决于\textbf{探针孔径大小}(可达 $\SI{20}{nm} \sim \SI{50}{nm}$)和\textbf{探针与样品的距离}。

\begin{figure}[htbp]
    \centering
    [图片: 倏逝波探测原理与SNOM探针示意图]
    \caption{近场光学探测原理:探针将束缚在表面的倏逝波转化为传播波}
    \label{fig:snom}
\end{figure}
\end{spacing}