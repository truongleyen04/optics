\section{全息术的基本原理}
\insertfigure{picture/2026-01-04-15-34-56.png}{0}{fig:0}
如图所示,一激光束经显微镜头、分束器、反射镜等元件,被扩束、分解和变向,而生成两束宽孔径的相干光束。其中,一束直接投射到记录介质(乳胶干板)$H$ 面上,称其为参考光波 $\tilde{R}$;另一束投射到物体上,经物体上各点漫反射而形成一物光波 $\tilde{O}$,传播到记录介质 $H$ 面上。于是,记录介质平面 $H$ 上存在 $\tilde{O}$ 光与 $\tilde{R}$ 光的干涉场。

$$ \tilde{U}_H(x,y) = \tilde{O}(x,y) + \tilde{R}(x,y) \eqno{(7.1)} $$

其中,参考光 $\tilde{R}$ 通常被调节为平面波或球面波,其传播方向或倾角,也可以人为地控制;而物光波 $\tilde{O}$,从微观上看,是物体上各点源发出的大量次波的相干叠加,

$$ \tilde{O}(x,y) = \sum_n \tilde{u}_n(x,y) = A_O(x,y) \cdot e^{i\varphi_O(x,y)} \eqno{(7.2)} $$

其中,每一次波 $\tilde{u}_n(x,y)$,决定于相应物点的亮度和位置,$H$ 面上存在的这 $\tilde{O}$ 光波前,正是这些次波的自相干场,其振幅分布 $A_O(x,y)$,尤其是相位分布 $\varphi_O(x,y)$ 反映了物体的三维形貌或形象,虽然记录的是二维光场信息。

当然,记录介质感受的依然是光强分布,这与普通照相的胶片并无区别。那么,$\tilde{O}$ 波与 $\tilde{R}$ 波叠加的干涉强度分布为

$$
\begin{aligned}
I_H(x,y) &= \tilde{U}_H \cdot \tilde{U}_H^* = (\tilde{O} + \tilde{R}) \cdot (\tilde{O}^* + \tilde{R}^*) \\
&= |\tilde{O}|^2 + |\tilde{R}|^2 + \tilde{R}^* \cdot \tilde{O} + \tilde{R} \cdot \tilde{O}^* \\
&= A_O^2(x,y) + A_R^2(x,y) + A_R e^{-i\varphi_R} \cdot \tilde{O} + A_R e^{i\varphi_R} \cdot \tilde{O}^* \end{aligned}
\eqno{(7.3)}
$$

这里,我们将人为安排的参考光波 $\tilde{R}$ 的波前函数表示为

$$ \tilde{R}(x,y) = A_R(x,y) \cdot e^{i\varphi_R(x,y)} \eqno{(7.4)} $$

这光强分布 $I_H$ 要被记录或存储下来,还必须经化学溶液的处理,即所谓的显影和定影,简言之“冲洗”,且要求这次冲洗满足线性条件——冲洗后的这张底片,其透过率函数 $\tilde{t}_H$ 与干涉强度函数 $I_H$ 之间是线性关系,

$$
\begin{aligned}
\tilde{t}_H(x,y) &= t_0 + \beta I_H(x,y) \\
&= t_0 + \beta (A_O^2 + A_R^2) + \beta \tilde{R}^* \cdot \tilde{O} + \beta \tilde{R} \cdot \tilde{O}^*
\end{aligned}
\eqno{(7.5)}
$$

这里,$t_0, \beta$ 是常数。于是,便制成了一张全息图。

\subsection{全息图的衍射场 —— 相因子分析法的运用}
\insertfigure{picture/2026-01-04-15-35-29.png}{0}{fig:0}
用一准单色光波 $\tilde{R}'$ 照射一张全息图,如图所示。那么,这全息图作为一个衍射屏,在 $\tilde{R}'$ 波照射下,将产生一复杂的衍射场,其波前函数为

$$
\begin{aligned}
\tilde{U}'_H(x,y) &= \tilde{t}_H \cdot \tilde{R}' \\
&= (t_0 + \beta A_R^2 + \beta A_O^2) \cdot \tilde{R}' + \beta \tilde{R}' \tilde{R}^* \cdot \tilde{O} + \beta \tilde{R}' \tilde{R} \cdot \tilde{O}^* \\
&= \tilde{T}_1 \cdot \tilde{R}' + \tilde{T}_2 \cdot \tilde{O} + \tilde{T}_3 \cdot \tilde{O}^*
\end{aligned}
\eqno{(7.6)}
$$

其中,照射光波 $\tilde{R}'$ 的波前函数可表示为

$$ \tilde{R}'(x,y) = A'_R(x,y) e^{i\varphi'_R(x,y)} \eqno{(7.7)} $$

凭借相因子分析法,在不同记录或照射条件下,可以逐项解析那三个操作系数的变换作用。具体说明如下:

\begin{itemize}
    \item[(1)] \textbf{变换因子 $\tilde{T}_1$},按 (7.6) 式,
    $$ \tilde{T}_1 = (t_0 + \beta A_R^2 + \beta A_O^2) \eqno{(7.8)} $$
    一般情况下,参考波 $\tilde{R}$ 是一列平面波或傍轴球面波,故其振幅分布 $A_R \approx$ 常数,而原物光波的振幅分布 $A_O(x,y)$ 虽复杂,但可近似考虑 $A_O \approx$ 常数。这样,变换因子 $\tilde{T}_1 \approx$ 常数,$\tilde{T}_1 \cdot \tilde{R}'$ 就表示了照射光波 $\tilde{R}'$ 的直接透射波,也就是全息图的 0 级衍射波。

    \item[(2)] \textbf{变换因子 $\tilde{T}_2$ 和 $\tilde{T}_3$},按 (7.6) 式,
    $$ \tilde{T}_2 = \beta \tilde{R}' \tilde{R}^* = \beta A'_R A_R e^{i(\varphi'_R - \varphi_R)} \eqno{(7.9)} $$
    $$ \tilde{T}_3 = \beta \tilde{R}' \tilde{R} = \beta A'_R A_R e^{i(\varphi'_R + \varphi_R)} \eqno{(7.10)} $$
\end{itemize}

\subsubsection{几种典型情况分析}

\textbf{典型情况之一:} $\tilde{R}'$ 波与 $\tilde{R}$ 波系同平面波,且正入射。这时,可设 $\varphi'_R = \varphi_R = 0$,于是 $\tilde{T}_2 = \tilde{T}_3 = \beta A'_R A_R =$ 常数,这就表明,

$$ \tilde{T}_2 \cdot \tilde{O} = \beta A'_R A_R \tilde{O} \quad \text{—— 物光波前的再现} $$
$$ \tilde{T}_3 \cdot \tilde{O}^* = \beta A'_R A_R \tilde{O}^* \quad \text{—— 物光共轭波前的伴生} $$

前者为 +1 级衍射波,是发散的,生成一虚像;后者为 -1 级衍射波,是会聚的,生成一实像。而且,在目前条件下,两者镜像对称,与原物尺寸亦相等。我们称这种情况下的这一对孪生像为“原生像”。

\textbf{典型情况之二:} $\tilde{R}'$ 波与 $\tilde{R}$ 波系同球面波,且斜入射。这时,相位分布函数 $\varphi'_R = \varphi_R =$ 线性相因子,于是
$$ \tilde{T}_2 = \beta A'_R A_R \approx \text{常数} $$
$$ \tilde{T}_3 = \beta A'_R A_R e^{i2\varphi_R} \quad \text{—— 等效棱镜} $$
显然,$\tilde{T}_2 \cdot \tilde{O}$ 项表示了原物光波前的真实再现,而 $\tilde{T}_3 \cdot \tilde{O}^*$ 项表明孪生的共轭波 $\tilde{O}^*$ 受到一等效棱镜的作用,发生了偏转。

\textbf{典型情况之三:} $\tilde{R}'$ 波与 $\tilde{R}$ 波系同球面波。这时,相位分布函数 $\varphi'_R = \varphi_R =$ 二次相因子,于是
$$ \tilde{T}_2 = \beta A'_R A_R \approx \text{常数} $$
$$ \tilde{T}_3 = \beta A'_R A_R e^{i2\varphi_R} \quad \text{—— 等效透镜} $$
故 $\tilde{T}_2 \cdot \tilde{O}$ 项表示了原物光波前的真实再现;而 $\tilde{T}_3 \cdot \tilde{O}^*$ 项表明孪生共轭波 $\tilde{O}^*$ 受到一等效透镜的作用,发生了移位、缩放和偏转。

\textbf{典型情况之四:} $\tilde{R}'$ 波与 $\tilde{R}$ 波互为一对共轭波。这时,波前函数 $\tilde{R}' = \tilde{R}^*$。例如,记录时参考波 $\tilde{R}$ 是一自上而下斜射的发散球面波束,则照射光 $\tilde{R}'$ 是一自下而上斜射的会聚球面波束。于是,相位函数 $\varphi'_R = -\varphi_R$,故 $\varphi'_R - \varphi_R = -2\varphi_R$,$\varphi'_R + \varphi_R = 0$。
此时 $\tilde{T}_3 \cdot \tilde{O}^*$ 项倒是一个真实的原物孪生像(实像)。

\subsubsection{波长变换}
照射光波 $\tilde{R}'$ 与参考波 $\tilde{R}$ 的波长可以不同,$\lambda' \neq \lambda$。例如,X 光全息图,可以用可见光照射而再现物光波前。其结果是再现物与原物相比,在几何上有一缩小或放大,其放大率正比于波长之比值:

$$ V \propto \frac{\lambda'}{\lambda} \eqno{(7.11)} $$

这为缩放图像提供了一新的技术途径。

