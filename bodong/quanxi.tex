\section{全息术的基本原理 (Basic Principles)}

全息术(Holography)由丹尼斯·伽博 (Dennis Gabor) 于 1948 年提出。与普通照相仅记录光强(振幅平方)不同,全息术利用**干涉**原理记录了物体光波的**振幅**和**相位**(全部信息),并利用**衍射**原理再现出物体光波。

\subsection{波前记录 (Recording)}
设物体发出的光波(物波)复振幅为 $O(x,y)$,参考光波(参考波)复振幅为 $R(x,y)$。两者在记录介质(全息底片)上叠加发生干涉。
光强分布为:
\begin{equation}
    I(x,y) = |O + R|^2 = (O+R)(O^*+R^*) = |O|^2 + |R|^2 + O R^* + O^* R
\end{equation}
\begin{itemize}
    \item $|O|^2$:物波自身强度(通常很弱且均匀)。
    \item $|R|^2$:参考波强度(均匀背景)。
    \item $O R^*$ 和 $O^* R$:**干涉项**,包含了物波 $O$ 的振幅和相位信息(调制在载波 $R$ 上)。
\end{itemize}
显影定影后,底片的振幅透射率 $t(x,y)$ 与曝光量 $E = I \cdot \tau$ 成线性关系(线性记录条件):
\begin{equation}
    t(x,y) = t_0 + \beta I(x,y) = t_0 + \beta (|O|^2 + |R|^2) + \beta O R^* + \beta O^* R
\end{equation}

\subsection{波前再现 (Reconstruction)}
用一束与参考波相同的光波 $R(x,y)$ 照射全息图。透射光场 $U(x,y)$ 为:
\begin{equation}
    U = R \cdot t = \underbrace{R(t_0 + \beta(|O|^2 + |R|^2))}_{U_0} + \underbrace{\beta |R|^2 O}_{U_1} + \underbrace{\beta R^2 O^*}_{U_2}
\end{equation}
\begin{enumerate}
    \item **0 级衍射波 ($U_0$)**:沿参考波方向传播的直透光,不含物体信息。
    \item **+1 级衍射波 ($U_1 \propto O$)**:正比于原始物波 $O$。人眼逆着光看去,能看到位于原物体位置的**虚像**(原始像)。这是全息术的核心——波前重建。
    \item **-1 级衍射波 ($U_2 \propto O^*$)**:正比于物波的共轭 $O^*$。产生一个位于全息图另一侧的**实像**(共轭像)。
\end{enumerate}

\section{全息图的分类 (Classification)}

\subsection{伽博全息图 (同轴全息)}
伽博最初的实验中,物波 $O$ 和参考波 $R$ 沿同一轴线传播(同轴)。
\begin{itemize}
    \item **缺点**:再现时,直透光 $U_0$、原始像 $U_1$ 和共轭像 $U_2$ 全部重叠在一起,互相干扰。这被称为**“孪生像”问题 (Twin Image Problem)**。
    \item **应用**:仅限于高透明度的物体(如电子显微镜下的微小粒子)。
\end{itemize}

\subsection{利思-乌帕特尼克斯全息图 (离轴全息)}
Leith 和 Upatnieks (1962) 引入激光,并提出**离轴全息 (Off-axis Holography)**。
\begin{itemize}
    \item **原理**:引入一个带有倾斜角度的平面参考波 $R = A_r e^{\ii k x \sin \theta}$(线性相因子)。
    \item **频谱分离**:在频域中,线性相因子将干涉项的频谱搬移到了载频 $f_0 = \frac{\sin \theta}{\lambda}$ 附近。
    \item **结果**:再现时,0 级、+1 级和 -1 级衍射波在空间上**角分离**。观察者可以清楚地看到分离的虚像,互不干扰。
\end{itemize}

\subsection{平面全息与体积全息}
根据记录介质厚度 $d$ 与干涉条纹间距 $\Lambda$ 的关系分类。
\begin{itemize}
    \item **平面全息图 (Thin Hologram)**:$Q = \frac{2\pi \lambda d}{n \Lambda^2} < 1$。表现为二维光栅,存在多级衍射,且对波长和角度选择性差。
    \item **体积全息图 (Volume Hologram)**:$Q > 10$。记录介质较厚,干涉条纹以“折射率层”的形式存在于介质内部(三维光栅)。
    \item **布拉格条件 (Bragg Condition)**:体积全息图的再现必须严格满足布拉格衍射条件:
    \begin{equation}
        2 d \sin \theta = k \lambda
    \end{equation}
    这意味着体积全息图具有极高的**角度选择性**和**波长选择性**。
\end{itemize}

\subsection{反射全息图 (丹尼苏克全息)}
Denisyuk (1962) 提出。参考光和物光分别从记录介质的两侧入射,相向传播。
\begin{itemize}
    \item **条纹结构**:干涉条纹面几乎平行于介质表面,层间距约为 $\lambda / 2$。
    \item **特点**:利用体积全息的波长选择性,可以**白光再现**(Lippmann 彩色摄影原理)。白光照射时,只有特定波长的光满足布拉格条件被反射,形成单色像。
\end{itemize}

\section{全息术的应用 (Applications)}

\subsection{全息干涉计量 (Holographic Interferometry)}
利用全息图记录的物波与实物波(或另一个时刻的物波)进行干涉。
\begin{itemize}
    \item **实时全息干涉**:全息图显影后复位,用原参考光照射。再现出的虚像波与当前物体的实时散射波干涉。可实时观察物体的微小形变。
    \item **双曝光全息干涉**:在同一张底片上,分别记录物体变形前后的两个状态。再现时,两个物波同时重建并干涉,条纹反映了物体两状态间的位移量。
    \item **时间平均全息**:用于振动物体(如扬声器纸盆)。曝光时间远大于振动周期,得到的图像上叠加了贝塞尔函数形式的条纹,亮纹对应振动节点。
\end{itemize}

\subsection{全息光学元件 (HOE)}
利用全息图的衍射特性制成的透镜、光栅、扫描盘等。优点是轻薄、可集成多种功能(如全息平视显示器 HUD)。

\subsection{全息数据存储}
利用体积全息的角度复用特性,在晶体同一点记录成百上千幅全息图,实现高密度、大容量存储。
