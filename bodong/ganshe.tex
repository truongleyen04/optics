\section{波动光学引论}

\begin{spacing}{1.5}

\subsection{光的电磁理论基础}
\mydef{麦克斯韦方程组与波动方程}
光是电磁波。在无源($\rho=0, \vec{J}=0$)、线性、各向同性、均匀介质中,电场 $\vec{E}$ 和磁场 $\vec{B}$ 均满足矢量波动方程:
\begin{equation}
    \nabla^2 \vec{E} - \mu\varepsilon \frac{\partial^2 \vec{E}}{\partial t^2} = 0
\end{equation}
相速度 $v = 1/\sqrt{\mu\varepsilon}$。真空中 $c \approx \SI{3e8}{m/s}$。
折射率 $n = c/v = \sqrt{\varepsilon_r \mu_r} \approx \sqrt{\varepsilon_r}$(大多数光学介质中 $\mu_r \approx 1$)。

\mydef{光强 }
光强定义为坡印廷矢量  $\vec{S} = \vec{E} \times \vec{H}$ 大小的物理平均值。在同一种介质中,光强正比于电场振幅的平方:
\begin{equation}
    I \propto \langle |\vec{E}|^2 \rangle
\end{equation}

\mydef{标量波近似}
当光波在均匀介质中传播,且不涉及边界上的偏振转换(如菲涅耳反射公式)或各向异性介质(如晶体双折射)时,可忽略电场的矢量性,用标量函数 $U(\vec{r}, t)$ 描述光场。

\subsection{定态光波与复振幅}
核心思想:\textbf{用复数运算简化波动分析}。
对于单色光(角频率 $\omega$),实数波函数 $U(\vec{r}, t) = A(\vec{r}) \cos[\omega t - \varphi(\vec{r})]$ 可写为:
\begin{equation}
    U(\vec{r}, t) = \operatorname{Re} [\tilde{U}(\vec{r}) e^{\ii \omega t}]
\end{equation}
其中 \textbf{复振幅} $\tilde{U}(\vec{r})$ 定义为:
\begin{equation}
    \tilde{U}(\vec{r}) = A(\vec{r}) e^{-\ii \varphi(\vec{r})}
\end{equation}
复振幅满足不含时的 \textbf{亥姆霍兹方程}:
\begin{equation}
    \nabla^2 \tilde{U} + k^2 \tilde{U} = 0, \quad k = \frac{n\omega}{c} = \frac{2\pi}{\lambda}
\end{equation}

\subsection{典型光波及其复振幅}
\begin{enumerate}
    \item \textbf{平面波}:传播方向为 $\vec{k}$。
    $$ \tilde{U}(\vec{r}) = A e^{-\ii \vec{k} \cdot \vec{r}} $$
    等相面方程:$\vec{k} \cdot \vec{r} = \text{C}$(平面)。
    
    \item \textbf{球面波}:从点源发散。
    $$ \tilde{U}(r) = \frac{A}{r} e^{-\ii kr} $$
    振幅 $A/r$ 随距离衰减,能量守恒($I \propto 1/r^2$)。

    \textbf{傍轴近似}:在距离 $z$ 远处,球面波因子 $e^{-\ii kr} \approx e^{-\ii k z} e^{-\ii k\frac{x^2+y^2}{2z}}$,这在衍射积分中非常重要。
\end{enumerate}

\subsection{波动传播的重要近似条件与模型}
\mydef{傍轴条件}
当光线传播方向与光轴(设为 $z$ 轴)的夹角 $\theta$ 很小($\theta \ll 1$ rad)时,满足傍轴条件。
\begin{itemize}
    \item \textbf{几何近似}:$\sin\theta \approx \tan\theta \approx \theta$,$\cos\theta \approx 1 - \theta^2/2$。
    \item \textbf{相位近似}:对于球面波传播距离 $r = \sqrt{z^2 + x^2 + y^2}$,泰勒展开保留至二次项:$r \approx z + \frac{x^2+y^2}{2z}$。该近似要求高阶相位项带来的误差远小于 $\pi$,即 $k \frac{(x^2+y^2)^2}{8z^3} \ll \pi$。
    \item \textbf{物理意义}:在傍轴条件下,球面波可近似视为一个带有\textbf{二次曲面相位因子}的平面波。
\end{itemize}

\mydef{远场条件}
通常指观察距离 $z$ 远大于波源或孔径尺寸 $D$ 的平方与波长的比值的情况。
\begin{itemize}
    \item \textbf{判据}:$z \gg \frac{D^2}{\lambda}$
    \item \textbf{物理意义}:在远场区域,波源各点发出的子波到达观察点时,其光程差主要由线性项决定,二次相位项 $\frac{k(x^2+y^2)}{2z}$ 趋于常数或可忽略。此时球面波波前在观察范围内非常平坦,局部可视为\textbf{平面波}。这也是\textbf{夫琅禾费衍射}成立的前提。
\end{itemize}



\end{spacing}

\subsection{光的干涉}

\subsubsection{波的叠加与相干条件}
\begin{spacing}{1.5}
\mydef{叠加原理}
$\tilde{U} = \tilde{U}_1 + \tilde{U}_2$。总光强 $I = \tilde{U}\tilde{U}^*$:
\begin{equation}
    I = I_1 + I_2 + 2\sqrt{I_1 I_2} \cos \delta
\end{equation}
干涉项 $J_{12} = 2\sqrt{I_1 I_2} \cos \delta$,其中 $\delta(\vec{r}) = \varphi_2 - \varphi_1$ 为相位差。

\mydef{相干条件}
为了观察到稳定的干涉图样($J_{12}$ 不随时间平均为零):
\begin{itemize}
    \item \textbf{频率相同} ($\omega_1 = \omega_2$):否则产生拍频,时间平均光强无干涉项。
    \item \textbf{振动方向平行}:$\vec{E}_1 \cdot \vec{E}_2 \neq 0$。
    \item \textbf{相位差恒定}:$\delta$ 不随时间随机跳变。
\end{itemize}

\mydef{条纹可见度}
描述干涉条纹的清晰程度:
\begin{equation}
    \gamma = \frac{I_{\max} - I_{\min}}{I_{\max} + I_{\min}} = \frac{2\sqrt{I_1 I_2}}{I_1 + I_2}
\end{equation}
当 $I_1 = I_2$ 且光源完全相干时,$\gamma=1$(最佳)。

\mydef{驻波}
两列振幅相同、相向传播的相干波叠加形成驻波。维纳实验证明了光波中的感光作用主要是由电场矢量引起的(波腹处感光最强)。
\end{spacing}

\subsection{分波前干涉}
物理上将点光源的波前分割为两部分,分别通过不同路径后汇聚。

\subsubsection{杨氏双缝干涉}
\textbf{装置参数}:缝间距 $d$,屏距 $D$,波长 $\lambda$。
\textbf{光程差} $\Delta \approx d \frac{x}{D}$。
\textbf{光强分布}:
\begin{equation}
    I(x) = I_0(1 + \cos k\frac{d}{D}x)
\end{equation}
\textbf{特征}:
\begin{itemize}
    \item 亮纹:$x_k = k \frac{D\lambda}{d}$;暗纹:$x_k = (k+0.5)\frac{D\lambda}{d}$。
    \item 条纹等间距 $\Delta x = \frac{D\lambda}{d}$。
\end{itemize}


\subsubsection{自然光的分解模型}
自然光是大量原子随机发光的集合,宏观上无固定偏振态,但在处理干涉时常采用以下模型:
\begin{itemize}
    \item \textbf{正交分解}:可将自然光等效为两束振动方向互相垂直($x$ 和 $y$)、振幅相等($A_x=A_y$)且\textbf{互不相干}的线偏振光的叠加。
    \item \textbf{光强关系}:$I_{\text{nat}} = I_x + I_y = \frac{1}{2}I_0 + \frac{1}{2}I_0$。
    \item \textbf{干涉表现}:在同一自然光源分成的两束小角度自然光之间的干涉,其衬比度为$\gamma = \frac{1}{2}(1+\cos\alpha)$
    \insertfigure{picture/2025-12-25-14-05-51.png}{0}{fig:0}
\end{itemize}

\subsection{单缝夫琅禾费衍射}

\subsubsection{理论模型与衍射公式}
设单缝宽度为 $a$,透镜焦距为 $f$,入射光波长为 $\lambda$。
在夫琅禾费近似下,衍射场是孔径函数的傅里叶变换。利用矢量图解法(振动矢量构成圆弧)或衍射积分法,可得衍射屏上的光强分布 $I(\theta)$:

\begin{equation}
    I(\theta) = I_0 \underbrace{\left( \frac{\sin \alpha}{\alpha} \right)^2}_{\text{单缝衍射因子}}
\end{equation}

其中关键参数定义为:
\begin{itemize}
    \item \textbf{半相位差 (宗量)}:$\alpha = \frac{\pi a}{\lambda} \sin \theta$
    \item \textbf{几何意义}:$\alpha$ 对应于边缘光线与中心光线到达场点的相位差。
    \item \textbf{矢量图解}:合成振幅 $A(\theta)$ 对应于弯曲成圆弧的振动矢量链的弦长,即 $A(\theta) = A_0 \frac{\sin \alpha}{\alpha}$。
\end{itemize}

\insertfigure{picture/2026-01-03-15-44-02.png}{0}{fig:0}

\subsubsection{图样主要特征}

\mydef{中央主极大 (零级斑)}
当 $\theta = 0$ 时,$\alpha = 0$,$\sinc(\alpha) \rightarrow 1$。
\textbf{特征}:光强最大 $I(0) = I_0$,集中了绝大部分能量。位置对应于几何光学像点。

\mydef{零点位置 (暗纹)}
当 $\sinc(\alpha) = 0$ 且 $\alpha \neq 0$ 时出现暗纹。
条件:$\alpha = k\pi \quad (k = \pm 1, \pm 2, \dots)$。
即\textbf{单缝衍射极小条件}:
\begin{equation}
    a \sin \theta = k \lambda
\end{equation}

\mydef{次极大}
位于两个暗纹之间。位置由 $\tan \alpha = \alpha$ 的超越方程决定。
数值特征如下表所示:
\begin{center}
\begin{tabular}{|c|c|c|c|}
\hline
 级次 & 位置 $x$ (近似) & $\sin \theta$ & 相对强度 $I/I_0$ \\
\hline
 1 & $\pm 1.43\pi$ & $\pm 1.43 \lambda/a$ & $4.7\%$ \\
 2 & $\pm 2.46\pi$ & $\pm 2.46 \lambda/a$ & $1.7\%$ \\
 3 & $\pm 3.47\pi$ & $\pm 3.47 \lambda/a$ & $0.8\%$ \\
\hline
\end{tabular}
\end{center}

\subsubsection{半角宽度与参数影响}

\mydef{半角宽度 $\Delta \theta_0$}
指零级衍射斑中心到第一极小值(暗纹)之间的角距离。
由 $a \sin \theta_1 = \lambda$ 且 $\theta_1$ 较小时:
\begin{equation}
    \Delta \theta_0 = \frac{\lambda}{a}
\end{equation}
它定量体现了衍射效应的强弱:$\Delta \theta_0$ 越大,衍射越发散。

\subsubsection{物理参数的演变规律}
\begin{itemize}
    \item \textbf{缝宽 $a$ 的影响}:
    \begin{itemize}
        \item[1.] \textbf{宽度}:$a$ 增大,半角宽度 $\Delta \theta_0$ 减小(反比),衍射斑压缩。
        \item[2.] \textbf{强度}:$I_0 \propto (ab)^2$。若 $a \rightarrow 2a$,则 $I_0 \rightarrow 4I_0$(能量更加集中且峰值更高)。
    \end{itemize}
    
    \item \textbf{波长 $\lambda$ 的影响}:
    \begin{itemize}
        \item[1.] \textbf{宽度}:$\lambda$ 增大(如红光),$\Delta \theta_0$ 增大,衍射效应显著。
        \item[2.] \textbf{强度}:$I_0 \propto 1/\lambda^2$。红光衍射峰比蓝光低且宽。
    \end{itemize}
    
    \item \textbf{关于相因子}:
    积分中的相因子 $e^{ik_0 L_0}$ 虽然在强度 $I(\theta)$ 计算中被模平方消除,但在全息与光信息处理中(涉及复振幅叠加)至关重要。
\end{itemize}
\subsection{二维孔径衍射}

根据惠更斯-菲涅耳原理,二维孔径的复振幅分布是孔径函数的二维傅里叶变换。

\subsubsection{矩孔衍射}
设矩形孔径宽 $a$($x$ 方向),高 $b$($y$ 方向)。
由于变量 $x, y$ 可分离,总光强分布是两个正交单缝衍射的乘积:
\begin{equation}
    I(x, y) = I_0 \sinc^2(\alpha) \sinc^2(\beta)
\end{equation}
其中 $\alpha = \frac{\pi a \sin \theta_x}{\lambda}, \beta = \frac{\pi b \sin \theta_y}{\lambda}$。
\textbf{图样特征}:呈十字形亮斑,沿孔径较窄的方向衍射扩展较大(反比关系)。

\subsubsection{圆孔衍射}
设圆孔直径为 $D = 2a$。由于旋转对称性,衍射图样为同心圆环。
光强分布涉及一阶贝塞尔函数 $J_1$:
\begin{equation}
    I(\theta) = I_0 \left[ \frac{2 J_1(x)}{x} \right]^2, \quad x = \frac{\pi D}{\lambda} \sin \theta
\end{equation}

\mydef{艾里斑}
中心最大的亮斑称为艾里斑,集中了 83.8\% 的能量。
第一暗环角半径 $\theta_0$ 满足 $x = 3.832$,即:
\begin{equation}
    \sin \theta_0 = 1.22 \frac{\lambda}{D}
\end{equation}
这定义了光学仪器的\textbf{分辨本领}(瑞利判据):两个非相干点源的角距离 $\Delta \theta \ge 1.22 \lambda/D$ 时才能被分辨。