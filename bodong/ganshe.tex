\section{波动光学引论}

\begin{spacing}{1.5}

\subsection{光的电磁理论基础}
\mydef{麦克斯韦方程组与波动方程}
光是电磁波。在无源($\rho=0, \vec{J}=0$)、线性、各向同性、均匀介质中,电场 $\vec{E}$ 和磁场 $\vec{B}$ 均满足矢量波动方程:
\begin{equation}
    \nabla^2 \vec{E} - \mu\varepsilon \frac{\partial^2 \vec{E}}{\partial t^2} = 0
\end{equation}
相速度 $v = 1/\sqrt{\mu\varepsilon}$。真空中 $c \approx \SI{3e8}{m/s}$。
折射率 $n = c/v = \sqrt{\varepsilon_r \mu_r} \approx \sqrt{\varepsilon_r}$(大多数光学介质中 $\mu_r \approx 1$)。

\mydef{光强 }
光强定义为坡印廷矢量  $\vec{S} = \vec{E} \times \vec{H}$ 大小的物理平均值。在同一种介质中,光强正比于电场振幅的平方:
\begin{equation}
    I \propto \langle |\vec{E}|^2 \rangle
\end{equation}

\mydef{标量波近似}
当光波在均匀介质中传播,且不涉及边界上的偏振转换(如菲涅耳反射公式)或各向异性介质(如晶体双折射)时,可忽略电场的矢量性,用标量函数 $U(\vec{r}, t)$ 描述光场。

\subsection{定态光波与复振幅}
核心思想:\textbf{用复数运算简化波动分析}。
对于单色光(角频率 $\omega$),实数波函数 $U(\vec{r}, t) = A(\vec{r}) \cos[\omega t - \varphi(\vec{r})]$ 可写为:
\begin{equation}
    U(\vec{r}, t) = \operatorname{Re} [\tilde{U}(\vec{r}) e^{\ii \omega t}]
\end{equation}
其中 \textbf{复振幅} $\tilde{U}(\vec{r})$ 定义为:
\begin{equation}
    \tilde{U}(\vec{r}) = A(\vec{r}) e^{-\ii \varphi(\vec{r})}
\end{equation}
复振幅满足不含时的 \textbf{亥姆霍兹方程}:
\begin{equation}
    \nabla^2 \tilde{U} + k^2 \tilde{U} = 0, \quad k = \frac{n\omega}{c} = \frac{2\pi}{\lambda}
\end{equation}

\subsection{典型光波及其复振幅}
\begin{enumerate}
    \item \textbf{平面波}:传播方向为 $\vec{k}$。
    $$ \tilde{U}(\vec{r}) = A e^{-\ii \vec{k} \cdot \vec{r}} $$
    等相面方程:$\vec{k} \cdot \vec{r} = \text{C}$(平面)。
    
    \item \textbf{球面波}:从点源发散。
    $$ \tilde{U}(r) = \frac{A}{r} e^{-\ii kr} $$
    振幅 $A/r$ 随距离衰减,能量守恒($I \propto 1/r^2$)。

    \textbf{傍轴近似}:在距离 $z$ 远处,球面波因子 $e^{-\ii kr} \approx e^{-\ii k z} e^{-\ii \frac{k}{2z}(x^2+y^2)}$,这在衍射积分中非常重要。
\end{enumerate}

\subsection{波动传播的重要近似条件与模型}
\mydef{傍轴条件}
当光线传播方向与光轴(设为 $z$ 轴)的夹角 $\theta$ 很小($\theta \ll 1$ rad)时,满足傍轴条件。
\begin{itemize}
    \item \textbf{几何近似}:$\sin\theta \approx \tan\theta \approx \theta$,$\cos\theta \approx 1 - \theta^2/2$。
    \item \textbf{相位近似}:对于球面波传播距离 $r = \sqrt{z^2 + x^2 + y^2}$,泰勒展开保留至二次项:$r \approx z + \frac{x^2+y^2}{2z}$。该近似要求高阶相位项带来的误差远小于 $\pi$,即 $k \frac{(x^2+y^2)^2}{8z^3} \ll \pi$。
    \item \textbf{物理意义}:在傍轴条件下,球面波可近似视为一个带有\textbf{二次曲面相位因子}的平面波。
\end{itemize}

\mydef{远场条件}
通常指观察距离 $z$ 远大于波源或孔径尺寸 $D$ 的平方与波长的比值的情况。
\begin{itemize}
    \item \textbf{判据}:$z \gg \frac{D^2}{\lambda}$
    \item \textbf{物理意义}:在远场区域,波源各点发出的子波到达观察点时,其光程差主要由线性项决定,二次相位项 $\frac{k(x^2+y^2)}{2z}$ 趋于常数或可忽略。此时球面波波前在观察范围内非常平坦,局部可视为\textbf{平面波}。这也是\textbf{夫琅禾费衍射}成立的前提。
\end{itemize}



\end{spacing}

\section{光的干涉}

\subsection{波的叠加与相干条件}
\begin{spacing}{1.5}
\mydef{叠加原理}
$\tilde{U} = \tilde{U}_1 + \tilde{U}_2$。总光强 $I = \tilde{U}\tilde{U}^*$:
\begin{equation}
    I = I_1 + I_2 + 2\sqrt{I_1 I_2} \cos \delta
\end{equation}
干涉项 $J_{12} = 2\sqrt{I_1 I_2} \cos \delta$,其中 $\delta(\vec{r}) = \varphi_2 - \varphi_1$ 为相位差。

\mydef{相干条件 (Coherence Conditions)}
为了观察到稳定的干涉图样($J_{12}$ 不随时间平均为零):
\begin{itemize}
    \item \textbf{频率相同} ($\omega_1 = \omega_2$):否则产生拍频,时间平均光强无干涉项。
    \item \textbf{振动方向平行}:$\vec{E}_1 \cdot \vec{E}_2 \neq 0$(菲涅耳-阿拉戈定律)。
    \item \textbf{相位差恒定}:$\delta$ 不随时间随机跳变。
\end{itemize}

\mydef{条纹可见度 (Visibility/Contrast)}
描述干涉条纹的清晰程度:
\begin{equation}
    V = \frac{I_{\max} - I_{\min}}{I_{\max} + I_{\min}} = \frac{2\sqrt{I_1 I_2}}{I_1 + I_2}
\end{equation}
当 $I_1 = I_2$ 且光源完全相干($|\gamma_{12}|=1$)时,$V=1$(最佳)。

\mydef{驻波}
两列振幅相同、相向传播的相干波叠加形成驻波。维纳实验证明了光波中的感光作用主要是由电场矢量引起的(波腹处感光最强)。
\end{spacing}

\subsection{分波前干涉}
物理上将点光源的波前分割为两部分,分别通过不同路径后汇聚。

\subsubsection{杨氏双缝干涉}
\textbf{装置参数}:缝间距 $d$,屏距 $D$,波长 $\lambda$。
\textbf{光程差} $\Delta \approx d \frac{x}{D}$。
\textbf{光强分布}:
\begin{equation}
    I(x) = I_0(1 + \cos k\frac{d}{D}x)
\end{equation}
\textbf{特征}:
\begin{itemize}
    \item 亮纹:$x_k = k \frac{D\lambda}{d}$;暗纹:$x_k = (k+0.5)\frac{D\lambda}{d}$。
    \item 条纹等间距 $\Delta x = \frac{D\lambda}{d}$。
\end{itemize}

\subsubsection{其他分波前装置}
全部将其转化为杨氏双缝干涉:
\begin{itemize}
    \item \textbf{菲涅耳双棱镜}:利用折射产生两个虚光源。$\Delta x = \frac{(B+C)\lambda}{2\alpha B}$
    \insertfigure{picture/2025-12-24-17-33-17.png}{0}{fig:0}
    \item \textbf{菲涅耳双面镜}:利用反射产生两个虚光源。$\Delta x = \frac{(B+C)\lambda}{2(n-1)\alpha B}$
    \insertfigure{picture/12}{12}{fig:12}
    \item \textbf{洛埃镜}:利用直射光与反射光干涉。$\Delta x = \frac{D\lambda}{2\alpha}$
    \insertfigure{picture/2025-12-24-17-37-11.png}{0}{fig:0}
    \textbf{重要现象}:在镜面边缘接触点处(光程差趋于0),出现暗条纹。这直接证明了光在光密介质表面反射时存在半波损失(相位突变 $\pi$)。
\end{itemize}

\subsubsection{自然光的分解模型}
自然光是大量原子随机发光的集合,宏观上无固定偏振态,但在处理干涉时常采用以下模型:
\begin{itemize}
    \item \textbf{正交分解}:可将自然光等效为两束振动方向互相垂直($x$ 和 $y$)、振幅相等($A_x=A_y$)且\textbf{互不相干}的线偏振光的叠加。
    \item \textbf{光强关系}:$I_{\text{nat}} = I_x + I_y = \frac{1}{2}I_0 + \frac{1}{2}I_0$。
    \item \textbf{干涉表现}:在同一自然光源分成的两束小角度自然光之间的干涉,其衬比度为$\gamma = \frac{1}{2}(1+\cos\alpha)$
\end{itemize}

\subsubsection{空间相干性}
实际光源具有宽度 $b$。
\begin{itemize}
    \item 光源上不同点发出的波列是不相干的,产生的干涉条纹在屏上发生位移。
    \item 当边缘点产生的条纹相对中心点位移达 $\Delta x / 2$ 时,条纹完全模糊。
\end{itemize}
\textbf{相干极限}:
\begin{equation}
    b \cdot \frac{d}{D} \le \lambda \implies b \cdot \theta_s \le \lambda
\end{equation}
其中 $\theta_s \approx d/D$ 为干涉孔径角。
\textbf{结论}:光源宽度越窄,允许的干涉孔径角越大,空间相干性越好。

\subsection{分振幅干涉}

\subsubsection{薄膜干涉基本原理}
\textbf{模型}:折射率 $n_2$,厚度 $h$ 的薄膜,置于 $n_1$ 和 $n_3$ 之间。
\textbf{光程差公式}:
\begin{equation}
    \Delta = 2 n_2 h \cos i_2 + \frac{\lambda}{2} (\text{若存在半波损失})
\end{equation}
\textbf{半波损失判据}:
当 $n_1 < n_2 < n_3$ 或 $n_1 > n_2 > n_3$ 时,上下表面反射性质相同,无附加项。
当 $n_1 < n_2$ 且 $n_2 > n_3$(如空气中肥皂膜),仅上表面有半波损失,需加 $\lambda/2$。

\subsubsection{等倾干涉}
\textbf{条件}:$h$ 均匀,面光源。
\textbf{规律}:$\Delta$ 仅随入射角 $i$ (或折射角 $i_2$) 变化。
\textbf{图样}:定域于无穷远(焦平面)的同心圆环。
\textbf{吞吐现象}:膜厚 $h$ 增加,光程差增大,中心条纹级次冒出(“吐”),条纹整体外扩变密。

\subsubsection{等厚干涉}
\textbf{条件}:$h$ 不均匀,平行光(或准平行光)入射。
\textbf{规律}:$\Delta$ 仅随厚度 $h$ 变化。条纹描绘了膜厚的等高线。
\textbf{定域}:薄膜表面附近。

\textbf{1. 劈尖}
条纹间距 $L = \frac{\lambda}{2n \sin \alpha} \approx \frac{\lambda}{2n\alpha}$。
应用:检测表面平整度(条纹弯曲度)、测量微小直径。

\textbf{2. 牛顿环}
光程差 $\Delta = 2h + \lambda/2 \approx r^2/R + \lambda/2$。
\begin{itemize}
    \item \textbf{反射光}中心为暗斑(接触点 $h=0, \Delta=\lambda/2$)。
    \item \textbf{透射光}中心为亮斑(无半波损失)。
\end{itemize}

\subsubsection{光学薄膜应用}
\begin{itemize}
    \item \textbf{增透膜}:利用干涉相消。单层膜条件:$n_{膜}h = \lambda/4$, $n_{膜} = \sqrt{n_{基}}$。
    \item \textbf{高反膜}:利用干涉相长。多层介质膜堆叠。
\end{itemize}

\subsection{迈克耳孙干涉仪与时间相干性}

\subsubsection{仪器结构与原理}
利用分光板 $G_1$ 分光。补偿板 $G_2$ 保证两臂玻璃光程相等,使仪器能用于白光干涉。
等效为空气层厚度 $d$ 的薄膜干涉。
\begin{itemize}
    \item $M_1, M_2$ 严格垂直 $\to$ 等倾干涉(圆环)。
    \item $M_1, M_2$ 有微小倾角 $\to$ 等厚干涉(直条纹,定域在楔形表面)。
\end{itemize}

\subsubsection{时间相干性}
实际光源非单色(谱宽 $\Delta \nu$ 或 $\Delta \lambda$)。
\textbf{波列长度}:原子发光过程持续时间有限(约 $\SI{e-8}{s}$),形成有限长度波列。
\textbf{相干长度} $L_c$:能够发生干涉的最大光程差。
\begin{equation}
    L_c \approx \frac{c}{\Delta \nu} = \frac{\lambda^2}{\Delta \lambda}
\end{equation}
\textbf{白光干涉}:由于白光 $\Delta \lambda$ 很大,$L_c$ 极短(仅几微米)。仅在零级条纹附近($\Delta \approx 0$)可见彩色条纹,中心为黑色(半波损失)。常用于确定“零光程差”位置。

\subsection{多光束干涉}

\subsubsection{法布里-珀罗 干涉仪}
由两块平行的高反射率($R \to 1$)平板组成。
\textbf{原理}:入射光在腔内多次反射,形成振幅递减、相位等差的无穷多束光叠加。

\todo{这里有内容需要补充}

\subsubsection{艾里公式}
透射光强 $I_T$ 随相位差 $\delta = \frac{4\pi n h \cos i}{\lambda}$ 的分布:
\begin{equation}
    I_T = I_{\max} \frac{1}{1 + F \sin^2 (\delta/2)}
\end{equation}
\textbf{精细度系数}:$F = \frac{4R}{(1-R)^2}$。
\begin{itemize}
    \item $R$ 越大,$F$ 越极大,条纹极其细锐。
    \item 半高全宽 $\varepsilon = 4/\sqrt{F}$。
\end{itemize}

\subsubsection{重要应用参数}
\begin{itemize}
    \item \textbf{自由光谱范围}:相邻两个干涉级次之间的波长差。$\Delta \lambda_{fsr} \approx \frac{\lambda^2}{2nh}$。
    \item \textbf{分辨本领}:$\mathcal{R} = \frac{\lambda}{\delta \lambda} = k \mathcal{N}$,其中 $\mathcal{N}$ 为有效光束数,正比于精细度 $\mathcal{F} = \frac{\pi\sqrt{F}}{2}$。
\end{itemize}
