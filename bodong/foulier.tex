\section{二维傅里叶变换 (Two-Dimensional Fourier Transform)}

在波动光学中,空间复振幅分布 $U(x, y)$ 与其空间频谱 $\mathcal{F}\{U\}$ 构成傅里叶变换对。这揭示了光场在**空域 (Space Domain)** 和**频域 (Frequency Domain)** 的对偶关系。

\subsection{定义与存在条件}
设 $f(x, y)$ 为空间域函数,$u, v$ 为空间频率变量(单位:$\text{mm}^{-1}$)。
\mydef{正变换 (Forward Transform)}
\begin{equation}
    F(u, v) = \mathcal{F}\{f(x, y)\} = \iint_{-\infty}^{\infty} f(x, y) e^{-\ii 2\pi (ux + vy)} \dd x \dd y
\end{equation}

\mydef{逆变换 (Inverse Transform)}
\begin{equation}
    f(x, y) = \mathcal{F}^{-1}\{F(u, v)\} = \iint_{-\infty}^{\infty} F(u, v) e^{\ii 2\pi (ux + vy)} \dd u \dd v
\end{equation}

\textbf{物理意义}:逆变换公式表明,任意复杂的波场 $f(x, y)$ 都可以看作是无穷多个不同方向、不同振幅和相位的**平面波** $e^{\ii 2\pi (ux + vy)}$ 的线性叠加。$F(u, v)$ 即为这些平面波的复振幅权重(角谱)。

\section{常用函数及其频谱 (Common Functions)}

钟锡华教材非常强调对这几个“积木”函数的熟练掌握,它们是分析复杂光学系统的基础。

\subsection{矩形函数 (Rectangle Function)}
定义:$\rect(x) = \begin{cases} 1, & |x| < 1/2 \\ 0, & |x| > 1/2 \end{cases}$
\begin{itemize}
    \item \textbf{二维矩孔}:$f(x, y) = \rect(\frac{x}{a}) \rect(\frac{y}{b})$
    \item \textbf{频谱}:$F(u, v) = |ab| \sinc(au) \sinc(bv)$
    \item \textbf{注}:钟书定义 $\sinc(x) = \frac{\sin(\pi x)}{\pi x}$。
\end{itemize}

\subsection{圆域函数 (Circle Function)}
定义:$\circfunc(r) = \begin{cases} 1, & r < 1 \\ 0, & r > 1 \end{cases}$,其中 $r = \sqrt{x^2+y^2}$。
\begin{itemize}
    \item \textbf{二维圆孔}:$f(x, y) = \circfunc(\frac{\sqrt{x^2+y^2}}{a})$
    \item \textbf{频谱}:$F(\rho) = |a|^2 \frac{J_1(2\pi a \rho)}{a \rho}$,其中 $\rho = \sqrt{u^2+v^2}$。
    \item 这是艾里斑强度的振幅基础。
\end{itemize}

\subsection{狄拉克 $\delta$ 函数 (Delta Function)}
\mydef{筛选性质}
$\int_{-\infty}^{\infty} f(x) \delta(x-x_0) \dd x = f(x_0)$。
\begin{itemize}
    \item $\mathcal{F}\{\delta(x, y)\} = 1$ (点源的频谱是均匀的平面波谱)。
    \item $\mathcal{F}\{1\} = \delta(u, v)$ (均匀平面波对应单一空间频率)。
\end{itemize}

\subsection{梳状函数 (Comb Function)}
定义:$\comb(x) = \sum_{n=-\infty}^{\infty} \delta(x-n)$。
\begin{itemize}
    \item \textbf{性质}:$\mathcal{F}\{\comb(x)\} = \comb(u)$。梳状函数的傅里叶变换仍是梳状函数。
    \item \textbf{应用}:描述光栅、采样过程。
\end{itemize}

\section{基本定理与卷积 (Basic Theorems)}

\subsection{位移定理 (Shift Theorem)}
\begin{itemize}
    \item \textbf{空域位移} $\leftrightarrow$ \textbf{频域线性相移}:
    $$ f(x-x_0, y-y_0) \Longleftrightarrow F(u, v) e^{-\ii 2\pi (u x_0 + v y_0)} $$
    \item \textbf{频域位移} $\leftrightarrow$ \textbf{空域线性相移}:
    $$ f(x, y) e^{\ii 2\pi (u_0 x + v_0 y)} \Longleftrightarrow F(u-u_0, v-v_0) $$
    这是**斜入射**照明分析的基础。
\end{itemize}

\subsection{卷积定理 (Convolution Theorem)}
定义卷积:$g(x, y) = f(x, y) * h(x, y) = \iint f(\xi, \eta) h(x-\xi, y-\eta) \dd \xi \dd \eta$。
\begin{itemize}
    \item \textbf{时域卷积} $\leftrightarrow$ \textbf{频域乘积}:
    $$ \mathcal{F}\{f * h\} = \mathcal{F}\{f\} \cdot \mathcal{F}\{h\} $$
    这构成了**线性系统理论**的基础:输出 = 输入 * 脉冲响应 $\implies$ 输出频谱 = 输入频谱 $\cdot$ 传递函数。
\end{itemize}

\subsection{巴塞伐尔定理 (Parseval's Theorem)}
能量守恒在空域和频域的体现:
$$ \iint |f(x, y)|^2 \dd x \dd y = \iint |F(u, v)|^2 \dd u \dd v $$

\section{相因子分析 (Phase Factor Analysis)}

这是本章的精华。在傍轴光学中,光场的变换主要体现为相位的调制。钟锡华老师将复杂的衍射积分简化为对两种核心相因子的操作。

\subsection{线性相因子 (Linear Phase Factor)}
形式:$e^{\ii 2\pi u_0 x}$ 或 $e^{\ii k x \sin \theta}$。
\begin{itemize}
    \item \textbf{物理意义}:代表光波传播方向的改变(倾斜)。
    \item \textbf{作用}:
        \begin{itemize}
            \item 在**空域**乘以线性相因子 $\implies$ **频谱**发生平移(位移定理)。
            \item 例如:棱镜的作用就是引入线性相因子。
        \end{itemize}
\end{itemize}

\subsection{二次相因子 (Quadratic Phase Factor)}
形式:$e^{\ii \frac{k}{2R} (x^2+y^2)}$。
\begin{itemize}
    \item \textbf{物理意义}:傍轴近似下的**球面波**。
    \item \textbf{参数符号}:
        \begin{itemize}
            \item $R > 0$:发散球面波(点源在左)。
            \item $R < 0$:会聚球面波(焦点在右)。
        \end{itemize}
    \item \textbf{自由空间传播}:
    菲涅耳衍射积分核即包含二次相因子 $e^{\ii \frac{k}{2z} (x^2+y^2)}$。这表明光在自由空间传播一段距离 $z$,相当于与一个二次相因子进行卷积。
\end{itemize}

\subsection{薄透镜的位相变换作用}
透镜是光学信息处理的核心元件。
\mydef{透镜的复振幅透射率}
$$ t_l(x, y) = e^{-\ii \frac{k}{2f} (x^2+y^2)} $$
(忽略常数相位因子 $e^{\ii k n \Delta_0}$)
\begin{itemize}
    \item \textbf{物理图像}:透镜将入射的平面波(波前平直)转化为会聚球面波(波前弯曲)。
    \item \textbf{相位抵消}:若入射光是发散球面波 $e^{\ii \frac{k}{2d_1}(x^2+y^2)}$,经过透镜后:
    $$ U_{out} = e^{\ii \frac{k}{2d_1}(x^2+y^2)} \cdot e^{-\ii \frac{k}{2f}(x^2+y^2)} = e^{\ii \frac{k}{2} (\frac{1}{d_1} - \frac{1}{f}) (x^2+y^2)} $$
    若满足成像公式 $\frac{1}{d_1} + \frac{1}{d_2} = \frac{1}{f}$,则输出变为 $e^{-\ii \frac{k}{2d_2}(x^2+y^2)}$,即会聚到 $d_2$ 处的球面波。
\end{itemize}

\subsection{透镜的傅里叶变换性质}
这是傅里叶光学的核心结论。
\textbf{结论}:若物体置于透镜的**前焦面**,则在透镜的**后焦面**上,光场分布准确地正比于物函数的傅里叶变换。
$$ U_f(x_f, y_f) \propto \mathcal{F}\{t_0(x_0, y_0)\} \bigg|_{u = \frac{x_f}{\lambda f}, v = \frac{y_f}{\lambda f}} $$
透镜的作用就是在后焦面上物理地实现二维傅里叶变换。
