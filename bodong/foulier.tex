
\section{相因子分析}

这是本章的精华。在傍轴光学中,光场的变换主要体现为相位的调制。钟锡华老师将复杂的衍射积分简化为对两种核心相因子的操作。

\subsection{线性相因子}
形式:$e^{\ii 2\pi u_0 x}$ 或 $e^{\ii k x \sin \theta}$。
\begin{itemize}
    \item \textbf{物理意义}:代表光波传播方向的改变(倾斜)。
    \item \textbf{作用}:
        \begin{itemize}
            \item 在空域乘以线性相因子 $\implies$ 频谱发生平移(位移定理)。
            \item 例如:棱镜的作用就是引入线性相因子$e^{-\ii k(n-1)\alpha x}$。
        \end{itemize}
\end{itemize}
\insertfigure{picture/2026-01-04-15-12-25.png}{0}{fig:0}
\subsection{二次相因子}
形式:$e^{\ii \frac{k}{2R} (x^2+y^2)}$
\begin{itemize}
    \item \textbf{物理意义}:傍轴近似下的球面波。
    \item \textbf{参数符号}:
        \begin{itemize}
            \item $R > 0$:发散球面波(点源在左)。
            \item $R < 0$:会聚球面波(焦点在右)。
        \end{itemize}
    \item \textbf{自由空间传播}:
    菲涅耳衍射积分核即包含二次相因子 $e^{\ii \frac{k}{2z} (x^2+y^2)}$。这表明光在自由空间传播一段距离 $z$,相当于与一个二次相因子进行卷积。
\end{itemize}

\subsection{薄透镜的位相变换作用}
透镜是光学信息处理的核心元件。
\mydef{透镜的复振幅透射率}
$$ t_l(x, y) = e^{-\ii \frac{k}{2f} (x^2+y^2)} $$
(忽略常数相位因子 $e^{\ii k n \Delta_0}$)
\begin{itemize}
    \item \textbf{物理图像}:透镜将入射的平面波(波前平直)转化为会聚球面波(波前弯曲)。
    \item \textbf{相位抵消}:若入射光是发散球面波 $e^{\ii \frac{k}{2d_1}(x^2+y^2)}$,经过透镜后:
    $$ U_{out} = e^{\ii \frac{k}{2d_1}(x^2+y^2)} \cdot e^{-\ii \frac{k}{2f}(x^2+y^2)} = e^{\ii \frac{k}{2} (\frac{1}{d_1} - \frac{1}{f}) (x^2+y^2)} $$
    若满足成像公式 $\frac{1}{d_1} + \frac{1}{d_2} = \frac{1}{f}$,则输出变为 $e^{-\ii \frac{k}{2d_2}(x^2+y^2)}$,即会聚到 $d_2$ 处的球面波。
\end{itemize}
\subsection{余弦光栅}
这是光学信息处理中最基础的组件之一,也是理解复杂滤波器(如全息图)的基石。

\subsubsection{屏函数与制备}
\textbf{定义}:透射率函数呈余弦变化的衍射屏。
\begin{equation}
    t(x,y) = t_0 + t_1 \cos(2\pi f_0 x + \varphi_0)
\end{equation}
其中 $f_0 = 1/d$ 为空间频率,$d$ 为光栅常数。
\begin{itemize}
    \item \textbf{制备原理}:利用双光束干涉全息记录。将两束平面波的干涉条纹 $I(x) = I_0(1+\gamma \cos(2\pi f x))$ 记录在线性感光材料上。
    \item \textbf{线性条件}:要求显影后的振幅透射率 $t$ 与曝光量 $E$ (即强度 $I$) 成线性关系,即 $t = \alpha + \beta I$。
\end{itemize}

\subsubsection{衍射特征}
利用欧拉公式展开屏函数,可将出射光场分解为三项:
\begin{equation}
    U_{out}(x) = A \cdot t(x) = \underbrace{A t_0}_{0\text{级}} + \underbrace{\frac{1}{2} A t_1 e^{\ii(2\pi f_0 x + \varphi_0)}}_{+1\text{级}} + \underbrace{\frac{1}{2} A t_1 e^{-\ii(2\pi f_0 x + \varphi_0)}}_{-1\text{级}}
\end{equation}
\textbf{物理图景}:
\begin{itemize}
    \item \textbf{0级斑}:沿原方向传播的平面波。
    \item \textbf{$\pm 1$级斑}:分别向上、下倾斜的平面波,衍射角满足 $\sin \theta_{\pm 1} = \pm f_0 \lambda$。
    \item \textbf{特点}:余弦光栅\textbf{只产生 0 级和 $\pm 1$ 级}三个衍射斑,没有更高级次,这是其与普通黑白光栅最大的区别。
\end{itemize}
\subsection{阿贝成像原理}
阿贝提出显微镜成像的“两步成像论”,深刻揭示了相干成像的物理机制。
\insertfigure{picture/2026-01-04-15-39-58.png}{0}{fig:0}
\begin{itemize}
    \item \textbf{第一步:分频(衍射)}
    物体(光栅)在相干光照明下,通过物镜,在透镜后焦面上形成夫琅禾费衍射图样。这个焦平面被称为\textbf{频谱面}。每一点对应物体的一个空间频率分量。
    \item \textbf{第二步:合成(干涉)}
    频谱面上发出的次波在像面上相干叠加,重建(合成)出像。
\end{itemize}
\textbf{核心结论}:成像过程本质上是两次傅里叶变换($\mathcal{F} \to \mathcal{F}^{-1}$)。改变频谱面上的光阑(空间滤波器),就能改变像的结构。
\subsection{4F系统}
4F系统是最标准的光学信息处理系统。由两块焦距为 $f$ 的透镜组成,物面、透镜L1、频谱面、透镜L2、像面依次间隔 $f$ 排列。
\begin{itemize}
    \item \textbf{输入面 ($P_1$)}:放置物体 $t(x,y)$。
    \item \textbf{频谱面 ($P_2$)}:位于 L1 后焦面(即 L2 前焦面)。在此处光场分布为 $T(u,v) = \mathcal{F}\{t\}$。
    \item \textbf{输出面 ($P_3$)}:位于 L2 后焦面。在此处光场为 $t(-x, -y)$(倒立实像)。
\end{itemize}

\subsection{图像加减运算(基于余弦光栅)}
利用 4F 系统和余弦光栅滤波器,通过控制相位实现图像 $A$ 和 $B$ 的代数运算。
\insertfigure{picture/2026-01-04-15-37-54.png}{0}{fig:0}
\subsubsection{原理与光路}
\begin{itemize}
    \item \textbf{输入面}:放置两幅图像,中心分别位于 $x=0$ (图像A) 和 $x=-a$ (图像B)。
    \item \textbf{频谱面}:放置频率为 $f_0$ 的余弦光栅作为滤波器。
    \item \textbf{输出面}:余弦光栅将每个图像的频谱一分为三。
        \begin{itemize}
            \item 图像 A 的 $+1$ 级像 $A_{+1}$ 向下平移。
            \item 图像 B 的 $-1$ 级像 $B_{-1}$ 向上平移。
        \end{itemize}
\end{itemize}

\subsubsection{重合与运算条件}
为了让 $A_{+1}$ 与 $B_{-1}$ 在输出面上完全重合,必须满足\textbf{几何位置匹配条件}:
\begin{equation}
    a = 2 f_0 \lambda F
\end{equation}
此时,重合区域的复振幅为两图像的叠加。通过微调光栅滤波器的位置 $\Delta u$,引入附加相位差 $\delta$:
\begin{itemize}
    \item \textbf{图像相加}:当 $\delta = 0, 2\pi$ 时,输出 $A+B$。
    \item \textbf{图像相减}:当 $\delta = \pi$ 时,输出 $A-B$(实现差分)。
\end{itemize}
实现“相减”所需的\textbf{特征位移量} $\Delta u_0$ 为光栅周期的四分之一:
\begin{equation}
    \Delta u_0 = \frac{1}{4 f_0} = \frac{d_0}{4}
\end{equation}
\subsection{图像微分(基于复合光栅)}
图像微分的本质是提取图像的边缘,即计算 $\tilde{t}(x+\Delta x, y) - \tilde{t}(x,y)$。
\insertfigure{picture/2026-01-04-15-37-32.png}{0}{fig:0}
\subsubsection{复合光栅滤波器}
为了同时实现“微小错位”和“相位相反”,使用包含两个接近频率 $f_1$ 和 $f_2$ 的\textbf{复合光栅}:
\begin{equation}
    \tilde{H}(u) = t_0 + t_1 \cos(2\pi f_1 u) + t_2 \cos(2\pi f_2 u)
\end{equation}
设差频 $\Delta f = f_2 - f_1 \ll f_1, f_2$。

\subsubsection{运算机制}
\begin{enumerate}
    \item \textbf{微小错位}:由于频率不同,两组 $\pm 1$ 级衍射光的衍射角略有差异,导致在像面上形成两个微小错位的像。错位量 $\Delta x'$ 为:
    \begin{equation}
        \Delta x' = \lambda F \Delta f
    \end{equation}
    \item \textbf{$\pi$ 相位差}:通过精密平移复合光栅 $\Delta u$,使两个频率分量产生相对相移。为了实现相减(微分),需满足相位差 $\delta = \pi$。
    所需的滤波器位移量为:
    \begin{equation}
        \Delta u_0 = \frac{1}{2 \Delta f}
    \end{equation}
\end{enumerate}
\textbf{结果}:在像面上获得了图像的差分 $\frac{\partial \tilde{t}}{\partial x} \Delta x'$,实现了边缘增强效果,如熊猫图所示,轮廓鲜明而内部被抵消。
\subsubsection{显色滤波($\theta$ 调制)}
一种将黑白图像假彩色化的技术。
\begin{itemize}
    \item \textbf{编码}:将物体不同灰度或区域,分别用不同取向($\theta$)的光栅进行编码(例如:天空区域贴横条纹,地面区域贴竖条纹)。
    \item \textbf{频谱分离}:在 4F 系统的频谱面上,不同取向的光栅衍射斑出现在不同方位角上。
    \item \textbf{滤波与染色}:在频谱面上不同方位的衍射斑处放置不同颜色的滤色片。
    \item \textbf{合成}:像面上不同区域将呈现不同颜色。
\end{itemize}
\insertfigure{picture/2026-01-04-15-37-11.png}{0}{fig:0}
\subsection{泽尼克相衬法}
\insertfigure{picture/2026-01-04-15-36-46.png}{0}{fig:0}
\textbf{背景}:生物标本等“相位物体”透明度均匀,仅改变透射光的相位。人眼和普通探测器只能分辨光强(振幅),看不见相位变化。
\textbf{原理}:将不可见的相位信息转换为可见的振幅(光强)变化。

设相位物体复透射率为 $t(x) = e^{\ii \phi(x)}$。当 $\phi(x) \ll 1$ 时(弱相位近似):
\begin{equation}
    t(x) \approx 1 + \ii \phi(x)
\end{equation}
\begin{itemize}
    \item \textbf{直射光(背景)}:对应常数项 $1$,集中在频谱面中心(零频)。
    \item \textbf{衍射光(信息)}:对应 $\ii \phi(x)$,分布在频谱面周围。注意两者有 $\pi/2$ 的相位差(因子 $\ii$)。
\end{itemize}
普通显微镜下光强 $I \approx |1 + \ii \phi|^2 \approx 1 + \phi^2 \approx 1$(反差极小)。



\textbf{相衬显微镜操作}:
在频谱面中心(零频位置)放置一块\textbf{相板},使直射光发生 $\pm \pi/2$ 的相移(乘以因子 $\pm \ii$)。
\begin{itemize}
    \item \textbf{正相衬}:相移 $+\pi/2$(乘以 $\ii$)。直射光变为 $\ii$。
    像面复振幅:$U_{im} \approx \ii \cdot 1 + \ii \phi(x) = \ii [1 + \phi(x)]$。
    像面光强:$I \approx |1 + \phi(x)|^2 \approx 1 + 2\phi(x)$。
    \textbf{结果}:相位越大的地方越亮。
    \item \textbf{负相衬}:相移 $-\pi/2$(乘以 $-\ii$)。光强 $I \approx 1 - 2\phi(x)$。
\end{itemize}
此方法即荣获诺贝尔奖的“泽尼克相衬法”。
\subsection{透镜的傅里叶变换性质}
这是傅里叶光学的核心结论。
\textbf{结论}:若物体置于透镜的前焦面,则在透镜的后焦面上,光场分布准确地正比于物函数的傅里叶变换。
$$ U_f(x_f, y_f) \propto \mathcal{F}\{t_0(x_0, y_0)\} \bigg|_{u = \frac{x_f}{\lambda f}, v = \frac{y_f}{\lambda f}} $$
透镜的作用就是在后焦面上物理地实现二维傅里叶变换。
