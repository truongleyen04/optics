\section{一维多元结构:光栅衍射 (Diffraction Gratings)}

光栅是大量等宽、等间距的平行狭缝(或反射面)的组合。它是多光束干涉与单缝衍射的综合体现。

\subsection{光栅衍射公式}
设光栅有 $N$ 条狭缝,缝宽为 $a$,不透光部分宽度为 $b$,光栅常数(周期) $d = a + b$。
对于波长 $\lambda$,衍射角 $\theta$,光强分布 $I(\theta)$ 为:
\begin{equation}
    I(\theta) = I_0 \underbrace{\left( \frac{\sin \alpha}{\alpha} \right)^2}_{\text{单缝衍射因子}} \cdot \underbrace{\left( \frac{\sin (N \delta / 2)}{\sin (\delta / 2)} \right)^2}_{\text{多缝干涉因子}}
\end{equation}
其中参数定义为:
\begin{itemize}
    \item \textbf{半波带相位差}:$\alpha = \frac{\pi a}{\lambda} \sin \theta$
    \item \textbf{槽间相位差}:$\delta = \frac{2\pi d}{\lambda} \sin \theta$
\end{itemize}

\subsection{主极大与极小 (Maxima and Minima)}
\mydef{主极大 (Principal Maxima)}
由干涉因子决定。当分母 $\sin(\delta/2) = 0$ 时,利用洛必达法则可得干涉因子趋于 $N^2$。
条件:$\frac{\delta}{2} = k\pi \quad (k = 0, \pm 1, \pm 2, \dots)$。
即光栅方程:
\begin{equation}
    d \sin \theta = k \lambda
\end{equation}
\textbf{特征}:亮度极大($I \propto N^2$),条纹细锐。

\mydef{极小值 (Minima)}
当分子 $\sin(N\delta/2) = 0$ 但分母不为 0 时,光强为 0。
条件:$\frac{N\delta}{2} = m\pi$ 且 $m \neq N \cdot \text{整数}$。
\textbf{特征}:在两个主极大之间,有 $N-1$ 个暗纹。

\mydef{次极大 (Secondary Maxima)}
在两个主极大之间,存在 $N-2$ 个次极大。随着 $N$ 增大,次极大光强迅速衰减(相比主极大可忽略),背景变得非常干净。

\subsection{缺级现象 (Missing Orders)}
当某一干涉主极大级次 $k$ 的位置,恰好落在单缝衍射的暗纹位置(衍射因子为 0)时,该级主极大消失。
\begin{itemize}
    \item 干涉主极大条件:$d \sin \theta = k \lambda$
    \item 衍射暗纹条件:$a \sin \theta = m' \lambda$
\end{itemize}
两式相比得\textbf{缺级条件}:
\begin{equation}
    \frac{d}{a} = \frac{k}{m'}
\end{equation}
例如:若 $d=3a$,则 $k = \pm 3, \pm 6, \dots$ 等级次缺级。

\section{二维孔径衍射 (Diffraction by 2D Apertures)}

根据惠更斯-菲涅耳原理,二维孔径的复振幅分布是孔径函数的二维傅里叶变换。

\subsection{矩孔衍射 (Rectangular Aperture)}
设矩形孔径宽 $a$($x$ 方向),高 $b$($y$ 方向)。
由于变量 $x, y$ 可分离,总光强分布是两个正交单缝衍射的乘积:
\begin{equation}
    I(x, y) = I_0 \sinc^2(\alpha) \sinc^2(\beta)
\end{equation}
其中 $\alpha = \frac{\pi a \sin \theta_x}{\lambda}, \beta = \frac{\pi b \sin \theta_y}{\lambda}$。
\textbf{图样特征}:呈十字形亮斑,沿孔径较窄的方向衍射扩展较大(反比关系)。

\subsection{圆孔衍射 (Circular Aperture)}
设圆孔直径为 $D = 2a$。由于旋转对称性,衍射图样为同心圆环。
光强分布涉及一阶贝塞尔函数 $J_1$:
\begin{equation}
    I(\theta) = I_0 \left[ \frac{2 J_1(x)}{x} \right]^2, \quad x = \frac{\pi D}{\lambda} \sin \theta
\end{equation}

\mydef{艾里斑 (Airy Disk)}
中心最大的亮斑称为艾里斑,集中了 83.8\% 的能量。
第一暗环角半径 $\theta_0$ 满足 $x = 3.832$,即:
\begin{equation}
    \sin \theta_0 = 1.22 \frac{\lambda}{D}
\end{equation}
这定义了光学仪器的\textbf{分辨本领}(瑞利判据):两个非相干点源的角距离 $\Delta \theta \ge 1.22 \lambda/D$ 时才能被分辨。

\section{分形光学基础 (Fractal Optics)}

钟锡华老师教材的特色章节。分形 (Fractal) 的核心特征是\textbf{自相似性 (Self-similarity)},即局部与整体在形态上相似。

\subsection{分形结构的衍射}
当衍射屏的透光孔径具有分形结构时,其远场衍射图样(频谱)也往往表现出分形特征。

\mydef{康托尔集 (Cantor Set)}
最简单的一维分形。
\begin{itemize}
    \item \textbf{生成规则}:将一条线段三等分,去掉中间一段;对剩下的两段重复此操作,无限迭代。
    \item \textbf{分维数}:$D_f = \frac{\ln 2}{\ln 3} \approx 0.6309$。
\end{itemize}

\subsection{康托尔集的衍射图样}
\begin{itemize}
    \item \textbf{结构自相似}:康托尔光栅在空间结构上具有 $S=3$ 的标度不变性。
    \item \textbf{频谱自相似}:其衍射光强分布在空间频率域也表现出“层层嵌套”的包络结构。
    \item \textbf{主要结论}:$N$ 级康托尔集的衍射因子是各级单缝衍射因子与多缝干涉因子的无穷乘积。衍射图样中会出现“光强隙”,对应于空间结构中的“空隙”。
\end{itemize}

\subsection{二维分形:谢尔宾斯基地毯与三角}
\begin{itemize}
    \item \textbf{谢尔宾斯基地毯 (Sierpinski Carpet)}:将正方形九等分,挖去中间一个;迭代。
    \item \textbf{衍射特征}:衍射图样也是二维分形结构,具有极丰富的高频分量和嵌套的零点分布。
\end{itemize}
