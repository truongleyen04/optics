\section{典型干涉装置 (Interferometers)}

除了前述的杨氏双缝(分波前)和薄膜干涉(分振幅),还有几种重要的干涉仪在现代光学中应用广泛。

\subsection{迈克耳孙干涉仪 (Michelson Interferometer)}
\begin{itemize}
    \item \textbf{结构}:分光板 $G_1$ 将光束分为互相垂直的两臂,经反射镜 $M_1, M_2$ 反射后汇聚。
    \item \textbf{特点}:两臂完全分离,可以独立调节光程差。
    \item \textbf{应用}:
        \begin{itemize}
            \item 测量长度和微小位移。
            \item \textbf{傅里叶变换光谱仪 (FTS)}:通过测量干涉图样随光程差的变化(干涉图),经傅里叶变换得到光源的光谱分布。
        \end{itemize}
\end{itemize}

\subsection{马赫-曾德尔干涉仪 (Mach-Zehnder Interferometer)}
\begin{itemize}
    \item \textbf{结构}:利用两个分光板和两个全反射镜,使两束光在空间上\textbf{单向传输}且分离较远,最后汇聚。
    \item \textbf{特点}:工作空间大,被测物体仅置于其中一臂,不重叠光路。
    \item \textbf{应用}:广泛用于空气动力学流场显示(如风洞实验)、等离子体诊断及透明相位物体的折射率分布测量。
\end{itemize}

\subsection{萨格纳克干涉仪 (Sagnac Interferometer)}
\begin{itemize}
    \item \textbf{结构}:分光后的两束光沿同一闭合光路\textbf{反向旋转}传播,最后汇聚干涉。
    \item \textbf{共光路 (Common-path)}:两束光经过相同的光学元件,对抗震动和环境干扰能力极强。
    \item \textbf{萨格纳克效应}:当干涉仪整体相对于惯性空间旋转时,顺时针和逆时针传播的光产生相位差:
    \begin{equation}
        \Delta \varphi = \frac{4\pi \mathcal{A} \Omega}{\lambda c}
    \end{equation}
    其中 $\mathcal{A}$ 为环路面积,$\Omega$ 为旋转角速度。
    \item \textbf{应用}:光纤陀螺仪 (FOG)、激光陀螺仪,用于惯性导航。
\end{itemize}

\section{光场的时空相干性 (Coherence of Light Fields)}

干涉的本质是**相关性 (Correlation)**。实际光场是随机涨落的,必须用统计光学理论描述。

\subsection{互相干函数 (Mutual Coherence Function)}
设光场中两点 $\vec{r}_1, \vec{r}_2$ 在时刻 $t$ 和 $t+\tau$ 的复振幅分别为 $U(\vec{r}_1, t)$ 和 $U(\vec{r}_2, t+\tau)$。
定义**互相干函数** $\Gamma_{12}(\tau)$ 为两点光场的互相关:
\begin{equation}
    \Gamma_{12}(\tau) = \langle U(\vec{r}_1, t) U^*(\vec{r}_2, t+\tau) \rangle = \lim_{T \to \infty} \frac{1}{2T} \int_{-T}^{T} U(\vec{r}_1, t) U^*(\vec{r}_2, t+\tau) \dd t
\end{equation}
定义归一化的**复相干度 (Complex Degree of Coherence)**:
\begin{equation}
    \gamma_{12}(\tau) = \frac{\Gamma_{12}(\tau)}{\sqrt{\Gamma_{11}(0) \Gamma_{22}(0)}} = \frac{\Gamma_{12}(\tau)}{\sqrt{I_1 I_2}}
\end{equation}
\textbf{一般干涉定律}:
\begin{equation}
    I = I_1 + I_2 + 2\sqrt{I_1 I_2} |\gamma_{12}(\tau)| \cos[\alpha_{12}(\tau) - \delta]
\end{equation}
其中条纹可见度 $V = |\gamma_{12}(\tau)|$(设 $I_1=I_2$)。

\subsection{时间相干性 (Temporal Coherence)}
只考虑同一点 $\vec{r}$ 在不同时刻的相关性,即 $\vec{r}_1 = \vec{r}_2$。
\mydef{物理根源}
光源的非单色性(频谱宽度 $\Delta \nu$)。

\mydef{自相干函数与功率谱密度}
根据**维纳-欣钦定理 (Wiener-Khinchin Theorem)**,光场的自相干函数 $\Gamma(\tau)$ 与其功率谱密度 $S(\nu)$ 构成傅里叶变换对:
\begin{equation}
    \Gamma(\tau) \Longleftrightarrow S(\nu)
\end{equation}
这意味着:\textbf{光谱越窄(单色性越好),相干时间越长}。

\mydef{相干时间与相干长度}
\begin{itemize}
    \item \textbf{相干时间} $\tau_c \approx 1 / \Delta \nu$。
    \item \textbf{相干长度} $L_c = c \tau_c \approx \lambda^2 / \Delta \lambda$。
\end{itemize}

\subsection{空间相干性 (Spatial Coherence)}
只考虑同一时刻 $\tau=0$ 不同空间点 $\vec{r}_1, \vec{r}_2$ 的相关性。
\mydef{物理根源}
光源的扩展线度(非点光源)。

\mydef{范西泰特-泽尼克定理 (Van Cittert-Zernike Theorem)}
对于非相干扩展光源,其远场照明区域的复相干度 $\gamma_{12}(0)$ 等于光源强度分布的归一化傅里叶变换。
简单推论:对于直径为 $b$ 的圆盘光源,在距离 $R$ 处,两个横向间距为 $d$ 的小孔处的相干度为:
\begin{equation}
    |\gamma_{12}(0)| = \left| \frac{2 J_1(u)}{u} \right|, \quad u = \frac{\pi b d}{\lambda R}
\end{equation}
其中 $J_1$ 为一阶贝塞尔函数。
\textbf{相干面积}:相干度下降到一定值(如 0.88)的空间范围。
$$ \text{相干孔径角} \, \theta_c \approx \frac{\lambda}{b} $$

\section{激光 (Lasers)}

激光 (LASER) 是 "Light Amplification by Stimulated Emission of Radiation" 的缩写。

\subsection{激光产生的三个条件}
\begin{enumerate}
    \item \textbf{受激辐射 (Stimulated Emission)}:
    在外来光子诱发下,处于高能级 $E_2$ 的原子跃迁到低能级 $E_1$,发出一口模一样(同频率、同相位、同偏振、同方向)的光子。
    \item \textbf{粒子数反转 (Population Inversion)}:
    必须使得高能级粒子数 $N_2 > N_1$(玻尔兹曼分布通常 $N_2 < N_1$),使受激辐射超过受激吸收。需借助泵浦源(光、电等)激励。
    \item \textbf{光学谐振腔 (Optical Resonator)}:
    \begin{itemize}
        \item 提供正反馈:光在腔内往返放大。
        \item 频率选择:满足驻波条件 $L = q \frac{\lambda}{2}$ 的频率才能振荡。
        \item 方向选择:只有沿轴向传播的光才能维持。
    \end{itemize}
\end{enumerate}

\subsection{激光的特性}
\begin{enumerate}
    \item \textbf{单色性好}:时间相干性极高,线宽极窄(可达 $\SI{e-8}{nm}$ 甚至更低)。
    \item \textbf{方向性好}:空间相干性好,发散角小(衍射极限量级)。
    \item \textbf{亮度极高}:单位面积、单位立体角内的光功率极大。
    \item \textbf{相干性强}:同时具备极高的时间相干性和空间相干性,体积相干区大。
\end{enumerate}

\subsection{纵模与横模}
\begin{itemize}
    \item \textbf{纵模 (Longitudinal Modes)}:
    由腔长 $L$ 决定的轴向驻波形式。频率间隔 $\Delta \nu = c / 2nL$。单纵模激光对应极好的单色性。
    \item \textbf{横模 (Transverse Modes)}:
    光场在横截面上的振幅分布,通常用 TEM$_{mn}$ 表示。
    TEM$_{00}$ 模(基模):光强呈高斯分布(Gaussian Beam),发散角最小,聚焦光斑最小。
\end{itemize}
