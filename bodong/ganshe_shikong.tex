\section{干涉装置和光场时空相干性}

\subsection{分波前装置}
全部将其转化为杨氏双缝干涉:
\begin{itemize}
    \item \textbf{菲涅耳双棱镜}:利用折射产生两个虚光源。$\Delta x = \frac{(B+C)\lambda}{2\alpha B}$
    \insertfigure{picture/2025-12-24-17-33-17.png}{0}{fig:0}
    \item \textbf{菲涅耳双面镜}:利用反射产生两个虚光源。$\Delta x = \frac{(B+C)\lambda}{2(n-1)\alpha B}$
    \insertfigure{picture/12}{12}{fig:12}
    \item \textbf{洛埃镜}:利用直射光与反射光干涉。$\Delta x = \frac{D\lambda}{2\alpha}$
    \insertfigure{picture/2025-12-24-17-37-11.png}{0}{fig:0}
    \textbf{重要现象}:在镜面边缘接触点处(光程差趋于0),出现暗条纹。这直接证明了光在光密介质表面反射时存在半波损失(相位突变 $\pi$)。
\end{itemize}

\subsection{点源位移导致条纹移动}
当点光源 $S$ 偏离光轴时,零级亮纹的位置也会发生改变。
\insertfigure{picture/2025-12-25-15-10-45.png}{0}{fig:0}
\begin{enumerate}
    \item \textbf{现象}:若点光源 $S$ 在垂直于光轴方向上移动 $\delta_s$,则干涉条纹整体将向\textbf{相反}方向移动。
    \item \textbf{定量关系}:
    设光源到双缝平面的距离为 $R$(对应上文 $B$),双缝到观察屏的距离为 $D$(对应上文 $C$)。
    根据光程差公式 $\Delta r = \frac{d}{D}x + \frac{d}{R}\delta_s = 0$(零级亮纹条件),可得条纹位移 $\delta x$:
    \[ \delta x = - \frac{D}{R} \delta_s \]
    \item \textbf{物理意义}:
    \begin{itemize}
        \item 负号表示条纹移动方向与光源移动方向相反。
        \item 位移量被放大了 $D/R$ 倍(干涉系统的放大倍率)。
    \end{itemize}
\end{enumerate}

\subsection{空间相干性}
只考虑同一时刻 $\tau=0$ 不同空间点 $\vec{r}_1, \vec{r}_2$ 的相关性。
\mydef{物理根源}
光源的扩展线度(非点光源)。
\insertfigure{picture/2025-12-25-15-11-50.png}{0}{fig:0}
\textbf{相干面积}:相干度下降到一定值(基本为$b_0^2$的空间范围)。
$$ \text{相干孔径角} \, \theta_c \approx \frac{\lambda}{b} $$
 $$\theta_c = \frac{b}{R} $$

\subsection{时间相干性}
只考虑同一点 $\vec{r}$ 在不同时刻的相关性,即 $\vec{r}_1 = \vec{r}_2$。
\mydef{物理根源}
光源的非单色性(频谱宽度 $\Delta \nu$)。


这意味着:\textbf{光谱越窄(单色性越好),相干时间越长}。

\mydef{相干时间与相干长度}
\begin{itemize}
    \item \textbf{相干时间} $\tau_c \approx 1 / \Delta \nu$。
    \item \textbf{相干长度} $L_c = c \tau_c \approx \lambda^2 / \Delta \lambda$。
\end{itemize}


