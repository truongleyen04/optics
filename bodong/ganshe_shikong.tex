\section{干涉装置和光场时空相干性}

\subsection{分波前装置}
全部将其转化为杨氏双缝干涉:
\begin{itemize}
    \item \textbf{菲涅耳双棱镜}:利用折射产生两个虚光源。$\Delta x = \frac{(B+C)\lambda}{2\alpha B}$
    \insertfigure{picture/2025-12-24-17-33-17.png}{0}{fig:0}
    \item \textbf{菲涅耳双面镜}:利用反射产生两个虚光源。$\Delta x = \frac{(B+C)\lambda}{2(n-1)\alpha B}$
    \insertfigure{picture/12}{12}{fig:12}
    \item \textbf{洛埃镜}:利用直射光与反射光干涉。$\Delta x = \frac{D\lambda}{2\alpha}$
    \insertfigure{picture/2025-12-24-17-37-11.png}{0}{fig:0}
    \textbf{重要现象}:在镜面边缘接触点处(光程差趋于0),出现暗条纹。这直接证明了光在光密介质表面反射时存在半波损失(相位突变 $\pi$)。
\end{itemize}

\subsection{点源位移导致条纹移动}
当点光源 $S$ 偏离光轴时,零级亮纹的位置也会发生改变。
\insertfigure{picture/2025-12-25-15-10-45.png}{0}{fig:0}
\begin{enumerate}
    \item \textbf{现象}:若点光源 $S$ 在垂直于光轴方向上移动 $\delta_s$,则干涉条纹整体将向\textbf{相反}方向移动。
    \item \textbf{定量关系}:
    设光源到双缝平面的距离为 $R$(对应上文 $B$),双缝到观察屏的距离为 $D$(对应上文 $C$)。
    根据光程差公式 $\Delta r = \frac{d}{D}x + \frac{d}{R}\delta_s = 0$(零级亮纹条件),可得条纹位移 $\delta x$:
    \[ \delta x = - \frac{D}{R} \delta_s \]
    \item \textbf{物理意义}:
    \begin{itemize}
        \item 负号表示条纹移动方向与光源移动方向相反。
        \item 位移量被放大了 $D/R$ 倍(干涉系统的放大倍率)。
    \end{itemize}
\end{enumerate}

\subsubsection{空间相干性}
实际光源具有宽度 $b$。
\begin{itemize}
    \item 光源上不同点发出的波列是不相干的,产生的干涉条纹在屏上发生位移。
    \item 当边缘点产生的条纹相对中心点位移达 $\Delta x / 2$ 时,条纹完全模糊。
\end{itemize}
\textbf{相干极限}:
\begin{equation}
    b \cdot \frac{d}{D} \le \lambda \implies b \cdot \theta_s \le \lambda
\end{equation}
其中 $\theta_s \approx d/D$ 为干涉孔径角。
\textbf{结论}:光源宽度越窄,允许的干涉孔径角越大,空间相干性越好。
\insertfigure{picture/2025-12-25-14-07-10.png}{0}{fig:0}
\subsection{分振幅干涉}

\subsubsection{薄膜干涉基本原理}
\textbf{模型}:折射率 $n_2$,厚度 $h$ 的薄膜,置于 $n_1$ 和 $n_3$ 之间。
\textbf{光程差公式}:
\begin{equation}
    \Delta = 2 n_2 h \cos i_2 + \frac{\lambda}{2} (\text{若存在半波损失})
\end{equation}
\insertfigure{picture/2025-12-25-14-08-03.png}{0}{fig:0}
\textbf{半波损失判据}:
当 $n_1 < n_2 < n_3$ 或 $n_1 > n_2 > n_3$ 时,上下表面反射性质相同,无附加项。
当 $n_1 < n_2$ 且 $n_2 > n_3$(如空气中肥皂膜),仅上表面有半波损失,需加 $\lambda/2$。

\subsubsection{等倾干涉}
\textbf{条件}:$h$ 均匀,面光源。
\textbf{规律}:$\Delta$ 仅随入射角 $i$ (或折射角 $i_2$) 变化。
\textbf{图样}:定域于无穷远(焦平面)的同心圆环。
\textbf{吞吐现象}:膜厚 $h$ 增加,光程差增大,中心条纹级次冒出(“吐”),条纹整体外扩变密。

\subsubsection{等厚干涉}
\textbf{条件}:$h$ 不均匀,平行光(或准平行光)入射。
\textbf{规律}:$\Delta$ 仅随厚度 $h$ 变化。条纹描绘了膜厚的等高线。
\textbf{定域}:薄膜表面附近。

\textbf{1. 劈尖}
条纹间距 $L = \frac{\lambda}{2n \sin \alpha} \approx \frac{\lambda}{2n\alpha}$。
应用:检测表面平整度(条纹弯曲度)、测量微小直径。

\textbf{2. 牛顿环}
光程差 $\Delta = 2h + \lambda/2 \approx r^2/R + \lambda/2$。
\begin{itemize}
    \item \textbf{反射光}中心为暗斑(接触点 $h=0, \Delta=\lambda/2$)。
    \item \textbf{透射光}中心为亮斑(无半波损失)。
\end{itemize}
\insertfigure{picture/2025-12-25-14-09-07.png}{0}{fig:0}
\subsubsection{光学薄膜应用}
\begin{itemize}
    \item \textbf{增透膜}:利用干涉相消。单层膜条件:$n_{膜}h = \lambda/4$, $n_{膜} = \sqrt{n_{基}}$。
    \item \textbf{高反膜}:利用干涉相长。多层介质膜堆叠。
\end{itemize}

\subsection{迈克耳孙干涉仪与时间相干性}

\subsubsection{仪器结构与原理}
利用分光板 $G_1$ 分光。补偿板 $G_2$ 保证两臂玻璃光程相等,使仪器能用于白光干涉。
等效为空气层厚度 $d$ 的薄膜干涉。
\begin{itemize}
    \item $M_1, M_2$ 严格垂直 $\to$ 等倾干涉(圆环)。
    \item $M_1, M_2$ 有微小倾角 $\to$ 等厚干涉(直条纹,定域在楔形表面)。
\end{itemize}
\insertfigure{picture/2025-12-25-14-09-49.png}{0}{fig:0}
\subsubsection{时间相干性}
实际光源非单色(谱宽 $\Delta \nu$ 或 $\Delta \lambda$)。
\textbf{波列长度}:原子发光过程持续时间有限(约 $\SI{e-8}{s}$),形成有限长度波列。
\textbf{相干长度} $L_c$:能够发生干涉的最大光程差。
\begin{equation}
    L_c \approx \frac{c}{\Delta \nu} = \frac{\lambda^2}{\Delta \lambda}
\end{equation}
\textbf{白光干涉}:由于白光 $\Delta \lambda$ 很大,$L_c$ 极短(仅几微米)。仅在零级条纹附近($\Delta \approx 0$)可见彩色条纹,中心为黑色(半波损失)。常用于确定“零光程差”位置。

\subsection{多光束干涉}

\subsubsection{法布里-珀罗 干涉仪}
由两块平行的高反射率($R \to 1$)平板组成。
\textbf{原理}:入射光在腔内多次反射,形成振幅递减、相位等差的无穷多束光叠加。
\insertfigure{picture/2025-12-25-14-11-44.png}{0}{fig:0}

\subsubsection{艾里公式}
透射光强 $I_T$ 随相位差 $\delta = \frac{4\pi n h \cos i}{\lambda}$ 的分布:
\begin{equation}
    I_T = I_{\max} \frac{1}{1 + F \sin^2 (\delta/2)}
\end{equation}
\textbf{精细度系数}:$F = \frac{4R}{(1-R)^2}$。
\begin{itemize}
    \item $R$ 越大,$F$ 越极大,条纹极其细锐。
\end{itemize}

\subsubsection{重要应用参数}
\begin{itemize}
    \item \textbf{自由光谱范围}:相邻两个干涉级次之间的波长差。$\Delta \lambda_{fsr} \approx \frac{\lambda^2}{2nh}$。
    \item \textbf{分辨本领}:$\mathcal{R_c} = \frac{\lambda}{\delta \lambda} =\pi K \frac{\sqrt{R}}{1-R}$。
    \item \textbf{半值宽度}:

$$
\left\{
\begin{aligned}
    & \varepsilon = \frac{2(1-R)}{\sqrt{R}} \quad \text{rad} \quad (\text{相位}) \\[10pt]
    & \Delta \theta_k = \frac{\lambda}{2\pi n h \sin\theta_k} \cdot \frac{1-R}{\sqrt{R}} \quad (\text{角}) \\[10pt]
    & \Delta \lambda_k = \frac{\lambda_k^2}{2\pi n h} \cdot \frac{1-R}{\sqrt{R}} \quad \text{nm} \quad (\text{谱})
\end{aligned}
\right.
$$
    \item \textbf{选频(正入射)——频率间隔}:$\Delta v =\frac{c}{2nh}$
\end{itemize}




